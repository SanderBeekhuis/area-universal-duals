%!TEX root = ../thesis.tex

% All results in this chapter hold

\section{Appendix: Additional results on triangulations and triangulations of the k-gon}

\subsection{Connectedness of triangulations}
  Let us first note that any maximally planar graph is $2$-connected. Suppose there is a cutvertex, then surly we can add an edge between the components found after removing this cutvertex.

  \begin{thrm}
    Any plane triangulation $T$ is $3$-connected.
    \label{th:plTri3Connected}
  \end{thrm}

  \begin{proof}
    Suppose that $T$ is not $3$-connected. Then there must be a $2$-cutset $S$, given by the vertices $x$ and $y$. Removing this cutset splits the graph into at least two connected components $C_i$ with all components incident to all cutvertices. Since otherwise we would have found a $1$-cutset.

    Because $S$ is a cutset, there can not be any edges connecting $C_1$ and $C_2$. But then the edge $xy$ should be separating the $2$ components on both sides. This is impossible since we can only draw this edge once.
    %\fxnote{We could add a figure to make this more clear}
  \end{proof}

  \begin{defi}[Irreducible triangulation]
  We call a triangulation irreducible if it has no separating triangles
  \end{defi}

  \fxnote{It is called irreducible because there is a reduction that works on separating triangles. We might show this reduction. predraft-2}

  \begin{thrm}
  Any irreducible plane triangulation $T$ is $4$-connected.
  \end{thrm}

  \begin{proof}
    Note that any plane triangulation is $3$-connected by Theorem \ref{th:plTri3Connected}.

    Suppose that $T$ is not $4$-connected. Then there must be some $3$-cutset (since it is $3$-connected) let us denote the vertices of this cutset by $x, y$ and $z$. Removing this cutset splits the graph into at least two connected components $C_i$ and all components are incident to all cutvertices otherwise we would have found a $2$- or $1$-cutset.

    However, now $xy$ must be an edge in the triangulation $T$ otherwise the graph is not maximal planar (There can not be an edge incident to both $C_1$ and $C_2$ because that would negate $x, y ,z$ being a cutset.). In the same way $yz$ and $xz$ are edges of $T$. But then $xyz$ is a separating triangle. This is an contradiction and thus $T$ is $4$-connected
  \end{proof}

\subsection{Connectedness of triangulations of the $k$-gon}
  \begin{defi}[Irreducible triangulation of the $k$-gon]
  We call a triangulation of the $k$-gon irreducible if it has no separating triangles.
  \end{defi}


  Note that triangulation of the $n$-gon $n\geq 4$ is not maximally planar and thus not plane triangulation.

  The \emph{completion} of a triangulation of the $k$-gon $G = (V, E)$. Is the graph $G'= (V', E')$ with vertex set $V' = V \cup \braces{s}$ and edge set $E' = E \cup \braces{ sv | v \text{ is a outer vertex}}$

  The completion is plane triangulation.  %Q does this stament need proof?
  Since the interior of the outer cycle of $G$ always consisted of faces of degree 3. The exterior of the outer cycle consisted of one face of degree $k$ (the outer face) but the completion has turned this into $k$ faces of degree $3$.

  \begin{thrm}
  A triangulation of the $k$-gon $G$ is $2$-connected.
  \end{thrm}
  \begin{proof}
  Suppose that $G$ has a cutvertex $v$. Then the set $\braces{s, v}$ is a $2$-cutset of the completion $G'$ of $G$. This however is in contradiction to Theorem \ref{th:plTri3Connected} stating that $G'$ is $3$-connected. Hence $G$ has no cutvertex and is thus $2$-connected.
  \end{proof}

  \begin{thrm}
    \label{th:triOfK3:VertexDisjointPaths}
    For every interior vertex $v$ of a triangulation of the $k$-gon $G$ is connected by at least $3$ vertex disjoint paths to different outer vertices.
  \end{thrm}
  \begin{proof}
  By Theorem \ref{th:plTri3Connected} the completion $G'$ of $G$ is $3$-connected. Hence there are 3 vertex-disjoint paths from $v$ to $s$. Since $v$ is on the interior and $s$ is on the exterior of the outer cycle $\C$ all these 3 paths cross the outer cycle at least once. These paths cross $\C$ for the first time in different vertices since they are vertex-disjoint. If we shorten the paths to their first crossing with $\C$ we obtain the $3$ paths in the theorem.
  \end{proof}

  \fxnote{We can sharpen this to $4$ if we have a irreducible triangulation of the $k$-gon with a chordfree outer cycle}
