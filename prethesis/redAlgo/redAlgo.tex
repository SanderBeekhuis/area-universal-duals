%!TEX root = thesis.tex

\documentclass[a4paper]{article}
\usepackage{amsthm}
\usepackage[utf8]{inputenc}
\usepackage{csquotes}
\usepackage[english]{babel}
\usepackage{graphicx}
\usepackage{enumitem}
\usepackage{subcaption}  %ALLOWS SUBFIGURES
\usepackage{wrapfig}

\usepackage[draft]{fixme}
\fxsetup{theme = color}
\definecolor{fxnote}{rgb}{0.0000, 0.6000,0.0000}
\definecolor{fxwarning}{rgb}{1.0000,0.5490,0.0000}
\definecolor{fxerror}{rgb}{1.0000,0.2706,0.0000}
\definecolor{fxfatal}{rgb}{1.0000,0.0000,0.0000}
\usepackage[backend=biber, giveninits =true, isbn=false, url=false, maxbibnames=100]{biblatex}
\usepackage{hyperref}

%Theorems
\newtheorem{thrm}{Theorem}
\newtheorem{lemma}[thrm]{Lemma}
\newtheorem{prop}[thrm]{Proposition}
\newtheorem{remark}[thrm]{Remark}

\theoremstyle{definition}
\newtheorem*{defi}{Definition}

%%BeginIpePreamble
\usepackage{amsmath}
\usepackage{amssymb}
\usepackage{amsopn}

\newcommand{\scr}[1]{\mathcal{#1}}
\newcommand{\Z}{\mathbb{Z}}
\newcommand{\F}{\mathbb{F}}
\newcommand{\R}{\mathbb{R}}
\newcommand{\N}{\mathbb{N}}
\newcommand{\Q}{\mathbb{Q}}


%Operators
\newcommand{\id}{\operatorname{Id}}



%braces etc
\newcommand{\braces}[1]{\left\lbrace {#1} \right\rbrace}
\newcommand{\sqbr}[1]{\left\lbrack {#1} \right\rbrack }
\newcommand{\abs}[1]{\left\lvert {#1} \right\rvert }
\newcommand{\ceil}[1]{\left\lceil{ #1 } \right\rceil}
\newcommand{\floor}[1]{\left \lfloor {#1}\right\rfloor}
\newcommand{\parens}[1]{\left( {#1} \right)}


%utility
\newcommand{\inv}[1]{{#1}^{-1}}
\newcommand{\half}{\frac{1}{2}}
\newcommand{\third}{\frac{1}{3}}
\newcommand{\goes}{\rightarrow}
\newcommand{\nin}{\not \in}
\newcommand{\sm}[1]{\setminus \braces{#1} }

%vectors and matrices
\newcommand{\zerov}{\vec{0}}
\newcommand{\onev}{\vec{1}}

\newcommand{\twovec}[2]{\parens{ \begin{array}{c}#1 \\ #2\end{array} }}
\newcommand{\threevec}[3]{\prens{ \begin{array}{c}#1 \\ #2\\#3 \end{array} }}
\newcommand{\fourvec}[4]{\parens{ \begin{arr\newcommand{\ifftext}{if and only if }ay}{c}#1 \\ #2\\#3\\#4 \end{array} }}
\newcommand{\twomatrix}[4]{\parens{\begin{array}{cc}#1 & #2 \\ #3 & #4 \end{array}  }}
\newcommand{\twodiagmatrix}[2]{\parens{\begin{array}{cc}#1 & 0 \\ 0 & #2 \end{array}  }}

%%%%THIS THESIS
\newcommand{\intplus}{\operatorname{Int^{+}}}
\newcommand{\interior}{\operatorname{Int}}
\newcommand{\spl}{\operatorname{split}}
\newcommand{\mrg}{\operatorname{merge}}


\newcommand{\ext}[1]{\bar{#1}}
\newcommand{\tightext}[1]{\bar{#1}_t}
\newcommand{\dualgraph}[1]{\G(#1)}
\newcommand{\extdualgraph}[1]{\G_{\scr E}(#1)}



\newcommand{\W}{\scr W}
\renewcommand{\P}{\scr P}
\newcommand{\C}{\scr C}
\newcommand\restrict[1]{\raisebox{-.5ex}{$|$}_{#1}}
\newcommand{\restC}[1]{\ensuremath{\C\restrict{#1}}}

%p is for pole
\newcommand{\pN}{\mathrm{N}}
\newcommand{\pS}{\mathrm{S}}
\newcommand{\pE}{\mathrm{E}}
\newcommand{\pW}{\mathrm{W}}

\newcommand{\cpath}{\C \setminus \braces{\pS}} %cycle path

\newcommand{\rel}{\text{regular edge labeling }}

%%EndIpePreamble


%bib stuff
\bibstyle{plain}


\newcommand\restrict[1]{\raisebox{-.5ex}{$|$}_{#1}}

\begin{document}
\maketitle


\newcommand{\W}{\scr W}
\renewcommand{\P}{\scr P}
\newcommand{\restC}[1]{\ensuremath{\C\restrict{#1}}}

\newcommand{\mrN}{\mathrm{N}}
\newcommand{\mrS}{\mathrm{S}}
\newcommand{\mrE}{\mathrm{E}}
\newcommand{\mrW}{\mathrm{W}}

\newcommand{\cpath}{\C \setminus \braces{\mrS}} %cycle path


\restC{\W} versus $\C_\W$ And then $\restC{\W} | \W$ versus $\C_\W|\W$

\paragraph{Notational concerns}
We will use $\C$ to indicate the current sweep line cycle. 
Note that we can consider the path $\C \setminus {\mathrm{S}}$. We will order it from $\mrW$ to $\mrE$. 

We will let $\W$ denote a interior walk  \fxnote{have i defined this already}. Given such a walk of $k$ vertices we index it's nodes $w_1, \ldots, w_k$  in such a way that $w_1$ is closer to $W$ then $w_k$ is (and thus that $w_k$ is closer to $E$ then $w_1$ is). 

Then $w_1$ and $w_k$ indicate the two unique vertices of the walk that are also part of the cycle. We will then let $\restC{\W}$ denote the part of $\C\setminus {\mathrm{S}}$ that is between $w_1$ and $w_k$ (including). $\C_\W$ will denote the closed walk formed when we paste $\restC{W}$ and $\W$.

Since paths are a subclass of walks all of the above notation can also be used for a path $\P$. Note that the closed walk $\C_\P$ in this case will actually be a cycle.


\paragraph{prelim}
\emph{nondistinct corner.}

\section{Outline}
We will show that there is a algorithm if there are no $4$ cycles.

If graph $G$ has non-distinct corners we remove them.


The main algorithm will recieve as input a extended graph $\ext G$ without non-distinct corners and will return a regular edge labeling such that all red faces are $(1-\infty)$ using a sweepcycle approach inspired by \Fusy \fxnote{spelling Fusy and cite} \cite{F}.

We will start by creating a walk $W$. This walk may not be a valid path, it doesn't even have to be a path. During the algorithm we will make a number of moves that will turn this candidate walk into a valid path. In each move we shrink $C$ by employing a valid path and change the candidate walk..

One invariant we will always maintain is that the area bounded by $\C_\W$ will never have interior vertices. \fxnote{What is exactly the area bounded by a closed walk}.

\subsection{The initial candidate walk}
We define the \emph{level} of a vertex of $G$ as the distance of this vertex to the cycle minus $\mrS$ $\cpath$. Let $v_i$ denote all the level $0$ vertices in $G$ in the order that they occur on $\C$  That is $\cpath$ is given by $\mrW v_1 \ldots v_n \mrE$. 
As candidate walk we will start with $W$, we will then take the level $1$ vertices adjacent to $v_1$ in clockwise order, followed by the level $1$ vertices adjecent to $v_2$ in clockwise order (in so far they didn't already occur) and so further until we add the level $1$ vertices adjacent to $v_n$ and finish with $E$.

\fxfatal{To proof:This is a walk
}
\subsection{moves}

The candidate walk can have two kinds of problems. It either is non-simple or it has chords.\fxnote{cf Kusters. Where there are also two problems for a proper boundary path} Otherwise it is a valid path.



\end{document}