%!TEX root = thesis.tex

\documentclass[a4paper]{article}
\usepackage{marvosym}
\usepackage{a4wide}                     % Tell latex to use the room there is on the page
\usepackage{fancyhdr}                   % Nice and shiny headers
\usepackage{paralist}                   % Compact enumerate and itemize
\usepackage{datetime}                   % For formatting dates
\usepackage{makeidx}                    % Needed for the index
\usepackage{hyperref}                   % For inserting PDF options (author etc.)

\usepackage{amsmath}
\usepackage{amsthm}
\usepackage[utf8]{inputenc}
\usepackage{csquotes}
\usepackage[english]{babel}
\usepackage{graphicx}
\usepackage{enumitem}
\usepackage{subcaption}  %ALLOWS SUBFIGURES
\usepackage{wrapfig}
\usepackage{fnpct}
\usepackage{fixme}
\fxsetup{theme = color}
\definecolor{fxnote}{rgb}{0.0000, 0.6000,0.0000}
\definecolor{fxwarning}{rgb}{1.0000,0.5490,0.0000}
\definecolor{fxerror}{rgb}{1.0000,0.2706,0.0000}
\definecolor{fxfatal}{rgb}{1.0000,0.0000,0.0000}
\newcommand{\fxinnote}[1]{\fxnote[inline, nomargin]{#1}}

\usepackage[backend=biber, giveninits =true, isbn=false, url=false, maxbibnames=100]{biblatex}
\usepackage{hyperref}

\usepackage{accents}





% For the TUe style...
% Use the package 'textpos' with the option 'absolute' to place
% a textbox a given distance from the edge of the page
\usepackage[absolute]{textpos}


%
% Layout
%

% header!

%Common header attributes - see also http://en.wikibooks.org/wiki/LaTeX/Page_Layout
\fancyfoot[L]{\doctitle}
\fancyfoot[C]{}
%\fancyfoot[RE,LO]{\me}
\fancyfoot[R]{\thepage}
\renewcommand{\footrulewidth}{0.4pt} %thickness of the decorative lines on the footer

% Headers for the pages that start a chapter, also for the TOC
\fancypagestyle{plain}{
\fancyhead[R]{}
\fancyhead[L]{}
\renewcommand{\headrulewidth}{0pt} %thickness of the decorative lines on the header
}

%Header for sections
\fancypagestyle{FooBar}{%
    \fancyhead{}
    \renewcommand{\headrulewidth}{1pt}
}

% Headers for the other pages
\pagestyle{fancy}{
\fancyhead[R]{\slshape \leftmark} % \rightmark = represent section heading
\fancyhead[L]{\slshape } % \leftmark = represent chapter heading
}

%Theorems
\newtheorem{thrm}{Theorem}
\newtheorem{con}[thrm]{Conjecture}
\newtheorem{lemma}[thrm]{Lemma}
\newtheorem{prop}[thrm]{Proposition}
\newtheorem{remark}[thrm]{Remark}
\newtheorem{observation}[thrm]{Observation}

\theoremstyle{definition}
\newtheorem*{defi}{Definition}

%%BeginIpePreamble
\usepackage{amsmath}
\usepackage{amssymb}
\usepackage{amsopn}

\newcommand{\scr}[1]{\mathcal{#1}}
\newcommand{\Z}{\mathbb{Z}}
\newcommand{\F}{\mathbb{F}}
\newcommand{\R}{\mathbb{R}}
\newcommand{\N}{\mathbb{N}}
%\newcommand{\Q}{\mathbb{Q}}


%Operators
\newcommand{\id}{\operatorname{Id}}



%braces etc
\newcommand{\braces}[1]{\left\lbrace {#1} \right\rbrace}
\newcommand{\sqbr}[1]{\left\lbrack {#1} \right\rbrack }
\newcommand{\abs}[1]{\left\lvert {#1} \right\rvert }
\newcommand{\ceil}[1]{\left\lceil{ #1 } \right\rceil}
\newcommand{\floor}[1]{\left \lfloor {#1}\right\rfloor}
\newcommand{\parens}[1]{\left( {#1} \right)}


%utility
\newcommand{\inv}[1]{{#1}^{-1}}
\newcommand{\half}{\frac{1}{2}}
\newcommand{\third}{\frac{1}{3}}
\newcommand{\goes}{\rightarrow}
\newcommand{\nin}{\not \in}
\newcommand{\sm}[1]{\setminus \braces{#1} }

%vectors and matrices
\newcommand{\zerov}{\vec{0}}
\newcommand{\onev}{\vec{1}}

\newcommand{\twovec}[2]{\parens{ \begin{array}{c}#1 \\ #2\end{array} }}
\newcommand{\threevec}[3]{\prens{ \begin{array}{c}#1 \\ #2\\#3 \end{array} }}
\newcommand{\fourvec}[4]{\parens{ \begin{arr\newcommand{\ifftext}{if and only if }ay}{c}#1 \\ #2\\#3\\#4 \end{array} }}
\newcommand{\twomatrix}[4]{\parens{\begin{array}{cc}#1 & #2 \\ #3 & #4 \end{array}  }}
\newcommand{\twodiagmatrix}[2]{\parens{\begin{array}{cc}#1 & 0 \\ 0 & #2 \end{array}  }}

%%%%THIS THESIS
\newcommand{\intplus}{\operatorname{Int^{+}}}
\newcommand{\interior}{\operatorname{Int}}
\newcommand{\spl}{\operatorname{split}}
\newcommand{\mrg}{\operatorname{merge}}


\newcommand{\ext}[1]{\bar{#1}}
\newcommand{\tightext}[1]{\bar{#1}_t}
\newcommand{\dualgraph}[1]{\G(#1)}
\newcommand{\extdualgraph}[1]{\G_{\scr E}(#1)}

\newcommand{\mypar}[1]{\medbreak \noindent {\bfseries #1.}}
\renewcommand{\paragraph}{\mypar}
\newcommand{\rev}[1]{\accentset{\leftharpoonup}{#1}}



\newcommand{\W}{\scr W}
\renewcommand{\P}{\scr P}
\newcommand{\C}{\scr C}
\newcommand{\Q}{\scr Q}
\renewcommand{\L}{\scr L}
\newcommand{\I}{\scr I}

\newcommand{\G}{\scr G}
\newcommand\restrict[1]{\raisebox{-.5ex}{$|$}_{#1}}
\newcommand{\restC}[1]{\ensuremath{\C\restrict{#1}}}

%p is for pole
\newcommand{\pN}{\mathrm{N}}
\newcommand{\pS}{\mathrm{S}}
\newcommand{\pE}{\mathrm{E}}
\newcommand{\pW}{\mathrm{W}}

\newcommand{\cpath}{\C \setminus \braces{\pS}} %cycle path
%%EndIpePreamble


%invariant environment
\newenvironment{invariants}{%
  \refstepcounter{thrm}%
  {\bfseries Invariants~\theprop}%
  \renewcommand*{\theenumi}{\theprop\,(I\arabic{enumi})}%
  \renewcommand*{\labelenumi}{(I\arabic{enumi})}%
  \enumerate
}{%
  \endenumerate
}

%bib stuff
\bibstyle{plain}

%%Chapter headings
\usepackage{titlesec}

\newfont{\chapterNumber}{wncysc10 scaled 7000}

\titleformat{\section}[display]%
{\relax}{\mbox{}\hspace*{13.5cm}\vspace*{-2\baselineskip}\color{lightgray}\chapterNumber\thesection}{0pt}%
	{\LARGE\itshape}[\normalsize\vspace*{.8\baselineskip}\titlerule]%

\titlespacing*{\section}{0pt}{0cm}{1cm}

\titleformat{\subsection}{\Large}{}{0em}{\makebox[0cm][r]{\thesubsection\hspace{1em}}\scshape\lowercase}[\titlerule]
\titlespacing*{\subsection}{0pt}{\baselineskip}{\baselineskip}
\titleformat{\subsubsection}{\large}{}{.6em}{\thesubsubsection \hspace{.5em} \itshape}
\titlespacing*{\subsubsection}{0pt}{\baselineskip}{.5\baselineskip}



\begin{document}
\maketitle

%Depracted
\newcommand{\mrN}{\mathrm{N}}
\newcommand{\mrS}{\mathrm{S}}
\newcommand{\mrE}{\mathrm{E}}
\newcommand{\mrW}{\mathrm{W}}


\paragraph{Notational concerns}
We will use $\C$ to indicate the current sweep line cycle. 
We will repeatedly only consider the path $\cpath$. In that case we will always order it from $\mrW$ to $\mrE$. 

We will let $\W$ denote a interior walk  \fxnote{have i defined this already}. Given such a walk of $k$ vertices we index it's nodes $w_1, \ldots, w_k$  in such a way that $w_1$ is closer to $W$ then $w_k$ is (and thus that $w_k$ is closer to $E$ then $w_1$ is). 

Then $w_1$ and $w_k$ indicate the two unique vertices of the walk that are also part of the cycle. We will then let $\restC{\W}$ denote the part of $\C\setminus {\mathrm{S}}$ that is between $w_1$ and $w_k$ (including). $\C_\W$ will denote the closed walk formed when we paste $\restC{W}$ and $\W$.

Since paths are a subclass of walks all of the above notation can also be used for a path $\P$. Note that the closed walk $\C_\P$ in this case will actually be a cycle.


\paragraph{prelim}
\emph{nondistinct corner.}

\emph{chordfree path}

Chords to the left/right of a path

\begin{lemma}
If a boundray path is without chords adding a pole to it will not create a sep triangle (cf. Yeap)
\end{lemma}
\fxnote{to prove}

\subsection{Outline}
We will show that there is a algorithm if there are no separating $4$-cycles in $G$ and no separating $3$-cycles in $\ext G$.

If graph $G$ has non-distinct corners or cutvertices or it is empty we treat them separately and recurse on a smaller graph. \fxnote{TODO}


The main algorithm will receive as input a extended graph $\ext G$ without non-distinct corners and no separating $4$ cycles and will return a regular edge labeling such that all red faces are $(1-\infty)$ using a sweep-cycle approach inspired by \Fusy \fxnote{spelling Fusy and cite} \cite{Fusy2006}.

We will start by creating a walk $W$. This walk may not be a valid path, it doesn't even have to be a path. During the algorithm we will make a number of moves that will turn this candidate walk into a valid path. In each move we shrink $C$ by employing a valid path and change the candidate walk.

One invariant we will always maintain is that the area bounded by $\C_\W$ will never have interior vertices. \fxnote{What is exactly the area bounded by a closed walk}.

\subsection{Treating nondistinct corners and cutvertices of $G$}
-Still to be written -

\subsection{The initial candidate walk}
\fxerror{change this to handle nondistinct corners}
Let $v_i$ denote all the vertices of $\C \setminus \braces{\pW, \pS, \pE}$ in the order that they occur on $\cpath$.  That is $\cpath$ is given by $\mrW v_1 \ldots v_n \mrE$. 
As candidate walk we will start with $\pW$, we will then take the vertices adjacent to $v_1$ between $\pE$ and $v_2$ in clockwise order (exclusive), followed the vertices adjacent to $v_2$ between $v_1$ and $v_3$ in clockwise order and so further until we finally add the vertices adjacent to $v_n$ between $v_{n-1}$ and $\pE$ in clockwise order and finally we finish by adding $E$.

\begin{lemma}
After removing subsequent duplicates the collection $W$ described above is indeed a walk.
\end{lemma}

\fxnote{introduce a term for "edges subsequent to each other in clockwise order around $v$"}

\begin{proof}
To show that $W$ is a walk it's sufficient to show that every vertex is adjacent to the next vertex. Let us suppose that $w$ and $w'$ are two subsequent vertices in $W$, we will show that they are connected if $\braces{w, w'} \cap \braces{\pW, \pE} = \emptyset$ after that we will consider this edge case. There are then two main case for $w, w'$. Either $(a)$ $w$ and $w'$ are  vertices adjecent to some $v_i$ subsequent in clockwise order or $(b)$ $w$ was the last vertex adjacent to some $v_i$ and thus $w'$ is the first vertex adjacent to $v_{i+1}$.

The following two situations can also be seen in Figure \ref{fig:walkproof}.

\begin{figure}
    \centering
    \begin{subfigure}[b]{0.5\linewidth}
        \includegraphics[width=\linewidth]{img/walkProofA}
        \caption{}
    \end{subfigure}%
    \begin{subfigure}[b]{0.5\linewidth}
        \includegraphics[width=\linewidth]{img/walkProofB}
        \vspace{1cm}   
                
        \caption{}
    \end{subfigure}

    	\caption{The two main cases of the proof showing that $W$ is a walk after removing duplicates.}
	\label{fig:walkproof}
\end{figure}


In case $(a)$ we note that $v_i w$ and $v_i w'$ are edges next to each other in clockwise order around $v_i$. Since every interior face of $\ext G$ is a triangle $ww'$ must be an edge. We thus see that $w, w'$ are adjacent and not duplicates.

In case $(b)$ we note that $v_i w$ and $v_i v_{i+1}$ are edges subsequent in clockwise order, hence $wv_{i+1}$ is also an edge. Hence $w$ is the first vertex adjacent to $v_{i+1}$ after $v_i$ in clockwise order. Thus $w= w'$, they are duplicates and we will remove $w$.

Now for the edge cases: Let $w_1$ be the first vertex adjacent to $v_1$ and let $w_m$ be the last vertex adjacent to $v_n$. $\pW$ and $w_1$ are vertices adjacent to $v_1$ subsequent in clockwise order, and hence connected. $w_m$ and $\pE$ are vertices adjacent to $v_n$ subsequent in clockwise order and hence connected. 
\end{proof}


\subsection{Irregularities}

We will distinguish two kinds of \emph{irregularties} on the candidate walk.
\begin{enumerate}
\item The candidate walk is non-simple in a certain vertex. That is, if we traverse the sequence of vertices in $\W$ we see that $w_i = w_j$ for some $i<j$. 
\item The candidate walk has a chord. That is, there is an edge $w_i w_j$ in $G$ with $i<j$ and $i$ and $j$ not subsequent (i.e. $i < j-1$). 

Note that such a chord can only lie on the right of $\W$ ($\W$ being oriented from $\pW$ to $\pE$), since if it would lie on the left of $\W$ the vertices $w_{i+1},\ldots, w_{j-1}$ would not have been chosen by the construction.  
\end{enumerate}

\begin{lemma}
If a candiate walk has no irregularities it is a valid path.
\end{lemma}
\begin{proof} \fxnote{refer by labels instead of text}
We will show that all the requirements of being a valid path are met.
 \begin{itemize}
\item [Path] Let us begin by noting that since there are no non-simple points we actually have a path and not just a walk.

\item[(E1)] It is clear that both $w_1$ and $w_k$ are not $\pS$ by the construction of the candidate walk.

\item[(E2)] For $\W$ or $\restC \W$ to have only one edge we need to have that $\pW \pE$ is an edge (since $\W$ is constructed as walk from $\pW$ to $\pE$). However, then one of the $3$-cycles $\pW \pE \pN$ or $\pW \pE \pS$ is separating since the graph $G$ is non-empty. Hence both $\W$ and $\restC \W$ have more than one edge.

\item[(E3)] There are no interior edges in $\C_{\W}$ that are adjacent to $w_1 =W$ or $w_k = E$ since $w_2 v_1$ and $w_{k-1} v_n$ are edges in $\ext G$. This can be seen in the construction of the path. But it is also enough to realize that $w_1=\pW$ and $w_2$ are subsequent neighbors in the clockwise order around $v_1$. The same holds for $w_{k-1}$ and $w_k=\pE$ around $v_n$. 

Furthermore there are no interior edges with both vertices adjacent to $\restC{\W}$ because these edges would be chords offending Invariant \ref{} \fxnote{fix ref}. 

Finally there are no interior edges with both adjacencies to vertices in $\W$ because $\W$ has no chords since it has no irregularities.

\item[ (E4)] The cycle $\C'$ is simply $\pS \W$ (since $\W$ is  walk from $\pW$ to $\pE$ by construction) and since $\W$ has no chords $\C'$ has none not involving $\pS$.
\end{itemize}
Hence, if $\W$ has no irregularities it is a valid path. 
\end{proof}

\begin{defi}[Range of a irregularity]
For a non-simple point \fxnote{Is point the right word? it is def not a vertex} $w_i = w_j$ with $i<j$ has \emph{range} $\braces{i, \ldots, j} \subset \N$.
A chord $w_i w_j$ with $i< j-1$ has \emph{range} $\braces{i, \ldots, j} \subset \N$.
\end{defi}

Note that a chord can't have the same range as a non-simple point since then $w_i w_j$ will be a loop and we are considering simple graphs. Furthermore two chords have different ranges because we otherwise have a multiedge. Two nonsimple points with the same range are, in fact, the same. This leads us to the following remark. 
\begin{remark}
\label{rk:diffIregDiffRange}
Different irregularities have different ranges.
\end{remark}

\begin{defi}[Maximal irregularity]
A irregularity is maximal if it's range is not strictly contained\footnote{Because of Remark \ref{rk:diffIregDiffRange} being contained is the same as being strictly contained} in the range of any other irregularity.
\end{defi}

\begin{lemma}
Maximal irregularities have ranges whose overlap is at most one integer.
\fxnote{we might redifine range to make this nicer. However it may make the rest of the algo more ugly. Revisit}
\end{lemma}
\begin{proof}
We let $I$ and $J$ denote two distinct maximal irregularities with ranges $\braces{i_1, \ldots i_2}$ and $\braces{j_1, \ldots, j2}$. Let us for the moment suppose that $I$ and $J$ have ranges that overlap more then one. Since $I$ and $J$ are both maximal their ranges can not be strictly contained in each other and by Remark \ref{rk:diffIregDiffRange} they can't be equal. Hence the ranges must partially overlap. 

Without loss of generality we then have $i_1 < j_1 < i_2 < j_2$. Any additional equality in this chain would offend the ranges not being contained in each other or the overlap being larger then one integer.

\fxnote{I need to note somewhere that every vertex in the candidate walk is adjacent to a subpath of $\cpath$}
Now two chords, both laying to the right of $\W$, would cross each other in this case (but we have a planar graph).

A non-simple point $w_{i_1} = w_{i_2}$ is adjacent to two ranges of vertices in $\cpath$. $v_a \ldots v_b$ and $v_c \ldots v_d$ we need that $b$ and $c$ are not subsequent otherwise we have a separating $3$ cycle $w_{i_1} v_b v_c$, now however $\tilde(C) = w_{i_1} v_b \ldots v_c$ is a cycle. And because of the clockwise order of adjacencies around the vertices of $\cpath$ we have that $w_{i_1 +1}, \ldots, w_{i_2 -1}$ are inside this cycle while $ w_1 \ldots w_{i_1 -1}$ and $ w_{i_2 +1} \ldots w_k$ are outside the cycle. See Figure.  \fxfatal{add figure}

Now $J$ being a chord will imply a edge crossing $\tilde{C}$, which can't be. The same argumentation holds symmetrically for $J$ being a non-simple point and $I$ being a chord. Two nonsimple points would imply that the vertex $w_{j_i} = w_{j_2}$ is at the same time inside and outside $\tilde{C}$. 
\end{proof}

\subsection{Moves}
\newcommand{\U}{\scr U}
We will now show how to remove these maximal irregularities. These maximal irregularities don't influence each other because their ranges only overlap at most one. Other irregularities contained in such a maximal irregularity are solved in the recurrence. 

\subsubsection{Chords}
If we encounter a chord we will extract a subgraph and recurse on this subgraph. A chord $w_iw_j$ has a triangular face on the left and on the right (like every edge). The third vertex in the face to the left will be called $x$. $x$ is not necessarily distinct from $w_{i+1}$ and/or $w_{j-1}$ but that is also not necessary for the rest of the argument. \fxnote{work out these in a example}

The vertex $v_a$ on the cycle is uniquely determined as the vertices adjacent to both $w_i$ and $w{i+1}$. In the same way $v_b$ is the unique neighbor of $w_{j-1}$ and $w_j$.

We will describe a path $\scr U$ running from $v_a$ to $v_b$. This path consists of all vertices adjacent to $w_i$ in clockwise order from $v_a$ (inclusive) to $x$(inclusive) and subsequently all vertices adjacent to $w_j$ in clockwise order from $x$ (exclusive) to $v_b$ (inclusive). This path is given in bold in Figure \ref{fig:removeChord}. 

\begin{lemma}
$\U$ is in fact a path, moreover it has no chords.
\end{lemma}
\begin{proof}
\fxnote{restructure}
$\U$ cant have a non-simple point $x'$ since it would have to be connected to at least two vertices. However a vertex $x'$ that is distinct from $x$ and is connected to both $w_i$ and $w_j$ will induce a separating triangle $w_i x' w_j$. 
\fxnote{make "connected" more precise: connected where?}

$\scr U$ can't have chords $u_i u_j$ since they would either induce a separating $3$- or $4$-cycle either $w_i u_i u_j$ or $w_j u_i u_j$ or $w_i u_i u_j w_j$ depending on the vertex adjacent to $u_i$ and $u_j$.
\fxnote{Here we use no 4-cycles} 
\end{proof}


We then consider the interior of the cycle $\C_\scr U$ and the cycle $\C_\U$ itself as the subgraph $H$. We then set $v_a = \pW$ and $v_b = \pE$ and we connect all vertices in $\restC{\U}$ to a new north pole $\pN$ and all vertices in $\U$ to a new south pole $\pS$. We then arrive at the graph $H'$ upon which we will recurse. See also Figure \ref{fig:removeChord}. Since $\C$ is chordfree by invariant \ref{I} \fxnote{ref} so is $\restC{\U}$. We have also just shown that $\U$ is chordfree. Hence adding the north and south pole doesn't create seperating triangles. Furthermore since $H$ is a induced subgraph of $G$ it contains no seperating $4$-cycles not involving the poles.    



\begin{figure}[h!]
\centering
\includegraphics[scale=1]{img/removeChord}

\caption{Removing a chord 
    \label{fig:removeChord}}
\end{figure}

\subsubsection{Nonsimple points}

Removing a non-simple point is done is a similar manner.


The vertex $v_a$ on $\C$ is uniquely determined as the vertices adjacent to both $w_i=w_j$ and $w{i+1}$. In the same way $v_b$ is the unique neighbor of $w_{j-1}$ and $w_j=w_i$. Note that it may be that $w_{i+1} = w{j-1}$ this does not matter for the rest of the argument. \fxnote{show example}

We will describe a path $\scr U$ running from $v_a$ to $v_b$. This path consists of all vertices adjacent to $w_i = w_j$ in clockwise order from $v_a$ (inclusive) to $v_b$(inclusive). This path is given in bold in Figure \ref{fig:removeNonSimplePoint}. 

\begin{lemma}
$\U$ is in fact a path, moreover it has no chords.
\end{lemma}
\begin{proof}
\fxnote{restructure}
$\U$ cant have a non-simple point $x$ since such a point would have to be connected to at least two vertices. However every vertex is only connected to $w_i=w_j$.
\fxnote{make "connected" more precise: connected where? On the right of the path}

$\U$ can't have chords on the right of the path by construction. Furthermore $\scr U$ can't have chords $u_i u_j$ on the left since they would either induce a separating $3$-cycle $w_i u_i u_j$.
\fxnote{Here we use no 4-cycles} 
\end{proof}


We then consider the interior of the cycle $\C_\scr U$ and the cycle $\C_\U$ itself as the subgraph $H$. We then set $v_a = \pW$ and $v_b = \pE$ and we connect all vertices in $\restC{\U}$ to a new north pole $\pN$ and all vertices in $\U$ to a new south pole $\pS$. We then arrive at the graph $H'$ upon which we will recurse. See also Figure \ref{fig:removeChord}. Since $\C$ is chordfree by invariant \ref{I} \fxnote{ref} so is $\restC{\U}$. We have also just shown that $\U$ is chordfree. Hence adding the north and south pole doesn't create separating triangles. Furthermore since $H$ is a induced subgraph of $G$ it contains no separating $4$-cycles not involving the poles.   


\begin{figure}[h!]
\centering
\includegraphics[scale=1]{img/removeNonSimplePoint}

\caption{Removing a non-simple point 
    \label{fig:removeNonSimplePoint}}
\end{figure}


\section{TODO}
Cool examples: The multiple non-simple point $v_i = v_j =v_k$
Example of page $F1$

example with lots of chords
\end{document}