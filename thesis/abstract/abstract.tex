%!TEX root = ../thesis.tex

\section*{Abstract}
\thispagestyle{plain}

  A  \emph{rectangular layout} (or simply \emph{layout}) $\L$ is a partition of an axis-parallel rectangle into a finite set of interior-disjoint axis-parallel rectangles. Hence, the interior of this rectangle contains vertical and horizontal line segments. Any line segment that can not extended farther on either side is a \emph{maximal segment}. A rectangular layout is \emph{one-sided} if every maximal segment has only one rectangle on one of its sides and \emph{$k$-sided} if every maximal segment has at most $k$ rectangle on one of its sides.

  All graphs in this thesis will be \emph{triangulations of the $k$-gon}. They have an outer face of degree $k$ and interior faces of degree $3$.
  Vertices bordering the outer face are \emph{outer vertices} while all other vertices are \emph{interior vertices}.

  Two vertices are \emph{adjacent} when they are connected by an edge. Two rectangles are \emph{adjacent} when their boundaries overlap. A \emph{rectangular dual} of $G$ is a rectangular layout whose adjacencies are the same as those of $G$ for a bijection between vertices and rectangles.

  A \emph{corner assignment} $\ext G$ of $G$ is an augmentation of $G$ with $4$ vertices ,which we call its \emph{poles}, such that every interior face has degree $3$, the exterior face has degree $4$ and all poles are incident to the outer face.


  It is known that a triangulation of the $k$-gon $\G$ has a rectangular dual if and only if it has a corner assignment without separating triangles $\ext \G$. \fxnote{Suddenly sepearting triangle and equivalent}
  A graph $G$ can have multiple rectangular duals. $G$ can even have duals that are not equivalent.

  %In for example atlases \emph{rectangular cartograms} are used to display information. A rectangular cartogram is a map where the regions are replaced by rectangles while keeping their adjacencies. The size of each region changes according to the variable displayed in the cartogram.  A rectangular cartogram is the rectangular dual of the adjacency graph of the map $G$.
  %If the areas change it might be that a certain rectangular layout can not fulfill its adjacencies anymore and we have to switch to another non-equivalent rectangular dual of $G$.

  %We would like to find a rectangular dual that has adjacencies that hold regardless of the area sizes we assign to each rectangle. We say such a dual is \emph{area-universal}.
  %Eppstein et al. have shown that rectangular duals are area-universal exactly when they are one-sided.~\cite{Eppstein2012} Unfortunately not all graphs admit a one-sided dual. One such graph is given by Rinsma.~\cite{Rinsma1987} \fxwarning{TODO figure of this graph with ciation}

  %Unfortunately $k$-sided layouts for $k>1$ are not area-universal but we suspect that for small $k$ they are more robust to changes in the areas of their rectangles.
  A graph is $k$-sided when it has a $k$-sided rectangular dual. This thesis presents two results on $k$-sided graphs.

  \begin{enumerate}
    \item There is a family of graphs $G_k$ that for any $k \in \N$ has members that are not $k$-sided (Theorem \ref{fix:th:family}).
    \item Graphs $G$ that have a corner assignment without separating 4-cycles are $d-1$-sided, where $d$ is the maximal degree of the vertices of $G$. (Theorem \ref{th:dsided})
  \end{enumerate}
