\section{Future research question}

\mypar{Things we must still do}
-realisation of the worst case
-example execution


\mypar{Things we can still add to the thesis}
- A real proof that the algo by Fusy does not get stuck
- a vertically one-sided algo (just stop halfway the full algorithm)
- What's going on in a cycle? How does length and coloring influence the indside? (i.e. theory of counting forwards and backwards)
  -How about the contents of a monochromatic cycle

\mypar{easy}
Show that a Regular edge labeling does not admit any directed cycles (and not only no monochromatic directed cycles).

\mypar{hard}
- A horizontally one-sided algo may provide more insight.
-maybe one can do something with a planar separartor theorhem.
- Is for every graph without a separating 4-cycle 2-sided possible, or k-sided for any k?

\mypar{Possible misdirection}
- Can we prevent the cases with a split vertex exaclty on the end of the left outer edge of a split vertex?
      - Maybe for example by forbiding any 3-chords? I dont think this works in a straightforward manner

-One can do one type of topfan flip on one half of the graph and the other type on the other half of the graph. That goes okay, it is just constant switching that's difficult. (i.e. we cant switch back)
