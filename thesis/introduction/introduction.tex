%!TEX root = ../thesis.tex

%What does each section contain in one sectence?
%\fxwarning{Work this out in predaft-2}

\section{Introduction}

\mypar{Rectangular layout}
  A  \emph{rectangular layout} (or simply \emph{layout}) $\L$ is a partition of a rectangle into a finite set of interior-disjoint rectangles. Hence the interior of this rectangle contains vertical and horizontal line segments. We will call any such line segment that is not extended any further on either side a \emph{maximal segment}. Such an layout is \emph{one-sided} if every maximal segment has only one rectangle on one of its sides.

  We say two layouts are  \emph{combinatorially equivalent} or simply \emph{equivalent} when their rectangles have the same adjacencies with the same orientation(horizontal or vertical) between their rectangles.

\mypar{Graphs}
  A \emph{graph} $G$ is an abstraction for a network. The objects are represented by a set of \emph{vertices}. Connections are represented by \emph{edges}. Each edge connects two \emph{vertices}. All graphs are \emph{simple}. That is, every pair of vertices is connected by at most one edge and there are no edges starting and ending at the same vertex. An edge is \emph{incident} to a vertex $v$ if that edge connects $v$ to another vertex. The \emph{degree} of a vertex is the number of edges incident to this vertex.
  All graphs in this thesis are \emph{planar}. That is, they can be embedded in the plane without crossings in their edges. A \emph{face} is connected component of the maximal subset of the plane that is disjoint from the embedded graph. The \emph{degree} of a face is the number of vertices on its boundary. A \emph{triangular} face is a face of degree $3$. One and only one of the faces will be unbounded, we call this face the \emph{outer face}.

  Not only are all graphs in this thesis planar, they are all \emph{triangulations of the $k$-gon}. Their outer face has degree $k$ and all interior faces have degree $3$.
  Vertices bordering the outer face are \emph{outer vertices} while all other vertices are \emph{interior vertices}. Furthermore, the cycle formed by all vertices outer vertices is the \emph{outer cycle}.
  Triangulations of the $k$-gon are called \emph{(plane) triangulated graphs} by some authors.

\mypar{Rectangular duals}
  Two vertices are \emph{adjacent} when they are connected by an edge. Two rectangles are \emph{adjacent} when their boundaries overlap. A \emph{rectangular dual} of $G$ is a rectangular layout whose adjacencies are the same as those of $G$ for a bijection between vertices and rectangles.
  Note that a graph $G$ can have multiple rectangular duals. $G$ can even have duals that are not equivalent.

\mypar{Corner assignments}
  If we want to consider which graphs do have a rectangular dual then we need to introduce the notion of a \emph{corner assignment}.
  \fxinnote{Q: This page has annoying typesetting, what to do?}
  \begin{defi}[Corner assignment]
    A corner assignment $\ext G$ of $G$ is a augmentation of $G$ with $4$ vertices (which we call its \emph{poles}). Such that
    \begin{enumerate}
    \item every interior face has degree $3$ and the exterior face has degree $4$.
    \item all poles are incident to the outer face
    \end{enumerate}
  \end{defi}

  A corner assignment of $G$ only exists if $G$ is a triangulation of the $k$-gon for some $k$. Otherwise there is no way of adding poles that makes all the interior faces of degree $3$. Note furthermore that after the corner assignment $\ext G$ is a triangulation of the $k$-gon. Each corner assignment fixes which of the rectangles are in the corners of the rectangular dual $\L$.

  The following result on the existence of rectangular dual is due to Kozminski and Kinnen

  \begin{thrm}[Existence of a rectangular dual]
    \label{th:rect:exsitenceREctangularDual}
    A plane triangulated graph $\G$ has a rectangular dual if and only if it has an corner assignment without separating triangles $\ext \G$
    \fxnote{Q how should this be done differently, p9}
  \end{thrm}

  \begin{proof}
    This is shown independently in \cite{Kozminski1984} and  \cite{Ungar1953}
  \end{proof}

\mypar{Rectangular cartograms}
  In for example atlases \emph{rectangular cartograms} are used to display information. A rectangular cartogram is a map where the regions are replaced by rectangles while keeping their adjacencies. The size of each region changes according to the variable displayed in the cartogram.  A rectangular cartogram is the rectangular dual of the adjacency graph of the map $G$.
  \fxnote{Figure of cartogram, where to find. Pres of speckman?}
  If the areas change it might be that a certain rectangular layout can not fulfill its adjacencies anymore and we have to switch to another non-equivalent rectangular dual of $G$.

  It would be better to find a rectangular dual that has adjacencies that hold regardless of the area sizes we assign to each rectangle. We say such a dual is \emph{area-universal}.
  Eppstein et al. have shown that rectangular duals are area-universal exactly when they are one-sided.~\cite{Eppstein2012} Unfortunately not all graphs admit a one-sided dual. One such graph is given by Rinsma.~\cite{Rinsma1987} \fxnote{Q figure of this graph??}

\mypar{Results}
  A rectangular layout $\L$ is \emph{$k$-sided} if all maximal segments have at most $k$ rectangles on one of their sides. This is a generalization of one-sidedness. A $k$-sided layout is not area-universal but it is more robust to changes in the areas of it's rectangles. This thesis has two results relating to this concept of $k$-sidedness.
  \begin{enumerate}
    \item There are triangulations of the $k$-gon that admit only $\infty$-sided rectangular duals
    \item Triangulations of the $k$-gon $G$ that have a corner assignment without separating 4-cycles are $d$-sided, where $d$ is the maximal degree of the vertices of $G$.
  \end{enumerate}

\mypar{Structure}
  \fxinnote{Q: Do we need to declare structure explicitly?}
