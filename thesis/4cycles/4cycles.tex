%!TEX root = ../thesis.tex

\newcommand{\D}{\scr D}

\mypar{notation}

We note that the interior of some cycle $\interior(C)$ are all vertices strictly in the interior of this cycle. We will sometimes also take $\interior(C)$ to refer to the induced subgraph of these vertices.

We let $\intplus(C)$ denote the the vertices of $C$ and $C$'s interior vertices. We will also sometimes let it refer to the subgraph of $G$ induced by these vertices.


\mypar{coloring}
In this section we will use lots of figures to demonstrate how to handle each type of $4$-cycle.

\mypar{Usefull lemma's and notions}
All edges adjacent to the same exterior vertex of $\D$ in the interior of $\D$ have the same color and orientation.

Once we have chosen a direction and color for one such exterior vertex this choice follows for the rest of the exterior vertices.

Hence it is trivial to recurse on a $4$-cyles once we haven chosen edge colors.

\section{Seperating 4 cycles}
\thispagestyle{plain}
Let $\D$ be a maximal separating $4$-cycle.

Note that the only problem is given by $4$-cycles that are entirely inside the cycle $\C$ maintained by the algorithm. If $\C$ is currently crossing $\D$ then the is not any longer a problem.

We can discern $7$ types of adjacency for $4$-cycles to a cycle if it is entirely inside some cycle.
\begin{enumerate}
  \renewcommand*{\labelenumi}{(\alph{enumi})}%
  \renewcommand*{\theenumi}{(\alph{enumi})}%

  \item $\D$ has $1$ edge on the cycle
  \label{t:1}
  \item $\D$ has 2 consecutive edges on the cycle
  \label{t:2cons}
  \item 2 non-consecutive edges on the cycle
  \label{t:2alt}
  \item 3 edges on the cycle
  \label{t:3}
  \item Just a vertex on the cycle
  \label{t:v1}
  \item Two consecutive vertices on the cycle
  \label{t:v2cons}
  \item Two non-consecutive vertices on the cycle
  \label{t:v2alt}
\end{enumerate}

We will show by case distinction that everything will be okay. Sometimes we will make a move (updating cycle and prefence) to remove a irregularity and sometimes this will not be necessary.



We know that $\D \subseteq \intplus(\C)$. Either $\D \cap \C \neq \emptyset$ or
$\D \cap \C = \emptyset$. In the first case we will say that $\D$ is on the cycle.
In the second case we have that $\D \subseteq \intplus(\C_\F)$ since


\subsection{On the cycle}
  Note that type \ref{t:2alt}, \ref{t:3} and \ref{t:v2cons} can not occur on the cycle $\C$ maintained by the algorithm since they give a chord, offending Invariant \ref{i:noChords}.

  \mypar{Type \ref{t:1}}
  This is a \emph{short chord}, that is a chord with a range of size only 3. Note that we allow a choice of edge flip to be made later.

  \mypar{Type \ref{t:2cons}}
  We can do a simple move evading the problem.
  %But maybe a normal algo run will also work. However for the normal algo we assume no 4-cycles, so we should do this separately.

  We make the move depicted in Figure \ref{fig:4c:cycle_b}
  \begin{figure}[h]
    \centering
    \includegraphics[scale=1]{4cycles/img/cycle_b}
    \caption{Removing a Type \ref{t:2cons} separating $4$-cycle}
    \label{fig:4c:cycle_b}
  \end{figure}

  This moves the the cycle $\C$ past the problematic $4$-cycle $\D$. Note that we do not allow any freedom in the interior edges in this case.

  \mypar{Type \ref{t:v1}}
  This type of separating $4$-cycle does not induce a irregularity in the fence. Hence no operation is necessary.
  As can be seen in Figure \ref{fig:4c:v1}, we can just ''corner-slice" it. This separating $4$ cycle is no problem since it produces no irregularity on the (pre)fence $\W$.

  \begin{figure}[h]
    \centering
    \includegraphics[scale=1]{4cycles/img/cycle_e}
    \caption{A prefence has no problem with a Type \ref{t:v1} separating $4$-cycle}
    \label{fig:4c:v1}
  \end{figure}

  Note that the remainder of this maximal $4$-cycle may contain another separating $4$-cycle. Even when the two green edges are not a $4$-cycle. Such a $4$-cycle however is by design non-separating since the fence takes the topmost path.

  \mypar{Type \ref{t:v2alt}}
  This is just a combination of a ordinary non-simple point and a Type \ref{t:2cons} case. If we first recurse on the inner non-simple point we can then solve the rest like the Type \ref{t:2cons} case.

\subsection{On the fence}
  Note that Types \ref{t:1}, \ref{t:2cons} and \ref{t:v1} do not provide irregularities when they are on the fence. Hence we do not have to do anything to deal with these separating $4$-cycles. Instead we can treat them as being ``on the cycle'' in later iterations of this algorithm.

  \fxnote{Show that these types are okay with an image}

  \mypar{Type \ref{t:2alt}}
  See Figure \ref{fig:4c:fence_c}.

  \begin{figure}[h]
    \centering
    \includegraphics[scale=1]{4cycles/img/fence_c}
    \caption{Removing Type \ref{t:2alt} on the fence}
    \label{fig:4c:fence_c}
  \end{figure}

  \mypar{Type \ref{t:3}}
  See Figure \ref{fig:4c:fence_d}


  \begin{figure}[h]
    \centering
    \includegraphics[scale=1]{4cycles/img/fence_d}
    \caption{Removing Type \ref{t:3} on the fence}
    \label{fig:4c:fence_d}
  \end{figure}

  \mypar{Type \ref{t:v2alt}}


  \mypar{Type \ref{t:v2cons}}
    This is the inside of a chord

\subsection{Not on the cycle or on the fence}
  Then $\D$ certainly causes no problems  is no problem.

\subsection{$4$-cycles with a complicated interior}
  We can just recurse

\subsection{Adjecent 4-cycles}
  Give huge problems. See counterexample in section 2

  Can share only 1 edge. Otherwise they are not maximal or the graph does not make sense.
