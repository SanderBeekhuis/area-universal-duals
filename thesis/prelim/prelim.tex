%!TEX root = ../thesis.tex

\section{Preliminaries}
All graphs in this thesis are simple. Paths and cycles are always simple while walks are not necessarily simple. A path is a simple walk and a cycle is a closed path.
The \emph{length} of a walk (and thus of a path or cycle) is the number of edges it contains.

We will let $\braces {i, \ldots, j}$ denote all the integers $x$ such that $i \leq x \leq j$.

The \emph{degree} of a face is the number of vertices it is incident to and a \emph{triangular} face is a face of degree $3$.

We will use \emph{$k$-cycle} to denote a cycle of length $k$. Moreover a \emph{triangle} is simply a cycle of length $3$ (i.e. a $3$-cycle).

\fxnote{We could cite Jordan's paper here. Should I?}
By Jordan's curve theorem a cycle splits the plane into two parts, one bounded and one unbounded. We will call the bounded part the \emph{interior} of this cycle and the unbounded part the \emph{exterior} of this cycle.
We will call a cycle \emph{separating} if there are vertices in both it's interior and exterior.
An \emph{interior edge} of a cycle is then an edge contained in the interior of the cycle.


We will sometimes give vertices of a path $\P$ explicitly using $p_1 \ldots p_k$. A chord of a walk is an edge that connects two vertices in a walk that is not part of the walk. If a walk has no chords we will call it \emph{chordfree}. Since paths and cycles are special types the same definitions will hold for them.

Once we fix a planar embedding of a graph we can talk about the \emph{rotation} at a vertex $v$. The rotation at a vertex is clockwise order of the edges incident to $v$. We will identify these edges with their other endpoints. Two vertices $x, y$ are said to be \emph{consecutive} in the rotation at $v$ when the edges $vx$ and $vy$ are consecutive. In most of this thesis we implicitly fix an embedding since triangulations arr 3-connected and thus only have one planar embedding by a theorem due to Whitney (1933). This statement is proven in for example \cite[p. 267]{Bondy2008}

Given a path $\P$ with vertices $p_1 \ldots p_k$ we will say that a vertex $v \nin \P$ adjacent to $p_i, i \in \braces{2, \ldots, k-1}$ lies on the \emph{left} of $\P$ if it lies between $p_{i-1}$ and $p{i+1}$ in the rotation at $p_{i}$. Otherwise $v$ lies between $p_{i+1}$ and $p_{i-1}$ in the
rotation at $p_i$. In this case we say that $v$ lies to the \emph{right} of $\P$.

We will use the same notion of left and right for edges.
\fxnote{We could provide a picture illustrating these concepts}

\subsection{Plane triangulations}

\begin{defi} [Plane triangulation]
A graph in which all faces are of degree $3$.
\end{defi}

\begin{defi} [Maximal planar graph]
A graph such that adding any one edge makes it non-planar.
\end{defi}

\begin{thrm}
Any graph $G$ is a plane triangulation if and only if it is maximal planar
\end{thrm}

\begin{proof}
We will prove the equivalence of the negations.

Suppose that $G$ is not maximally planar. Then there is a face $F$ to which we can add an edge while keeping $G$ planar, however this face must then have degree of degree $4$ or larger since we can split into two faces with an edges. But a face has at least degree 3. Hence $G$ is not a plane triangulation.

Suppose that $G$ is not a plane triangulation. Then there must be a face $F$ of degree larger then $3$. This face will thus admit an extra edge without violating planarity and hence $G$ is not maximally planar.
\end{proof}

Because every face of a plane triangulation is triangular we can make the following remark.

\begin{remark}
  Every edge is incident to two triangular faces
\end{remark}

\begin{lemma}
  \fxnote{Is this more a remark then a lemma?}
  \label{lm:prelim:rotationEdge}
  If two vertices $x, y$ are consecutive in the rotation at $v$ then $xy$ is an edge in $G$ and $vxy$ is a triangle.
\end{lemma}
\begin{proof}
  in the described situation we have a partial face $\ldots x v y \ldots$.  Since every face of $G$ is a triangle $xy$ must be an edge and $vxy$ must be a triangle.
\end{proof}

\subsubsection{Connectedness}
Let us first note that any maximally planar graph is $2$-connected. Suppose there is a cutvertex, then surly we can add an edge between the components found after removing this cutvertex.
\begin{thrm}
Any plane triangulation $T$ is $3$-connected.
\label{th:plTri3Connected}
\end{thrm}

\begin{proof}
Suppose that $T$ is not $3$-connected. Then there must be a $2$-cutset $S$, given by the vertices $x$ and $y$. Removing this cutset splits the graph into at least two connected components $C_i$ and all components are incident to all cutvertices otherwise we would have found a $1$-cutset.

Since $S$ is a cutset, there can't be any edges incident to both $C_1$ and $C_2$. But then the edge $xy$ should be separating the $2$ components on both sides. This is impossible since we can only draw this edge once.
\fxnote{We could add a figure to make this more clear}
\end{proof}

\begin{defi}[Irreducible triangulation]
We call a triangulation irreducible if it has no separating triangles
\end{defi}

\fxnote{It is called irreducible because there is a reduction that works on separating triangles. We might show this reduction}

\begin{thrm}
Any irreducible plane triangulation $T$ is $4$-connected.
\end{thrm}

\begin{proof}
Note that any plane triangulation is $3$-connected by Theorem \ref{th:plTri3Connected}.

Suppose that $T$ is not $4$-connected. Then there must be some $3$-cutset (since it is $3$-connected) let us denote the vertices of this cutset by $x, y$ and $z$. Removing this cutset splits the graph into at least two connected components $C_i$ and all components are incident to all cutvertices otherwise we would have found a $2$- or $1$-cutset.

However, now $xy$ must be an edge in the triangulation $T$ otherwise the graph is not maximal planar (There can't be an edge incident to both $C_1$ and $C_2$ because that would negate $x, y ,z$ being a cutset.). In the same way $yz$ and $xz$ are edges of $T$. But then $xyz$ is a separating triangle. This is an contradiction and thus $T$ is $4$-connected
\end{proof}

\subsection{Triangulations of the $k$-gon}

\begin{defi}[Triangulation of the $k$-gon]
We call a graph a triangulation of the $k$-gon if the outer face has degree $k$ and all interior faces have degree $3$.
\end{defi}
Vertices bordering the outer face are \emph{outer vertices} while all other vertices are \emph{interior vertices}. Furthermore the cycle formed by all vertices outer vertices is the \emph{outer cycle}.

Sometimes such triangulations of the $k$-gon are called \emph{(plane) triangulated graphs}.


\begin{defi}[Irreducible triangulation of the $k$-gon]
We call a triangulation of the $k$-gon irreducible if it has no separating triangles.
\end{defi}


Note that triangulation of the $n$-gon $n\geq 4$ is not maximally planar and thus not plane triangulation.

The \emph{completion} of a triangulation of the $k$-gon $G = (V, E)$. Is the graph $G'= (V', E')$ with vertex set $V' = V \cup \braces{s}$ and edge set $E' = E \cup \braces{ sv | v \text{ is a outer vertex}}$

The completion is plane triangulation.  %Q does this stament need proof?
Since the interior of the outer cycle of $G$ always consisted of faces of degree 3. The exterior of the outer cycle consisted of one face of degree $k$ (the outer face) but the completion has turned this into $k$ faces of degree $3$.

\begin{thrm}
A triangulation of the $k$-gon $G$ is $2$-connected.
\end{thrm}
\begin{proof}
Suppose that $G$ has a cutvertex $v$. Then the set $\braces{s, v}$ is a $2$-cutset of the completion $G'$ of $G$. This however is in contradiction to Theorem \ref{th:plTri3Connected} stating that $G'$ is $3$-connected. Hence $G$ has no cutvertex and is thus $2$-connected.
\end{proof}

\begin{thrm}
  \label{th:triOfK3:VertexDisjointPaths}
  For every interior vertex $v$ of a triangulation of the $k$-gon $G$ is connected by at least $3$ vertex disjoint paths to different outer vertices.
\end{thrm}
\begin{proof}
By Theorem \ref{th:plTri3Connected} the completion $G'$ of $G$ is $3$-connected. Hence there are 3 vertex-disjoint paths from $v$ to $s$. Since $v$ is on the interior and $s$ is on the exterior of the outer cycle $\C$ all these 3 paths cross the outer cycle at least once. These paths cross $\C$ for the first time in different vertices since they are vertex-disjoint. If we shorten the paths to their first crossing with $\C$ we obtain the $3$ paths in the theorem.
\end{proof}

\fxnote{We can sharpen this to $4$ if we have a irreducible triangulation of the $k$-gon with a chordfree outer cycle}
