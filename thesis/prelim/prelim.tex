%!TEX root = ../thesis.tex

\section{Preliminaries}
\thispagestyle{plain}
\label{s:prelim}

Before we continue with the rest of thesis we will first define some basic definitions for graphs, paths and cycles in this section.

\mypar{Graphs}
  A \emph{graph} $G$ is an abstraction of a network. The objects are represented by a set of \emph{vertices}.
  Connections between objects are represented by a set of \emph{edges}; each edge connects two vertices.
  Two distinct edges do not have the same vertices and no edge starts and ends at the same vertex.
  That is, all graphs in this thesis are \emph{simple}.
  An edge is \emph{incident} to a vertex $v$ if that edge connects $v$ to another vertex.
  The \emph{degree} of a vertex is the number of edges incident to this vertex.
  All graphs in this thesis are \emph{planar}.
  That is, they can be embedded in the plane without their edges crossing. A \emph{face} is connected component of the maximal subset of the plane that is disjoint from the embedded graph. The \emph{degree} of a face is the number of vertices on its boundary.
  A face of degree $3$ is a \emph{triangular} face. The \emph{outer face} is the one and only unbounded face.
  A vertex is \emph{incident} to a face when it lies on its boundary.

  All graphs in this thesis are \emph{triangulations of the $k$-gon}. A triangulation of the $k$-gon has an outer face of degree $k$ and interior faces of degree $3$. Note that every corner assignment is a triangulation of the $4$-gon.
  Vertices bordering the outer face are \emph{outer vertices} while all other vertices are \emph{interior vertices}.
  Triangulations of the $k$-gon are called \emph{(plane) triangulated graphs} by some other authors.


\mypar{Paths}
  A path $\P$ is a sequence of vertices such that every two consecutive vertices are connected by an edge. The first and last vertex of the path are its \emph{extreme} vertices while the rest are \emph{internal} vertices of this path. The \emph{length} of a path is the number of edges used to connect the vertices. That, is one less than the number of vertices. In this thesis all paths are \emph{simple}, that is, no vertex occurs twice in the path except possibly the extreme vertices.

\mypar{Cycles}
  A cycle is a path whose extreme vertices coincide. Because a cycle is a path the start and end vertex are the only vertices that occurs more than once. We call a cycle of length $k$  a \emph{$k$-cycle}. A \emph{triangle} is cycle of length $3$ (i.e. a $3$-cycle). By Jordan's curve theorem a cycle splits the plane into two parts, one bounded and one unbounded. We call the bounded part the \emph{interior} of this cycle and the unbounded part the \emph{exterior} of this cycle.
  Furthermore, the cycle of all vertices bordering the outer face is the \emph{outer cycle}.
  We call a cycle \emph{separating} if there are vertices in both its interior and exterior.
  An \emph{interior edge} of a cycle is then an edge contained in the interior of the cycle.
  An \emph{interior path} is a path connecting two distinct vertices off the cycle and whose edges are interior edges.


\fxnote{Unclear what the purpose of this pargraph is}
 \mypar{Different kinds of rectangular layouts}
   A rectangular layout $\L$ can have different properties.
   Recall that $\L$ is area-universal if, no matter the areas we assign to the rectangles of $\L$, some combinatorially equivalent layout $\L'$ has rectangle of the assigned areas.
   Recall also that a \emph{maximal line segments} (or simply \emph{maximal segments}) of $\L$ is a line segment not contained in any other line segment. That is, it can not be extended any further. In this definition a \emph{line segment} (or simply \emph{segment}) of $\L$ is a sequence of consecutive inner boundary segments forming a line.
   A segment is \emph{one-sided} if it is on the boundary of a single rectangle. We call it \emph{$k$-sided} if on one of the sides it is the boundary of at most $k$ rectangles.

   We say a layout \emph{one-sided} if all maximal segments are one-sided.
   Furthermore,  we say it is \emph{vertically one-sided} or \emph{horizontally one-sided} if all vertical or horizontal maximal segments are one-sided, respectively. Finally, a layout is \emph{$k$-sided} if all maximal line segments are $k$-sided.

\fxnote{Unclear what the purpose of this pargraph is}
\mypar{Corner assignments}
  If we want to determine which graphs do have a rectangular dual, then we need to introduce the notion of a \emph{corner assignment}.
  A corner assignment $\ext G$ of $G$ is an augmentation of $G$ with $4$ external vertices ,which we call its \emph{poles}, with the following 3 properties (i) every interior face has degree $3$, (ii) the exterior face has degree $4$ and (iii) $\ext G$ has no separating triangles.
  A corner assignment fixes which rectangles are in the corners of the rectangular dual $\L$, which explains the terminology.

  A corner assignment of $G$ only exists if $G$ is a triangulation of the $k$-gon for some $k$. Otherwise, there is no way of adding poles that makes all the interior faces of degree $3$. Because of this, we only consider triangulations of the $k$-gon in this thesis. A corner assignment $\ext G$ of $G$ is an example of a triangulation of the $4$-gon. A corner assignment fixes which rectangles are in the corners of the rectangular dual $\L$, which explains the terminology.
