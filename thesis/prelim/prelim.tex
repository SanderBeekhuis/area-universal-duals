%!TEX root = ../thesis.tex

\section{Preliminaries}
\fxinnote{Lead in paragraph}
\fxinnote{define graph}
\fxinnote{define notations and concepts in one go}

All graphs in this thesis are simple. Paths and cycles are always simple while walks are not necessarily simple. A path is a simple walk and a cycle is a closed path.
The \emph{length} of a walk (and thus of a path or cycle) is the number of edges it contains.
We will let $\braces {i, \ldots, j}$ denote all the integers $x$ such that $i \leq x \leq j$.
The \emph{degree} of a face is the number of vertices it is incident to and a \emph{triangular} face is a face of degree $3$.

\mypar{Paths}
  A path $\P$ is a sequence of vertices such that every two consecutive vertices are connected with each other. The first and last vertex of the path are its \emph{extreme} vertices while the rest are \emph{interior} vertices of this path. The \emph{length} of a path is the number of edges used to connect the vertices. That, is one less then the number of vertices. All paths are \emph{simple}, that is, no vertex occurs twice in the path except the extreme vertices.

\mypar{Cycles}
  We call a cycle of length $k$  a \emph{$k$-cycle}. A \emph{triangle} is cycle of length $3$ (i.e. a $3$-cycle).
  By Jordan's curve theorem a cycle splits the plane into two parts, one bounded and one unbounded. We will call the bounded part the \emph{interior} of this cycle and the unbounded part the \emph{exterior} of this cycle.
  We call a cycle \emph{separating} if there are vertices in both its interior and exterior.
  An \emph{interior edge} of a cycle is then an edge contained in the interior of the cycle.
  An \emph{interior path} is a path connecting two distinct vertices off the cycle and whose edges are interior edges.

\mypar{Path notations}
  We denote the vertices of a path $\P$ by $p_1 \ldots p_k$.
  With $\rev{P}$ we denote the \emph{reversed path} $p_k \ldots p_1$. We use $\oplus$ to denote the \emph{concatenation} of paths. That is, given a second path $\Q$ with vertices $q_1 \ldots q_l$ and $p_k = q_1$ the path $\P \oplus \Q$ consists of $p_1 \ldots p_{k-1} q_1 q_2 \ldots q_l$.
  Recall that a cycle is simply a path starting and ending at the same vertex. Hence if we have two  internally disjoint paths $\P, \Q$ from $s$ to $t$ then $\P \oplus \rev{\Q}$ is a cycle.
  Furthermore we use a vertical bar to denote the \emph{restriction} of a path to a certain set of vertices. So $\P|_{p_i, p_j}$ with $i<j$ is the subpath of $\P$ with vertices $p_i \ldots p_j$.

\mypar{Chords}
  A \emph{chord} of a path is an edge that connects two vertices in this path, but is not part of the path. A path without chords is \emph{chordfree}.
  A \emph{k-chord} is a path $\Q$ of length $k$ that connects two non-subsequent vertices $p_i, p_j$ of $\P$ such that $\P \cap \Q = \braces{p_i, p_j}$.
  Since a cycle is a special type of path the same definitions also apply to them.
  Note that $\P|_{v_i, v_j} \oplus \rev{\Q}$ is a cycle. The ($k$-)chord $\Q$ is \emph{separating} if this cycle is separating.

\mypar{Rotations}
  We fix an embedding for all graphs in this thesis. In this embedding the \emph{rotation} at a vertex $v$ is the clockwise order of the edges incident to $v$. We will identify these edges with their other endpoints.
  Two vertices $x, y$ are said to be \emph{consecutive} in the rotation at $v$ when the edges $vx$ and $vy$ are consecutive.
  Given a path $\P$ and a interior vertex $p_i$. A neighbor $v \nin \P$ of $p_i$ lies on the \emph{left} of $\P$ if it lies in the interval from $p_{i-1}$ to $p_{i+1}$ in the clockwise rotation at $p_{i}$. Otherwise $v$ lies in the interval from $p_{i+1}$ to $p_{i-1}$ in the rotation clockwise at $p_i$. In this case $v$ lies on the \emph{right} of $\P$.
  We will use the same notion of left and right for edges. That is, an edge $e\nin P$ adjacent to $p_i$ lies to left or right if its other end point lies to the left or right, respectively.

\mypar{Triangulations of the $\mathbf{k}$-gon}
  A graph is a \emph{triangulation of the $k$-gon} if the outer face has degree $k$ and all interior faces have degree $3$.
  Vertices bordering the outer face are \emph{outer vertices} while all other vertices are \emph{interior vertices}. Furthermore, the cycle formed by all vertices outer vertices is the \emph{outer cycle}.
  Triangulations of the $k$-gon are called \emph{(plane) triangulated graphs} by some authors.
