%!TEX root = ../thesis.tex

\section{Preliminaries}
\thispagestyle{plain}
\fxinnote{Lead in paragraph}
\fxinnote{define notations and concepts in one go}
\fxinnote{state results at and of this section}

\mypar{Adjacent vertices and chords}
  \fxnote{introduce where this gets relevant}
  We call a vertex \emph{$\pS$-adjacent} when it is adjacent to $\pS$ in the current corner assignment. In the same way we call a chord or $2$-chord \emph{$\pE$-bound}, \emph{$\pW$-bound} or \emph{polebound} when it is adjacent to $\pE$, $\pW$ or any pole,  respectively.


 \subsection{Different kinds of rectangular layouts}
   A rectangular layout $\L$ can have different properties.
   Recall that $\L$ is area-universal if, no matter the areas we assign to the rectangles of $\L$, some combinatorially equivalent layout $\L'$ has rectangle of the assigned areas.
   Recall also that a \emph{maximal line segments} (or simply \emph{maximal segments}) of $\L$ is a line segment not contained in any other line segment. That is, it can not be extended any further. In this definition a \emph{line segment} (or simply \emph{segment}) of $\L$ is a sequence of consecutive inner boundary segments forming a line.
   A segment is \emph{one-sided} if it is on the boundary of a single rectangle. We call it \emph{$k$-sided} if on one of the sides it is the boundary of at most $k$ rectangles.


   We say a layout \emph{one-sided} if all maximal segments are one-sided.
   Furthermore,  we say it is \emph{vertically one-sided} or \emph{horizontally one-sided} if all vertical or horizontal maximal segments are one-sided, respectively. Finally, a layout is \emph{$k$-sided} if all maximal line segments are $k$-sided.
