%!TEX root = ../thesis.tex

\section{Preliminaries}
\thispagestyle{plain}
\label{s:prelim}

Before we continue with the rest of thesis we will in this section first define some basic definitions for graphs, paths and cycles.

\mypar{Graphs}
  A \emph{graph} $G$ is an abstraction of a network. The objects are represented by a set of \emph{vertices}.
  Connections between objects are represented by a set of \emph{edges}; each edge connects two vertices.
  Two distinct edges do not have the same vertices and no edge starts and ends at the same vertex.
  That is, all graphs in this thesis are \emph{simple}.
  An edge is \emph{incident} to a vertex $v$ if that edge connects $v$ to another vertex.
  The \emph{degree} of a vertex is the number of edges incident to this vertex.
  All graphs in this thesis are \emph{planar}.
  That is, they can be embedded in the plane without their edges crossing. A \emph{face} is connected component of the maximal subset of the plane that is disjoint from the embedded graph. The \emph{degree} of a face is the number of vertices on its boundary.
  A face of degree $3$ is a \emph{triangular} face. The \emph{outer face} is the one and only unbounded face.
  A vertex is \emph{incident} to a face when it lies on its boundary.

  Vertices bordering the outer face are \emph{outer vertices} while all other vertices are \emph{interior vertices}.
  Triangulations of the $k$-gon are called \emph{(plane) triangulated graphs} by some other authors.

\mypar{Angular order}
  For a fixed embedding of $G$ the \emph{angular order} at a vertex $v$ is the clockwise order of the edges incident to $v$. We identify these edges with their other endpoints.
  Two vertices $x, y$ are said to be \emph{consecutive} in the angular order at $v$ when the edges $vx$ and $vy$ are consecutive in the angular order.


\mypar{Paths}
  A path $\P$ is a sequence of vertices such that every two consecutive vertices are connected by an edge. The first and last vertex of the path are its \emph{extreme} vertices while the rest are \emph{internal} vertices of this path. The \emph{length} of a path is the number of edges used to connect the vertices. That, is one less than the number of vertices. In this thesis all paths are \emph{simple}, that is, no vertex occurs twice in the path except possibly the extreme vertices.

\mypar{Cycles}
  A cycle is a path whose extreme vertices coincide. Because a cycle is a path the start and end vertex are the only vertices that occurs more than once. We call a cycle of length $k$  a \emph{$k$-cycle}. A \emph{triangle} is cycle of length $3$ (i.e. a $3$-cycle). By Jordan's curve theorem a cycle splits the plane into two parts, one bounded and one unbounded. We call the bounded part the \emph{interior} of this cycle and the unbounded part the \emph{exterior} of this cycle.
  Furthermore, the cycle of all vertices bordering the outer face is the \emph{outer cycle}.
  We call a cycle \emph{separating} if there are vertices in both its interior and exterior.
  An \emph{interior edge} of a cycle is then an edge contained in the interior of the cycle.
  An \emph{interior path} is a path connecting two distinct vertices off the cycle and whose edges are interior edges.
