%!TEX root = ../thesis.tex

\section{Preliminaries}
All graphs in this thesis are simple. Paths and cycles are always simple while walks are not necessarily simple. A path is a simple walk and a cycle is a closed path.
The \emph{length} of a walk (and thus of a path or cycle) is the number of edges it contains.

We will let $\braces {i, \ldots, j}$ denote all the integers $x$ such that $i \leq x \leq j$.

The \emph{degree} of a face is the number of vertices it is incident to and a \emph{triangular} face is a face of degree $3$.

We will use \emph{$k$-cycle} to denote a cycle of length $k$. Moreover a \emph{triangle} is simply a cycle of length $3$ (i.e. a $3$-cycle).

\fxnote{We could cite Jordan's paper here. Should I? Or is this overdoing it? My textbook doesn't have a reference p. 245}
By Jordan's curve theorem a cycle splits the plane into two parts, one bounded and one unbounded. We will call the bounded part the \emph{interior} of this cycle and the unbounded part the \emph{exterior} of this cycle.
We will call a cycle \emph{separating} if there are vertices in both it's interior and exterior.
An \emph{interior edge} of a cycle is then an edge contained in the interior of the cycle.


We will sometimes give vertices of a path $\P$ explicitly using $p_1 \ldots p_k$. A chord of a path is an edge that connects two vertices in this path but is not part of the path. If a path has no chords we will call it \emph{chordfree}. Since paths and cycles are special types the same definitions will hold for them.

Once we fix a planar embedding of a graph we can talk about the \emph{rotation} at a vertex $v$. The rotation at a vertex is clockwise order of the edges incident to $v$. We will identify these edges with their other endpoints. Two vertices $x, y$ are said to be \emph{consecutive} in the rotation at $v$ when the edges $vx$ and $vy$ are consecutive in the clockwise order. In most of this thesis we implicitly fix an embedding since triangulations are 3-connected and thus only have one planar embedding by a theorem due to Whitney (1933). This theorem is proven in for example \cite[p. 267]{Bondy2008}

Given a path $\P$ with vertices $p_1 \ldots p_k$ we will say that a vertex $v \nin \P$ adjacent to $p_i, i \in \braces{2, \ldots, k-1}$ lies on the \emph{left} of $\P$ if it lies between $p_{i-1}$ and $p_{i+1}$ in the rotation at $p_{i}$. Otherwise $v$ lies between $p_{i+1}$ and $p_{i-1}$ in the
rotation at $p_i$. In this case we say that $v$ lies to the \emph{right} of $\P$.

We will use the same notion of left and right for edges.

\subsection{Plane triangulations}

\begin{defi} [Plane triangulation]
A graph in which all faces are of degree $3$.
\end{defi}

\begin{defi} [Maximal planar graph]
A graph such that adding any one edge makes it non-planar.
\end{defi}

\begin{thrm}
Any graph $G$ is a plane triangulation if and only if it is maximal planar
\end{thrm}

\begin{proof}
We will prove the equivalence of the negations.

Suppose that $G$ is not maximally planar. Then there is a face $F$ to which we can add an edge while keeping $G$ planar, however this face must then have degree of degree $4$ or larger since we can split into two faces with an edges. But a face has at least degree 3. Hence $G$ is not a plane triangulation.

Suppose that $G$ is not a plane triangulation. Then there must be a face $F$ of degree larger then $3$. This face will thus admit an extra edge without violating planarity and hence $G$ is not maximally planar.
\end{proof}

Because every face of a plane triangulation is triangular we can make the following remark.

\begin{remark}
  Every edge is incident to two triangular faces
\end{remark}

\begin{lemma}
  \label{lm:prelim:rotationEdge}
  If two vertices $x, y$ are consecutive in the rotation at $v$ then $xy$ is an edge in $G$ and $vxy$ is a triangle.
\end{lemma}
\begin{proof}
  in the described situation we have a partial face $\ldots x v y \ldots$.  Since every face of $G$ is a triangle $xy$ must be an edge and $vxy$ must be a triangle.
\end{proof}

\subsection{Triangulations of the $k$-gon}

\begin{defi}[Triangulation of the $k$-gon]
We call a graph a triangulation of the $k$-gon if the outer face has degree $k$ and all interior faces have degree $3$.
\end{defi}
Vertices bordering the outer face are \emph{outer vertices} while all other vertices are \emph{interior vertices}. Furthermore the cycle formed by all vertices outer vertices is the \emph{outer cycle}.

Sometimes such triangulations of the $k$-gon are called \emph{(plane) triangulated graphs}.
