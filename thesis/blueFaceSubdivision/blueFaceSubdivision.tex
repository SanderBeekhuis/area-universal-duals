%!TEX root = ../thesis.tex

\subsection{Blue face subdivision}
\thispagestyle{plain}
\label{ss:subdiv}
  At this point we have a vertically one-sided graph (due to Lemma \ref{lm:topfan:oneSidedREL}) without large topfans except possibly for the locations provided in Lemma \ref{lm:topfan:remainingTopfans}.

  In this section we are going to recolor edges in blue faces to make all of them $d+1$-sided. While at the same time not recoloring too many edges above each other as too create a too large red face.

  We will start at the bottommost face $F$ in the creation order (which we will have to recalculate after the topfanflips, but every regular edge labeling has one). And we will recolor some of its edges if it is too large. \fxnote{I may need to expand a bit more on this order}
  We then mark the edges on the top boundary path of this face above the recolored edges as \emph{loaded}. This means that we will try to not flip above these edges in future iterations of the algorithm.
  Then we continue with the next face in the creation order.

  \mypar{Loads}
  As is mentioned above we will mark some edges with so-called \emph{loads}, we will in the rest of the section refer to these edges as \emph{loaded}.
  The exact use of these loaded edges will become clear in the rest of this section.

  It is important to note that if we load any blue edge we regard any other blue edge sharing at least one vertex with this edge to be loaded as well. The occurrence of this phenomenon will be called \emph{putting trough a load}. An example can be seen in Figure \ref{fig:subdiv:putTrougLoad}. Here we flit the thick edge. Hence we mark $uv$ as loaded. But because of putting trough load $uw$ also becomes loaded.

  \begin{figure}[h]
    \centering
    \includegraphics[scale=1]{blueFaceSubdivision/img/puttingTroughLoad.pdf}
    \caption{Putting trough load.}
    \label{fig:subdiv:putTrougLoad}
  \end{figure}


\mypar{Step requirements}
  We flip edges in each face, taking into account loads on the bottom boundary path. Such that

  \begin{enumerate}
    \item We never load the two edges next to a split or merge vertex on the top boundary path.
    \item We never load two adjacent edges on the top boundary path
  \end{enumerate}

  If we flips edges in line with the step requirements for every face then the following lemma holds for the bottom boundary path for yet untreated faces.

  \begin{lemma}
    \label{lm:}
    On the bottom boundary path of every face we never find two subsequent loaded edges. Even when we put trough loads on splits and merges.
  \end{lemma}
  \begin{proof}
    A single face would never load two subsequent edges. Hence the only way to get two subsequent loaded edges is using different faces and thus splits and merges.
    However due to never flipping the two edges next to a split or merge we neither get subsequent loaded edges in such a case.
  \end{proof}


\subsubsection{Faces without large topfans in the midlle}
  Let us first consider the base case: no failed topfan flips and thus all topfans are of size exactly $2$.

  \begin{lemma}
    \label{lm:subdiv:withoutTopfan}
    We can subdivide any blue face without large topfans into 5-sided chunks while obeying the load rules above.
  \end{lemma}

  \begin{proof}
  A worst case example is given in Figure \ref{fig:subdiv:worstCase}.

  \begin{figure}[h]
    \centering
    \includegraphics[scale=1]{blueFaceSubdivision/img/worstCase}
    \caption{A worst case blue face. We do not flip any edge in this face.}
    \label{fig:subdiv:worstCase}
  \end{figure}

  Note that we can flip to the right above each edge in the bottom boundary path.

  We will look at the vertex on the bottom fence that's adjacent to the freshly flipped edge, or if we haven't flipped an edge yet the vertex next to the split (and we will call it $v$). The following are then the rules for flipping above the edges following $v$.
  \begin{enumerate}
    \item We do not flip above the first edge.
    \item We flip above the second edge if it is unloaded.
    \item Otherwise we flip above the third edge.
    \item We never flip next to the merge the merge
  \end{enumerate}

  When flipping above a edge we always flip the right edge above that edge.

  The first edge give us the required separation of loaded edges along the top boundary path. The other items make sure we obey the other rules in a straightforward manner.

  The worst case is given by a combination of the last two items. We would in that case want to flip above the third edge. But we do not because the next edge is the merge. This gives at worst 5 edges along the whole bottom boundary path and hence a $4$-sided face.
  \end{proof}

  See Figure \ref{fig:subdiv:sampleExecution} for a sample execution of the algorithm described in Lemma \ref{lm:subdiv:withoutTopfan}.

  \begin{figure}[h]
    \centering
    \includegraphics[scale=1]{blueFaceSubdivision/img/sampleExecution}
    \caption{Sample execution of the algorithm.}
    \label{fig:subdiv:sampleExecution}
  \end{figure}


\subsubsection{Face starting with a large topfan}
  \fxnote{It might be $d-1$ in worst case}
  The same algorithm as in Lemma \ref{lm:subdiv:withoutTopfan} after skipping the first topfan instead of the first edge finds us a finds us an edge keeping this face as at most a $ d - 3 +3 = d$-sided face. A sample worst case scenario is given in Figure \ref{fig:subdiv:worstCaseWithTopFan}.

  \begin{figure}[h]
    \centering
    \includegraphics[scale=1]{blueFaceSubdivision/img/worstCaseWithTopFan}
    \caption{A worst case blue face. We do not flip any edge in this face.}
    \label{fig:subdiv:worstCaseWithTopFan}
  \end{figure}


\subsubsection{Face encountering a larger topfan}
  If we have a large topfan in the middle of the face then above the left outer edge of this topfan we can not have another topfan that failed to flip its left outer edges by Lemma \ref{lm:sweep:NoTwoSplitsAboveEachOther}.

  This means we can use the following rule: we flip the first edge of a topfan even above a loaded edge.
  We call such an edge a \emph{forced} flip. We can not have two such forced flips above each other because that would give a situation as in Figure \ref{fig:subdiv:forcedFlips}.
  However that would mean the fan with fanhandle $u$ must be the handle of a topfan that failed its flip and hence $v$ must have been a split vertex. But then by lemma \ref{lm:zflip:NoTwoSplitsAboveEachOtherVertOnesided} $w$ can not be the handle of a large topfan.

  \begin{figure}[h]
    \centering
    \includegraphics[scale=1]{blueFaceSubdivision/img/forcedFlips.pdf}
    \caption{Two forced flips above each other.}
    \label{fig:subdiv:forcedFlips}
  \end{figure}

  Since we don't allow a flip above a load (whether it was forced or not) this means that the worst thing that can happen is an ordinary flip followed by a \emph{forced} flip. That can't be followed by any other flip. Hence the worst case only makes chains of at most $2$ blue $Z$'s, that is a blue path of length $5$.


\subsubsection{Conclusion}
  This concludes the last step in the algorithm.
  Now all the steps in the algorithm are done all that is left is to show that we indeed generated a $d$-sided regular edge labeling.
  \begin{lemma}
    \label{lm:subdiv:2chaindedZ}
    Two chained $Z$'s give at worst a red $d-1$-sided face
  \end{lemma}
  \begin{proof}
    The two chained $Z$'s give a blue path $\P$ of length $5$ inside a red face.

    Before creating the $Z$'s in this section the regular edge labeling was vertically one-sided. That is, before recoloring the two edges in this section there where no paths of length $3$ inside the face. This also implies that any $Z$ we now create can have at most one blue fan on the top and one blue fan on the bottom, otherwise we would already have had a $3$-path.

    So for two $Z$'s we have at most three blue fans. Hence on one side we have at most one of these. Then the boundary path at this side of the face face has at most $d-3 + 1 +1 =d-1$ vertices not counting the split and merge vertex of the red face.
  \end{proof}

    Then we can now proof Theorhem \ref{th:dsided}.

  \begin{proof}[of Theorem \ref{th:dsided} ]
    By construction a blue faces are $d$-sided. We have chained at most two $Z's$ so all red face contain at most two blue $Z$. So red faces are $d-1$-sided by Lemma \ref{lm:subdiv:2chaindedZ}. Hence we have a $d$-sided rectangular edge labeling of $\ext G$ corresponding to a $d$-sided rectangular dual of $\ext G$
  \end{proof}
