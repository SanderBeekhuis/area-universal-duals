%!TEX root = ../thesis.tex

\section{Rectangular duals}
\newcommand{\G}{\scr G}
\renewcommand{\L}{\scr L}

In this section we will introduce the rectangular dual of a graph. We will prove some simple properties of graphs and their (rectangualar) duals.

We define a \emph{rectangular layout} (or simply \emph{layout}) $\L$ to be a partition of a rectangle into finitely many interiorly disjoint rectangles.

We will for simplicity of analysis assume that no four rectangles meet in one point. \fxwarning{Probably refer to \cite{Kozminski1984}}

We will then define the \emph{dual graph of a layout} $\L$ and denote this graph by $\G(\L)$. That is, we represent each rectangle by a vertex and we connect two vertices by an edge exactly when their rectangles are adjacent. Note that this graph is not the same as the \emph{graph dual} of $\L$ when we view it as a graph (namely we don't represent the outer face of $\L$ by a vertex).

So $\G(\L)$ is the dual graph of a layout $\L$. In the reverse direction we say a layout $\L$ is a \emph{rectangular dual} of a graph $\G$ if we have that $\G = \G (\L)$.

A plane triangulated graph $\G$ does not necessarily have a rectangular dual nor is this dual necessarily unique.
\fxnote{in what sense not unique, provide examples}


\subsection{Extended graphs}
A \emph{extended graph} $\ext G$ of $G$ is a augmentation of $G$ with $4$  vertices (which we will call it's \emph{poles}). Such that
\begin{enumerate}
\item every interior face has degree $3$ and the exterior face has degree $4$.
\item all poles are incident to the outer face
\item $\ext\G$ has no separating triangles (i.e separating $3$-cycles).
\end{enumerate}.

We sometimes call an extended graph $\ext G$ of $G$ an \emph{extension} of $G$.

Such a extended graph does not necessarily exist and is not necessarily unique.
However we have the following result due to Kozminski and Kinnen

\begin{thrm}[Existence of a rectangular dual]
A plane triangulated graph $\G$ has a rectangular dual \ifftext it has an extension $\ext \G$
\end{thrm}

\begin{proof}
  This proofed in \cite{Kozminski1984} \fxnote{Provide location, Kozminski \& kinnen and ungar, See siAM paper}
\end{proof}

We call any (plane triangulated) graph $G$ that has an extension a \emph{proper} graph.

A proper graph $G$ can have more then one extensions. Each such extension fixes which of the rectangles are in the corners of the rectangular dual $\L$. Hence sometimes such an extension is called a \emph{corner assignment} by other authors.

\subsection{Different kinds of rectangular duals}
\fxerror{This subsection still has to be written}
Here we talk about area-unviersal, onesided, vertical/horizontal onsided and psuedo-onsided/ (k,l) -s sided duals.
We also state the result by \cite{Eppstein2012} that area-universal duals are the same as onesided duals.


\subsection{Regular edge labeling}
\fxerror{This subsection still has to be written}

A regular edge labelling  of $\ext G$ corresponds to a rectangular dual $\L$ of some fixed extension \ext G.

A regualar edge labeling has certain rules.

We will refer to Kant and He \cite{Kant1997} and also note the naming by \cite{Fusy2006}


\subsection{Oriented regular edge labeling}
\fxerror{This subsection still has to be written}
In this subsection we will introduce the unique orientation on a REL. We wills define red faces and blue faces (which we can also do in the unoriented version i guess) with two sides (which is difficult in unoriented settings) and define split and merge vertices as the first and last vertices of such face.

We will also note it is equivalent to the oriented Fusy structure

\subsubsection{Being onesided in terms of REL}

\subsubsection{Being psudeo-onesided in terms of REL}
