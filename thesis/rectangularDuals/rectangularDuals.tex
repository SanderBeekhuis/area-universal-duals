%!TEX root = ../thesis.tex

\section{Regular edge labellings}
\label{s:rel}
\thispagestyle{plain}
\mypar{Adjacency graphs of layouts}
  In an \emph{adjacency graph} $\dualgraph{\L}$ of a layout $\L$ each rectangle is represented by a vertex.
  In an adjacency graph we connect two vertices by an edge exactly when their rectangles are adjacent.
  In the \emph{extended adjacency graph} $\extdualgraph{\L}$ we also add $4$ vertices $\pN, \pE, \pS, \pW$ (so-called \emph{poles}) in the outer face, one associated to the north, east, south, west boundary segment of the outer rectangle, respectively.
  In this graph two vertices are connected if their rectangles or boundary segments intersect.
  If we take the \emph{extended adjacency graph} of a layout and remove the $4$ vertices corresponding to the outer face we end up with the regular \emph{adjacency graph} of that layout.
  In this setting a layout $\L$ is a \emph{rectangular dual} of a graph $G$ if we have that $G = \dualgraph{\L}$.

\mypar{Regular edge labellings}
  Given a layout $\L$ we can easily find its adjacency graph and thus for which graph $G$ it is a rectangular dual. However, finding a rectangular dual of a graph $G$ is more involved. Due to the algorithm by He \cite{He1993} it is sufficient to find a \emph{regular edge labeling} on a corner assignment of $G$. In this section we introduce regular edge labellings.

  Regular edge labellings were first introduced by Kant and He \cite{Kant1997}.
  Fusy also studied these structures \cite{Fusy2009, Fusy2006} under the name of \emph{transversal structures}.
  A \emph{regular edge labeling} is a coloring and orientation of the edges of the extended adjacency graph $\extdualgraph{\L}$.
  This coloring and orientation is given by the following procedure.
  For every edge $vw$ in $\extdualgraph{\L}$ we consider whether the shared boundary of the rectangles is vertical or horizontal and we color the corresponding edge blue or red, respectively.
  If we color the edge blue we orient it from the leftmost rectangle to the rightmost rectangle and if we color it red we orient from bottom to top.
  We neither color nor orient the edges between the poles.

  From the nature of the adjacencies in a rectangular layout we can deduce the following two rules, both illustrated in Figure~\ref{fig:rel:conditions}, for a regular edge labeling.

  \begin{enumerate}
    \item (Interior vertex) In the angular order of every interior vertex we have the following subsequent non-empty sets: Incoming red edges, incoming blue edges, outgoing red edges and outgoing blue edges and these are the only sets that exist.
    \item (Pole) $\pN$ has only incoming red edges, $\pE$ has only incoming blue edges, $\pS$ has only outgoing red edges and $\pW$ has only outgoing blue edges, except for the uncolored edges between the poles.
  \end{enumerate}

  \begin{figure}
      \centering
      \begin{subfigure}[b]{0.2 \textwidth}
          \includegraphics[width = \textwidth]{rectangularDuals/img/interiorcondition.pdf}
          \caption{Interior vertex condition}
      \end{subfigure}
      ~
      \begin{subfigure}[b]{0.7 \textwidth}
          \includegraphics[width =\textwidth]{rectangularDuals/img/exteriorCondition.pdf}
          \caption{Pole condition}
      \end{subfigure}
      \caption{Regular edge labeling conditions}
  \label{fig:rel:conditions}
  \end{figure}
  In \cite{He1993} He showed that given a regular edge labeling of a corner assignment we can reconstruct a rectangular layout represented by this regular edge labeling.

  \mypar{Combinatorially equivalent}
    We say two duals of $G$ are  \emph{combinatorially equivalent} or simply \emph{equivalent} when they give raise to the same regular edge labeling of $\ext G$. That is, their rectangles have the same adjacencies with the same orientation (horizontal or vertical).

  \mypar{Properties}
    Since we often use regular edge labellings in this thesis we prove some properties for them. We show that a regular edge labeling has no mono-colored triangles and that the red and blue subgraph of a regular edge labeling both form a $st$-planar graph.

  \begin{observation}[Fusy, \cite{Fusy2009}]
    \label{obs:rel:noMonoColoredTriangles}
    A regular edge labeling has no monochromatic triangles even when ignoring orientation.
  \end{observation}

  \begin{proof}
    Suppose we have a monochromatic triangle. Without loss of generality we suppose this triangle is blue. Then at least one of the vertices has an incoming blue edge followed directly by an outgoing blue edge or an outgoing blue edge followed directly by an incoming blue edge in its angular order. Thus, this vertex has either an empty set of outgoing or incoming red edges and hence violates the interior vertex condition of a regular edge labeling.
  \end{proof}

  \mypar{$\mathbf{st}$-planar graphs}
    In this thesis we repeatedly use regular edge labellings, so it is a good idea to investigate their structure.
    Kant and He \cite{Kant1997} and Fusy \cite{Fusy2009} both note that a regular edge labeling is closely linked to a pair of \emph{$st$-planar graphs}. $st$-planar graphs are also known under the name of \emph{bipolar orientations}.
    We state this observation in the same way it is stated by Fusy, which is slightly different from the way Kant and He state it.
    An $st$-planar graph is an oriented planar graph with one source (in-degree 0) $s$ and one sink (out-degree 0) $t$. Both $s$ and $t$ lie on the outer face. Moreover, such an $st$-planar graph has no directed cycles.

    \begin{observation}[Fusy, \cite{Fusy2009}]
      \label{obs:rel:stPlanarGraphs}
      The blue edges of $G\sm{\pN,\pS}$ form an $st$-planar graph with $s= \pW$ and $t=\pE$. Moreover, the red edges of $G\sm{\pW,\pE}$ form an $st$-planar graph with $s= \pS$ and $t= \pN$.
    \end{observation}
    \begin{proof}
      Note that we have no monochromatic directed cycles because such a cycle would for example correspond to a  group of adjacent rectangles that have  no leftmost or topmost one. By the interior vertex condition interior vertices can not be sources or sinks, this leaves the non-removed poles to be the sources and sinks, as required.
    \end{proof}

    We refer to these $st$-planar graphs as the \emph{blue graph} and \emph{red graph} of some regular edge labeling and we refer to their faces as \emph{blue faces} and \emph{red faces}. An example of such a colored extended adjacency graph with the blue and red graph can be found in Figure~\ref{fig:rect:relSegmentFace}.

    Every face $F$ in an $st$-planar graph has the same structure. The boundary of $F$ consists of two directed paths, so-called \emph{boundary paths}, with common start vertex $v$ and end vertex $w$. We say $v$ is the \emph{split} vertex of $F$ and $w$ is the \emph{merge} vertex of $F$.
    These boundary paths are subsequent in the clockwise angular order at $v$. We say the first path in the angular order, starting at the beginning of the adjacent pair, is the \emph{top boundary path} (blue graph) or \emph{left boundary path} (red graph) of $F$ and the second one is the \emph{bottom boundary path} (blue graph) or \emph{right boundary path} (red graph).


    \mypar{Maximal segments}
    Due to the way we color a regular edge labeling of $\extdualgraph{\L}$ given a layout $\L$ a horizontal maximal segment corresponds to a blue face and a vertical maximal segment corresponds to a red face; as one can see in Figure~\ref{fig:rect:relSegmentFace}.

    \begin{figure}[t]
      \centering
      \includegraphics[scale=1]{rectangularDuals/img/relSegmentFaceRescale}
      \caption{An example regular edge labeling with corresponding rectangular dual.}
      \label{fig:rect:relSegmentFace}
    \end{figure}

    Let us consider a face corresponding to a maximal segment.
    The number of internal vertices of both boundary paths without counting the split and merge vertices is the number of rectangles on the respective sides of the maximal segment.
    Hence, a one-sided maximal segment corresponds to a face with one boundary path of length $2$  and a $k$-sided maximal segment to a face with the shortest boundary path of length at most $k+1$.
    We can not have faces with a boundary path of length $1$ since such a face can not enclose a valid segment.

    \begin{observation}
      \label{obs:rel:noBpOfLength1}
      No face can have a boundary path of length $1$.
    \end{observation}
    \begin{proof}
      A boundary path can not have length $1$ since by construction of a regular edge labeling a red or blue face encloses a maximal segment and thus has to go trough at least one intermediate rectangle, and thus trough at least one intermediate vertex.
    \end{proof}

    This observation generalizes the observation that we can not have monochromatic triangles.
