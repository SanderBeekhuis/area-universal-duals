%!TEX root = ../thesis.tex

\section{Rectangular duals}
\newcommand{\G}{\scr G}
\renewcommand{\L}{\scr L}

In this section we will introduce the rectangular dual of a graph. We will prove some simple properties of graphs and their (rectangular) duals.

We define a \emph{rectangular layout} (or simply \emph{layout}) $\L$ to be a partition of a rectangle into finitely many interiorly disjoint rectangles.

We will for simplicity of analysis assume that no four rectangles meet in one point. \fxwarning{Probably refer to \cite{Kozminski1984}}

We will then define the \emph{dual graph of a layout} $\L$ and denote this graph by $\G(\L)$. That is, we represent each rectangle by a vertex and we connect two vertices by an edge exactly when their rectangles are adjacent. Note that this graph is not the same as the \emph{graph dual} of $\L$ when we view it as a graph (namely we don't represent the outer face of $\L$ by a vertex).

So $\G(\L)$ is the dual graph of a layout $\L$. In the reverse direction we say a layout $\L$ is a \emph{rectangular dual} of a graph $\G$ if we have that $\G = \G (\L)$.

A plane triangulated graph $\G$ does not necessarily have a rectangular dual nor is this dual necessarily unique.
\fxnote{in what sense not unique, provide examples}


\subsection{Extended graphs}
When constructing the rectangular dual of a graph $G$. It will be useful to add four vertices to the graph. We will denote these vertices with $\pN, \pE, \pS, \pW$ in corresponding with the four cardinal directions.

We will define define two types of these \emph{extensions}. First we define a \emph{regular extension} $\ext G$ which we can apply to any graph $G$. Then we will define a tight extension $\tightext G$ for which we will require two distinct vertices on the outer cycle of $G$.

\paragraph{Regular extension}
A \emph{extension} $\ext G$ of $G$ is a augmentation of $G$ with $4$ vertices (which we will call it's \emph{poles}). Such that
\begin{enumerate}
\item every interior face has degree $3$ and the exterior face has degree $4$.
\item all poles are incident to the outer face
\item $\ext\G$ has no separating triangles (i.e separating $3$-cycles).
\end{enumerate}.ext
Such a extended graph does not necessarily exist and is not necessarily unique.
However we have the following result due to Kozminski and Kinnen

\begin{thrm}[Existence of a rectangular dual]
A plane triangulated graph $\G$ has a rectangular dual \ifftext it has an extension $\ext \G$
\end{thrm}

\begin{proof}
  This proofed in \cite{Kozminski1984} \fxnote{Provide location, Kozminski \& Kinnen and ungar, See Siam paper}
\end{proof}

We call any (plane triangulated) graph $G$ that has an extension a \emph{proper} graph.

A proper graph $G$ can have more then one extensions. Each such extension fixes which of the rectangles are in the corners of the rectangular dual $\L$. Hence sometimes such an extension is called a \emph{corner assignment} by other authors.

Note that a graph $G$ with a separating triangle can't be proper, since every possible extension will have a separating triangle.

\paragraph{Tight extension}
We only define the \emph{tight extension} $\tightext G$ of graph $G$ without separating triangles in two vertices $v, v'$ if the outer cycle is split into two chordfree paths $P, P'$ by $v,v'$. Otherwise the tight extension is undefined.

We can without loss of generality assume that the order of these vertices and paths is $v P v' P'$ going clockwise along the outer cycle. We will then set $\pW = v$, $\pE =v'$ and add two vertices $\pN, \pS$. We connect every vertex of $P$ to $\pN$ and every vertex of $P'$ to $\pS$.

\begin{lemma}
  If it is defined the tight extension of $G$ in $v, v'$ is a extension of $G \sm{v, v'}$.
\end{lemma}
\begin{proof}
  It is clear from the construction that every interior face of $\tightext G$ is of degree 3 and that the outer face is given by $\pN \pE \pS \pW$ an is thus of degree $4$.
  To see that $\tightext G$ has no separating triangles note that $G$ has none and that any separating triangle must thus have one of $\pN, \pS$ as a vertex. However a separating triangle containing $\pN$ or $\pS$ would imply a chord in $P$ or $P'$. However in this case the tight extension is not defined.
\end{proof}

A tight extension $\tightext G$ in $v , v'$ is uniquely determined if it exists.

\subsection{Different kinds of rectangular duals}
\fxerror{This subsection still has to be written}
Here we talk about area-universal, onesided, vertical/horizontal onesided and pseudo-onesided/ (k,l) -s sided duals.
We also state the result by \cite{Eppstein2012} that area-universal duals are the same as onesided duals.


\subsection{Regular edge labeling}
\fxerror{This subsection still has to be written}

A regular edge labeling  of $\ext G$ corresponds to a rectangular dual $\L$ of some fixed extension $\ext G$.

A regular edge labeling has certain rules.

We will refer to Kant and He \cite{Kant1997} and also note the naming by \cite{Fusy2006}


\subsection{Oriented regular edge labeling}
\fxerror{This subsection still has to be written}
In this subsection we will introduce the unique orientation on a REL. We wills define red faces and blue faces (which we can also do in the unoriented version i guess) with two sides (which is difficult in unoriented settings) and define split and merge vertices as the first and last vertices of such face.

We will also note it is equivalent to the oriented Fusy structure

\subsubsection{Being onesided in terms of REL}


\begin{lemma}
\label{lm:zInRedFace}
A face $F$ with at least $3$ edges on each side contains a $Z$
\end{lemma}
\begin{proof}
\fxerror{TODO}
\end{proof}

\subsubsection{Being pseudo-onesided in terms of REL}
