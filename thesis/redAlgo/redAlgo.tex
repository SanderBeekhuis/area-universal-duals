%!TEX root = ../thesis.tex


\section{Vertical one-sided dual}
\label{s:red}

We can also adapt Fusy's algorithm to generate a vertically one-sided dual. We then need top generate a regular edge labeling without red faces that have $3$ or more edges on both borders.

We will have an additional requirement on top of the requirement that $\ext G$ has no separating triangles. We will also require that $G$ has no separating four cycles.


\paragraph{Notational concerns}
  Just as in Section \ref{s:algo} we will use $\C$ to indicate the current sweep line cycle.
  We will repeatedly only consider the path $\cpath$. In that case we will always order it from $\pW$ to $\pE$.

  Instead of interior paths we will consider interior walks but we will use similar notation. That is a walk between two distinct vertices of $\C$ of which all vertices except the first and last one are in the interior of $\C$.

  We will let $\W$ denote a interior walk. Given such a walk of $k$ vertices we index it's nodes $w_1, \ldots, w_k$  in such a way that $w_1$ is closer to $W$ then $w_k$ is (and thus that $w_k$ is closer to $E$ then $w_1$ is).

  Then $w_1$ and $w_k$ indicate the two unique vertices of the walk that are also part of the cycle. We will then let $\restC{\W}$ denote the part of $\C\setminus {\mathrm{S}}$ that is between $w_1$ and $w_k$ (including). $\C_\W$ will denote the closed walk formed when we paste $\restC{W}$ and $\W$.

  Since paths are a subclass of walks all of the above notation can also be used for a path $\P$. Note that the closed walk $\C_\P$ in this case will actually be a cycle.


\subsection{The neighbor walk of a path}
  During this proof we will frequently use the concept of the left or right neighbor walk of a path.
  Given a path $P = p_1 \ldots p_k$ in a graph $G$
  The \emph{right neighbor walk} $W$ of $P$ will consist of $p_1$, we will then take the vertices adjacent to $p_{2}$ between $p_1$ and $p_{3}$ in the clockwise rotation at $p_{2}$, followed by the vertices between $p_{2}$ and $p_{4}$ in the rotation at $p_{3}$ and so further until we add the vertices between $p_{k-2}$ and $p_k$ in the rotation around $p_{k-1}$ and finally we finish by adding $p_k$ to $W$.
  We then remove all subsequent duplicates from $W$

  \begin{lemma}
    \label{lm:red:neighborWalk}
    The right neighbor walk $W$ is a walk.
  \end{lemma}
  \begin{proof}
    Let us first show that $W$ is indeed a walk. We will proof that every vertex is adjacent to the next vertex. Let us suppose that $w$ and $w'$ are two subsequent vertices in $W$, we will show that $ww'$ is an edge if $\braces{w, w'} \cap \braces{p_1, p_k } = \emptyset$. Afterwards we will consider this edge case. There are then two cases for $w, w'$. Either $(a)$ $w$ and $w'$ are vertices adjacent to some $p_i$ subsequent in clockwise order or $(b)$ $w$ was the last vertex adjacent to some $p_i$ and thus $w'$ is the first vertex adjacent to $p_{i+1}$.

    The following two situations can also be seen in Figure \ref{fig:walkproof}.

    \begin{figure}[h]
        \centering
        \begin{subfigure}[b]{0.5\linewidth}
            \includegraphics[width=\linewidth]{redAlgo/img/walkProofA}
            \caption{}
        \end{subfigure}%
        \begin{subfigure}[b]{0.5\linewidth}
            \includegraphics[width=\linewidth]{redAlgo/img/walkProofB}
            \vspace{1cm}

            \caption{}
        \end{subfigure}

          \caption{The two main cases of the proof showing that $W$ is a walk}
      \label{fig:walkproof}
    \end{figure}

     \fxwarning{TODO adapt figures to $p_i$ instead of $v_i$}


    In case $(a)$ we note that $p_i w$ and $p_i w'$ are edges next to each other in the clockwise rotation at $p_i$. Since every interior face of $\ext G$ is a triangle $ww'$ must be an edge. We thus see that $w, w'$ are adjacent and not duplicates.

    In case $(b)$ we note that $p_i w$ and $p_i p_{i+1}$ are edges subsequent in clockwise order, hence $wp_{i+1}$ is also an edge. Hence $w$ is the first vertex adjacent to $p_{i+1}$ subsequent to $v_i$ in the clockwise rotation. Thus $w= w'$. They are duplicates and one of them must have been removed.

    Now for the edge cases: Let $x$ be the first vertex adjacent to $p_{i+1}$ and let $y$ be the last vertex adjacent to $p_{j-1}$. $p_i$ and $x$ are vertices adjacent to $p_{i+1}$ subsequent in the clockwise rotation, and hence connected by Lemma \ref{lm:prelim:rotationEdge}. In the same way $y$ and $v_j$ are subsequent vertices in the rotation at $v_n$ and hence connected.

    Hence $\W$ is a walk.
  \end{proof}


  We will call a walk \emph{noncrossing} if at every vertex $w$ in the walk that is visited $k$ times such that $w_{i_1} = w_{i_2} = \ldots = w_{i_k}$ in the walk the clockwise intervals $[w_{i_j-1}, w_{i_j+1}]$ for $j \in \braces {1, \ldots, k}$ are disjoint in the rotation at $w$.
  \fxnote{We might add a (small) figure for clarity (i.e. of a crossing and a non-crossing walk)}
  \begin{lemma}
    \label{lm:red:neighborWalkNoncrossing}
    The right neighbor walk $W$ is a \emph{non-crosssing} walk.
  \end{lemma}
  \begin{proof}
    Suppose that the right neighbor walk is crossing at a vertex $w= w_i =w_j$. Then one of $w_{j-1}$ and $w_{j+1}$ is in the clockwise interval $[w_{i-1}, w_{i+1} ]$ at the rotation at $w$. We will denote this vertex by $w'$. It is clear that $w'$ cannot be on the path unless $w'$ is $p_1$ or $p_k$. In this case however we see that $w_{i-1}$ or $w_{i+1}$ respectively couldn't have been part of the path.

    So we continue with $w'$ not on the path. All neighbors of $w$ between $w_{i-1}$ and $w_{i+1}$ in the clockwise rotation are on the path. \fxwarning{TODO make this a lemma}. So we have a series of triangles by Lemma \ref{lm:prelim:rotationEdge}. Now $w'$ must be inside one of these triangles, otherwise we would have a crossing edge (and thus a non-planar graph.) Now the triangle containing $w$ is a separating triangle.

    We conclude that $W$ must be a non-crossing walk.
  \end{proof}

  The nice thing about non-crossing walks is that they if they return to their startpoint they allow a notion of interior and exterior. We can see this by applying Jordans curve theorem to a version of this walk that is very slightly perturbed at every vertex visited multiple times. Which we can do due to the disjoint intervals in the rotation.


   \fxwarning{TODO Define left and right of a walk}
  \begin{lemma}
    \label{lm:red:neighbourwalkChordFree}
    The left of the of a right neighbor walk and the right of the left neighbor walk are chordfree.
  \end{lemma}
  \begin{proof}
    Suppose that the right neighbor walk $W = w_1 \ldots w_k$  has a chord on the left, say between $w_i$ and $w_j$ with $i< j -1 $. There is a vertex $p_\ell \in P$ on the path such that $w_{i+1}$ is a neighour of $p_\ell$ to the left of $p_\ell$ Consider now the following non-crossing closed walk $P w_k \ldots w_{j+1} w_j w_i w_{i-1} \ldots w_1$
    (Thick in Figure \ref{fig:red:neihbourwalkChordFree})this walk has $w_{i+1}$ in its exterior. But then $p_\ell w_{i+1}$ is a crossing edge. Which is forbidden.

    \begin{figure}[h]
      \centering
      \includegraphics[scale=1]{redAlgo/img/neighbourWalkChords}
      \caption{The construction in the proof of Lemma \ref{lm:red:neighbourwalkChordFree}}
      \label{fig:red:neihbourwalkChordFree}
    \end{figure}
  \end{proof}


\subsection{Outline}
  To describe the algorithm two more definitions are necessary

  \begin{defi}[Prefence]
  A prefence $\W$ is a interior walk of $\C$ starting at $v_i \in \C$ and ending at $v_j \in \C$ a both adjacent to $\pS$
  \begin{enumerate}
   \renewcommand*{\labelenumi}{(P\arabic{enumi})}%
   \renewcommand*{\theenumi}{(P\arabic{enumi})}%
    \item  For every $v_i \in \C \setminus \braces{\pW, \pS, \pE}$ we have that all vertices between $v_{i+1}$ and $v_{i-1}$ in the rotation at $v_i$ are in $\W \sm {\pW, \pE}$
    \label{p:C}
    \item For every $w_i \in \W \sm{\pW, \pE}$ we have that all vertices between $w_{i-1}$ and $w_{i+1}$ in rotation at $w_i$ are in $\C \sm {\pW, \pS, \pE}$
    \label{p:W}
    \item $w_2$ and $v_{i+1}$ are consecutive in the rotation at $v_i$
    \label{p:pW}
    \item $v_{j-1}$ and $w_{k-1}$ are consecutive in the rotation at $v_j$
    \label{p:pE}
  \end{enumerate}
  \end{defi}


  We enforce these conditions because they imply \ref{e:crossingEdges} when $\W$ is a path as we will show in Lemma \ref{lm:red:regularPrefenceIsFence}.

  For a walk however the interior is not clearly defined.

  \begin{defi}[Fence]
    A fence is a valid path starting and ending at a vertex adjacent to $S$
  \end{defi}

  \fxnote{ expand on naming/reasons of fence}


  We will show that there is a algorithm if there are no separating $4$-cycles in $G$ and no separating $3$-cycles in $\ext G$.


  The algorithm will receive as input a extended graph $\ext G$ and will return a regular edge labeling such that all red faces are $(1-\infty)$ using a sweep-cycle approach inspired by \Fusy \cite{Fusy2006}.

  We will start by creating a prefence $W$. This may not be a valid path, it doesn't even have to be a path. During the algorithm we will make a number of moves that will turn this prefence into a fence. In each move we shrink $C$ by employing a valid paths and change the prefence.

\subsection{Finding a initial prefence}
  Let $v_i$ denote all the vertices of $\cpath$ in the following order $\pW =  v_1 \  v_2 \  \ldots v_{n-1} \  v_n = \pE$.
  Some intervals of these vertices will be adjacent to $\pS$. However, they can't be all adjacent to $S$ since then the sweepcycle would be non-separating since we can't have separating triangles. We denote by $v_i$ the last vertex of fist interval of vertices adjacent to $S$ and by $v_j$ the first vertex of the second interval.
  As candidate walk we will start with $v_i$, we will then take the vertices adjacent to $v_{i+1}$ between $v_i$ and $v_{i+2}$ in the rotation at $v_{i+1}$, followed the vertices between $v_{i+1}$ and $v_{i+3}$ in the rotation at $v_{i+2}$ and so further until we add the vertices between $v_{j-2}$ and $v_j$ in the rotation around $v_{j-1}$ and finally we finish by adding $v_j$.

  We then remove all subsequent duplicate vertices from $\W$.

  \begin{lemma}
    \label{lm:red:isPrefence}
  The collection $W$ described above is a prefence.
  \end{lemma}
  \begin{proof}
  We will first show that $W$ is a walk. We will proof that every vertex is adjacent to the next vertex. Let us suppose that $w$ and $w'$ are two subsequent vertices in $W$, we will show that $ww'$ is an edge if $\braces{w, w'} \cap \braces{v_i, v_j } = \emptyset$. Afterwards we will consider this edge case. There are then two cases for $w, w'$. Either $(a)$ $w$ and $w'$ are vertices adjacent to some $v_i$ subsequent in clockwise order or $(b)$ $w$ was the last vertex adjacent to some $v_i$ and thus $w'$ is the first vertex adjacent to $v_{i+1}$.

  The following two situations can also be seen in Figure \ref{fig:walkproof}.

  \begin{figure}[h]
      \centering
      \begin{subfigure}[b]{0.5\linewidth}
          \includegraphics[width=\linewidth]{redAlgo/img/walkProofA}
          \caption{}
      \end{subfigure}%
      \begin{subfigure}[b]{0.5\linewidth}
          \includegraphics[width=\linewidth]{redAlgo/img/walkProofB}
          \vspace{1cm}

          \caption{}
      \end{subfigure}

      	\caption{The two main cases of the proof showing that $W$ is a walk}
  	\label{fig:walkproof}
  \end{figure}


  In case $(a)$ we note that $v_i w$ and $v_i w'$ are edges next to each other in clockwise order around $v_i$. Since every interior face of $\ext G$ is a triangle $ww'$ must be an edge. We thus see that $w, w'$ are adjacent and not duplicates.

  In case $(b)$ we note that $v_i w$ and $v_i v_{i+1}$ are edges subsequent in clockwise order, hence $wv_{i+1}$ is also an edge. Hence $w$ is the first vertex adjacent to $v_{i+1}$ after $v_i$ in clockwise order. Thus $w= w'$. They are duplicates and one of them must have been removed.

  Now for the edge cases: Let $x$ be the first vertex adjacent to $v_{i+1}$ and let $y$ be the last vertex adjacent to $v_{j-1}$. $v_i$ and $x$ are vertices adjacent to $v_{i+1}$ subsequent in clockwise order, and hence connected by Lemma \ref{lm:prelim:rotationEdge}. In the same way $y$ and $v_j$ are subsequent vertices in the rotation at $v_n$ and hence connected.

  Hence $\W$ is a walk. The above also shows that $v_i v_{i+1} x$ and $v_{j-1} v_j y$ are triangles by Lemma \ref{lm:prelim:rotationEdge} and hence $\W$ satisfies properties \ref{p:pW} and \ref{p:pE} of being a prefence.

  Moreover this walk satisfies  \ref{p:C} because $\W$ by construction contains all neighbors of any vertex $v_i \in \restC{\W} \sm{v_i, v_j}$ between $v_{i-1}$ and $v_{i+1}$ in the rotation of $v_i$.

  Finally to see that $\W$ also satisfies \ref{p:W}. Consider a vertex $w_j \in \W \sm{v_i, v_j}$ then either it is $(a)$ the neighbor of some vertex $v_i$ and only of this vertex or it is $(b)$ the unique vertex neighboring in the interior of the cycle the $\ell +1$ vertices $v_i, \ldots, v_{i+\ell}$. This is essentially the same case distinction as above. However now $(a)$ $w_{i-1} w_i v_i$ and $v_i w_i w_{i+1}$ or $(b)$ $w_{i-1} w_i v_i$, $v_i w_i v_{i+1}, \ldots, v_{i+\ell -1} w_i v_{i+\ell} $
  and $v_{i+\ell} w_i w_{i+1}$ form a set of triangles spanning the area between $w_{i-1}$ and $w_{i+1}$ in the rotation at $w_i$. Thus any edge not going to $\restC{\W} \sm{v_i, v_j}$ in this sector will lead to a separating triangle. We however have assumed $G$ has no separating triangles. Hence \ref{p:pW} holds.\fxnote[footnote, nomargin]{I believe this is still true when separating triangles are allowed to occur. However the prove will have to be different.}
  \end{proof}

  We then orient $\W$ from $v_i$ (the vertex closest to $\pW$)to $v_j$ (the vertex closest to $\pE$) and denote it's vertices by $w_1 \ldots w_k$.

\subsection{Irregularities}
  We will distinguish two kinds of \emph{irregularities} in a prefence.
  \begin{enumerate}
    \item The candidate walk is non-simple in a certain vertex. That is, if we traverse the sequence of vertices in $\W$ we see that $w_i = w_j$ for some $i<j$.
    \item The candidate walk has a chord on the right. That is, there is an edge $w_i w_j$ on the right of $\W$ with $i<j$ and $i$ and $j$ not subsequent (i.e. $i < j-1$).
  \end{enumerate}

  Note that we can't have a chord can on the left of $\W$ ($\W$ being oriented from $\pW$ to $\pE$), since if it would lie on the left of $\W$ the vertices $w_{i+1},\ldots, w_{j-1}$ would not have been chosen in the construction of the prefence.

  \begin{lemma}
    \label{lm:red:regularPrefenceIsFence}
    If a prefence has no irregularities it is a fence.
  \end{lemma}
  \begin{proof}
    We will show that all the requirements of being a valid path are met.
   \begin{itemize}
     \item [Path] Let us begin by noting that since there are no non-simple points we have a path and not just a walk.

     \item[\ref{e:noS}] It is clear that both $w_1$ and $w_k$ are not $\pS$ by the construction of the candidate walk.

     \item[\ref{e:longBorders}] For $\W$ or $\restC \W$ to have only one edge we need to have that $v_i v_j$ is an edge. However, $v_i v_j$ can not be an edge in $\C$ since $v_i$ and $v_j$ are from different intervals of vertices adjacent to $\pS$. It can also not be an edge in $\ext G \setminus \C$ since that would be a chord of the cycle and these don't exist by Invariant \ref{i:noChords}

     \item[\ref{e:crossingEdges}]
     Every interior edge of $\C_\W$  with at least one endpoint on the cycle is of the required type by the conditions \ref{p:C} - \ref{p:pE}. We note that these edges in particular have both endpoints on the cycle $\C_\W$.

     Interior edges with both endpoints not on the cycle can a priori exist. However since a triangulation is a connected graph there must then also be an edge with one endpoint on $\C_\W$, and one inside $\C_\W$ but this can not be if $\W$ is a prefence. However by the argument above both endpoints must then be on $\C_\W$, this is a contradiction.

     \item[\ref{e:noNewChord}] The cycle $\C'$ only changes between $v_i$ and $v_j$. There can be no chord with one vertex from cycle $\C \setminus \restC{\W}$ and one from $\W$ since such a chord would cross $\pS v_i$ or $\pS v_j$. There is no chord with two vertices in $\W$ since that would be a irregularity and there is no chord with two vertices from $\C \setminus \restC{\W}$ by Invariant \ref{i:noChords}.
   \end{itemize}
   Hence, if $\W$ has no irregularities it is a valid path.

   Furthermore, $\W$ is a path starting and ending at a vertex adjacent to $\pS$ because it is prefence. And thus it is a fence.
  \end{proof}


  \begin{defi}[Range of a irregularity]
  For a non-simple point \fxnote{Is it better to call this a non-simple point or a non-simple vertex?} $w_i = w_j$ with $i<j$ has \emph{range} $\braces{i, \ldots, j} \subset \N$.
  A chord $w_i w_j$ with $i< j-1$ has \emph{range} $\braces{i, \ldots, j} \subset \N$.
  \end{defi}

  Note that a chord can't have the same range as a non-simple point since then $w_i w_j$ will be a loop and we are considering simple graphs. Furthermore two chords have different ranges because we otherwise have a multiedge. Two nonsimple points with the same range are, in fact, the same. This leads us to the following remark.
  \begin{remark}
  \label{rk:diffIregDiffRange}
  Distinct irregularities have distinct ranges.
  \end{remark}

  \begin{defi}[Maximal irregularity]
  A irregularity is maximal if it's range is not contained\footnote{Because of Remark \ref{rk:diffIregDiffRange} being contained is the same as being strictly contained} in the range of any other irregularity.
  \end{defi}

  \begin{lemma}
  \label{lm:rangeOverlap}
  Maximal irregularities have ranges whose overlap is at most one integer.
  \end{lemma}
  \begin{proof}
  We let $I$ and $J$ denote two distinct maximal irregularities with ranges $\braces{i_1, \ldots i_2}$ and $\braces{j_1, \ldots, j2}$. Let us for the moment suppose that $I$ and $J$ have ranges that overlap more then one integer. Since $I$ and $J$ are both maximal their ranges can not be contained in each other.

  Without loss of generality we thus have $i_1 < j_1 < i_2 < j_2$.

  Now two chords to the right of $\W$ would cross each other but we have a planar graph so this can't be the case.

  Now let us without loss of generality suppose that $I$ is a non-simple point. A non-simple point $w_{i_1} = w_{i_2}$ is adjacent to two ranges of vertices in $\cpath$. $v_a \ldots v_b$ and $v_c \ldots v_d$
  then $\tilde{C} = w_{i_1} v_b \ldots v_c$ is a cycle. And because of the rotation at $w_{i_1} = w_{i_2}$ we have that $w_{i_{1 +1}}, \ldots, w_{i_{2 -1}}$ are inside this cycle while $w_1 \ldots w_{i_1 -1}$ and $ w_{i_2 +1} \ldots w_k$ are outside the cycle. See Figure.
    \fxnote{We could add figure to clarify.}

  Now if $J$ is a chord we have $\tilde{C}$, which can't be. If $J$ is also a nonsimple point this would imply that the vertex $w_{j_i} = w_{j_2}$ is at the same time inside and outside $\tilde{C}$ which is clearly impossible.
  \end{proof}

\subsection{Moves}
  \newcommand{\U}{\scr U}
  The algorithm will remove these irregularities by recursing on a subgraph for each maximal irregularity. We shrink the cycle $\C$ with every valid path that is found in the recurrence, in the order they are found. Afterwards we update the prefence by removing $w_{i+1}, \ldots, w_{j-1}$. In subsection \ref{ss:validity} we will show that the updated prefence is a prefence for the updated cycle $\C$.


  We will first show how to remove these maximal irregularities in Subsections \ref{ss:chords} and \ref{ss:nonsimplePoints}. That is, we show which subgraph $H$ we recurse upon for both kinds of irregularity. Furthermore we show that these subgraphs suffice the requirements of the algorithm.

  Afterwards, in subsection \ref{ss:validity} we will make sure that the subgraphs we recurse upon are edge-disjoint. That is, they only overlap in border vertices.

  It is worth noting that other irregularities contained in such a maximal irregularity are solved in the recurrence.

  \subsubsection{Chords}
    \label{ss:chords}
    If we encounter a chord we will extract a subgraph and recurse on this subgraph. A chord $w_iw_j$ has a triangular face on the left and on the right (like every edge). The third vertex in the face to the left will be called $x$. $x$ is not necessarily distinct from $w_{i+1}$ and/or $w_{j-1}$ but this is also not necessary for the rest of the argument. \fxnote{We might also work these out in a Figure.}

    The vertex $v_a$ on the cycle is uniquely determined as the vertices adjacent to both $w_i$ and $w{i+1}$. In the same way $v_b$ is the unique neighbor of $w_{j-1}$ and $w_j$.

    We will describe a walk $\scr U$ running from $v_a$ to $v_b$. This path consists of all vertices adjacent to $w_i$ in clockwise order from $v_a$ (inclusive) to $x$(inclusive) and subsequently all vertices adjacent to $w_j$ in clockwise order from $x$ (exclusive) to $v_b$ (inclusive). This path is given in bold in Figure \ref{fig:removeChord}.

    \begin{lemma}
    $\U$ is a chordfree path
    \end{lemma}
    \begin{proof}
     \fxnote*{Is it nice to refer to a line of reasoning like this?}{We note that $\U$ is a walk by the same reasoning as is given in Lemma \ref{lm:red:isPrefence}.}

    $\U$ cant have a non-simple point $x'$ since it would have to be connected to at least two vertices. However a vertex $x'$ that is distinct from $x$ and is connected to both $w_i$ and $w_j$ will induce a separating triangle $w_i x' w_j$. $\U$ also can't be nonsimple at $x$ since $x$ is the the third vertex of the triangular face $w_i w_j x$. Hence $\U$ is a path.


    $\scr U$ can't have chords $u_i u_j$ since they would either induce a separating $3$- or $4$-cycle either $w_i u_i u_j$ or $w_j u_i u_j$ or $w_i u_i u_j w_j$ depending on the vertex adjacent to $u_i$ and $u_j$.
    \fxnote{We use that we have no 4-cycles here}
    \end{proof}


    We then consider the interior of the cycle $\C_\scr U$ and the cycle $\C_\U$ itself as the subgraph $H$. We then take the tight extension at $v_a$ and $v_b$. We will then recurse on this graph $\tightext H$. See also Figure \ref{fig:removeChord}. Since $\C$ is chordfree by invariant \ref{i:noChords} so is $\restC{\U}$. We have also just shown that $\U$ is chordfree. So $\tightext H$ is indeed defined. Furthermore, since $H$ is a induced subgraph of $G$, $\tightext H$ contains no separating $4$-cycles not involving the poles.

    We update the prefence by removing $w_{i+1}, \ldots, w_{j-1}$.

    \begin{figure}[h!]
    \centering
    \includegraphics[scale=1]{redAlgo/img/removeChord}

    \caption{Removing a chord
        \label{fig:removeChord}}
    \end{figure}

  \subsubsection{Nonsimple points}
    \label{ss:nonsimplePoints}
    Removing a non-simple point is done is a similar manner.

    The vertex $v_a$ on $\C$ is uniquely determined as the vertices adjacent to both $w_i=w_j$ and $w{i+1}$. In the same way $v_b$ is the unique neighbor of $w_{j-1}$ and $w_j=w_i$. Note that it may be that $w_{i+1} = w{j-1}$ this does not matter for the rest of the argument. \fxnote{We may show this in a figure.}

    We will describe a walk $\scr U$ running from $v_a$ to $v_b$. This path consists of all vertices in the rotation at $w_i=w_j$ from $v_b$ (inclusive) to $v_a$(inclusive). This path is given in bold in Figure \ref{fig:removeNonSimplePoint}.

    \begin{lemma}
    $\U$ is a chordfree path.
    \end{lemma}
    \begin{proof}
      If we orient $\U$ from $v_a$ to $v_b$  we see that $\U$ cant have a non-simple point since such a point would have edges to at least two vertices on the right. However every vertex can only be connected to $w_i=w_j$. Hence $\U$ is a path.

      $\U$ can't have chords on the right of the path by the way we construct $\U$. Furthermore $\scr U$ can't have chords $u_i u_j$ on the left since they would either induce a separating $3$-cycle $w_i u_i u_j$.
    \end{proof}


    We then consider the interior of the cycle $\C_\scr U$ and the cycle $\C_\U$ itself as the subgraph $H$.

    We then take the tight extension of $H$ at $v_a$ and $v_b$ to recurse on. See also Figure \ref{fig:removeNonSimplePoint}. Since $\C$ is chordfree by Invariant \ref{i:noChords} so is $\restC{\U}$. We have also just shown that $\U$ is chordfree. So $\tightext H$ is indeed defined. Furthermore, since $H$ is a induced subgraph of $G$, $\tightext H$ contains no separating $4$-cycles not involving the poles.

     We update the prefence by removing $w_{i+1}, \ldots, w_{j-1}$ and we also recognize that $w_i = w_j$ is now a duplicate subsequent occurrence of the same vertex. So we also remove $w_j$.

    \begin{figure}[h!]
    \centering
    \includegraphics[scale=1]{redAlgo/img/removeNonSimplePoint}

    \caption{Removing a non-simple point
        \label{fig:removeNonSimplePoint}}
    \end{figure}

  \subsubsection{Validity}
    \label{ss:validity}

    \begin{lemma}
    After doing a move the updated prefence $W$ is a prefence for the updated cycle $C$
    \end{lemma}
    \begin{proof}
    \fxerror{TODO}
    \end{proof}

    \begin{lemma}
    Let $H_I$ and $H_J$ be two recursion subgraphs for different maximal irregularities $I$ and $J$. Then $H_I$ and $H_J$ are edge disjoint.
    \end{lemma}
    \begin{proof}
    \fxerror{TODO}
    \end{proof}

\subsection{Correctness}
  As long as the interior of $\C$ is nonempty we can find a prefence. And thus we find valid paths. Since we continuously shrink the cycle with valid paths we end up with a regular edge labeling.
  See core algorithm

  The algorithm finishes because it keeps on recursing and shrinking until no graph is left.

  \subsubsection{The red faces}
  Let us then argue that the red faces are all $(1-\infty)$ faces, corresponding to one-sided vertical segments. As is shown in Lemma \ref{lm:zInRedFace} it is sufficient to show no two vertices subsequent on a blue path are first a merge and then a split or vice versa.

  We will show the following
  \begin{lemma}
    A split or merge always happens on a vertex that is adjacent to $S$ for some recursion.
  \end{lemma}

  \begin{proof}
    Every valid path we shrink the cycle by is found as a fence on some recursion level. In this recursion level both $w_1$ and $w_k$ are adjacent to $S$.
  \end{proof}

  \begin{lemma}
    \label{lm:pathsStayOnRecursionLevel}
    A path starting at a certain recursion level will stay at that recursion level. It may share vertices with the north boundary of a lower recursion level but never with the south boundary.
  \end{lemma}
  \begin{proof}
    A valid path can never leave the subgraph $H$ in which its start- and end-vertex are located. Because it is found as a fence in this subgraph. It can also never run trough a graph $H'$ on a lower recursion level (except for the north boundary path)  because in every move the vertices of the prefence in $H'$ are deleted.
  \end{proof}

  Recall that all our valid paths are oriented from a start vertex to end vertex.

  \begin{lemma}
    A split can't directly be followed by a merge along any valid path during the algorithm.
  \end{lemma}
  \begin{proof}
    One of paths after the split is no longer on the south boundary of this subgraph $H$, nor on the south boundary of any other subgraph by Lemma \ref{lm:pathsStayOnRecursionLevel}. This path hence can't contain a merge.

    The other path still potentially follow the south boundary. However merging from the southward side of the path is impossible by Lemma \ref{lm:pathsStayOnRecursionLevel} from the northward side is equally impossible since the split and merge have to be neighboring vertices in the rotations of these vertices and thus the path $\P$ that merged must also join again.

    But then it is not a valid path.
  \end{proof}

  However for a blue Z to occur there has to be a valid path that first has a split and then has a merge. Since this can't be all red faces must have only $2$ edges on at least one side. Hence the regular edge labeling this algorithm produces corresponds to a vertically one-sided rectangular dual.


  \fxnote[inline, nomargin]{Show how the algorithm works with some cool examples: For example:    The multiple non-simple point $v_i = v_j =v_k$;   Example of page $F1$;  Example with lots of layered chords  }
