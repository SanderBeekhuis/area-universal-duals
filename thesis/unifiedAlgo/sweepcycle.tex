%!TEX root = ../thesis.tex

\subsection{Sweepcycle algorithm}
\thispagestyle{plain}
\label{ss:sweep}
The first step of our algorithm is executing a sweepcycle algorithm inspired by the sweepcycle algorithm by Fusy \cite{Fusy2006}. We use $\C$ to indicate the current sweep cycle. We shrink $\C$ by updating it with interior paths.
The algorithm finishes when $\C$ has no more interior vertices. When the algorithm finishes, it has produced a regular edge labeling.
One of the nicest things about the regular edge labeling is Lemma \ref{lm:sweep:NoTwoSplitsAboveEachOther}.
 \fxnote*{Q: These are all words the reader does not know yet. How to explain? Should I explain here?}{This lemma states that we can not have the fan handle of a large topfan after a split vertex on a so-called bottom path.}

During the algorithm we maintain several invariants on $\C$. The first four are equivalent to those imposed by Fusy. The final invariant is new and allows us to prove Lemma \ref{lm:sweep:NoTwoSplitsAboveEachOther}.

\begin{invariants}
%  \itemsep=-4pt
%  \item \label{i:uni:SWandSE} The cycle $\C$ contains the two edges $\pS \pW$ and $\pS \pE$.
%  \item \label{i:uni:noChords} $\cpath$ has no chords.
%  \item \label{i:uni:intVertCond} The inner vertex condition holds for all vertices in the exterior of $C$.
%  \item \label{i:uni:redOutgoing} Every non-pole vertex on the sweepcycle has a red outgoing edge.
%  \item \label{i:uni:no2Chords} $\C\sm{\pS}$ has no separating 2-chords that do not use $\pS$.
\end{invariants}

We initialize the sweepcycle $\C$ with the outer cycle of $\ext G$.
We denote the vertices of the sweepcycle $\C$ by $\pS, v_1 = \pW, v_2, \ldots v_{n-1}, v_n = \pE, \pS$.
We repeatedly consider the path $\cpath$.
In which case we order it from $\pW$ to $\pE$. That the edges $\pS \pW$ and $\pS \pE$ are always in $\C$ is a result of Invariant \ref{i:uni:SWandSE}.


Each update of the sweepcycle consists of the following three steps.
\begin{enumerate}
  \itemsep=-4pt
  \item Take the right neighbor walk of a subpath of $\cpath$ to get the \emph{candidate path} $P$.
  \item Evade chords and separating $2$-chords on $P$ to get the \emph{updating path} $P'$.
  \item Update the sweepcycle with $P'$.
\end{enumerate}

We repeat these steps until the sweepcycle does not contain anymore interior vertices. At which point we can terminate the algorithm by coloring the edges of the cycle $\C$ blue and its interior edges red.

\subsubsection{Find the right neighbor path}
  Recall we denote all the vertices of $\cpath$ by $\pW =  v_1   v_2   \ldots v_{n-1}   v_n = \pE$.

  Suppose they are all adjacent to $S$, then any vertex still in the interior of $\C$ would lie in a separating triangle of $G$. So we have no interior vertices and hence we can terminate the algorithm as described in Section \ref{sss:terminating}.
  In the remainder we assume some vertices from $\cpath$ are not adjacent to $\pS$.

  Since $\cpath$ has some vertices incident to $\pS$ ($\pW , \pE$) and some that are not, we can consider maximal subpaths of $\cpath$ consisting of vertices adjacent to $\pS$.
  We denote by $v_i$ the last vertex of first maximal subpath of vertices adjacent to $\pS$ and by $v_j$ the first vertex of the second maximal subpath.
  As candidate path $P$ we take the right neighbor path of $\cpath|_{v_i, v_j}$. This right neighbor path does indeed exist since all internal vertices of $\cpath|_{v_i, v_j}$ are interior vertices of $G$ and $\cpath$ has no cycles or separating 2-chords by Invariants \ref{i:uni:noChords} and \ref{i:uni:no2Chords}.
  This situation is depicted in Figure \ref{fig:sweep:noIrregularity}.

\subsubsection{Evade chords and separating 2-chords}
  The candidate path $P$ we found can have two structures we want to avoid namely:
  \begin{enumerate}
    \itemsep=-4pt
    \item Chords
    \item Separating 2-chords
  \end{enumerate}

  By \emph{irregularities}, we mean these two classes of structures together and the \emph{middle} vertex of a $2$-chord is the only internal vertex of that $2$-chord.
  All irregularities are on the right of the candidate path due to Lemma \ref{lm:right:neighbourwalkChordFree} (no chords on the left) and Lemma \ref{lm:right:neighbourwalkNoInteriorVertex} (no separating 2-chords on the left).

  Before we can show how to evade these structures, we first introduce more notation. We orient $P$ from $v_i \in \C$ (the vertex closest to $\pW$)to $v_j \in \C$ (the vertex closest to $\pE$) and denote its vertices by $p_1 \ldots p_k$.
  The \emph{index} of a vertex $p_i \in P$ is its position in the path, that is, the index of $p_i$ is $i$.
  The \emph{start index} of an irregularity $I$ is the index of the first vertex in $P$ that is also in $I$. Similarly, the \emph{end index} is the index of the last vertex in $P$ that is also in $I$.
  The \emph{range} of an irregularity is given by its start and end index. We update the sweepcycle with some \emph{update path} $P'$, depending on the irregularities we find on the candidate path $P$. This update is described in Section \ref{sss:sweep:update}.

  While describing the our updating paths depending the irregularities of $P$ we show that the following two lemmas hold in every case.

  \begin{lemma}
    The updating paths has no chords or separating $2$-chords
    \label{lm:sweep:augNoIregularity}
  \end{lemma}

  \begin{lemma}
    \label{lm:sweep:noConnectingIregularity}
    There are no chords or separating $2$-chords not containing $\pS$ with one endpoint on the old sweepcycle and one endpoint on the updating path $P'$.
  \end{lemma}

  \mypar{We have no irregularity}
    When there are no irregularities, we update the sweepcycle with the entire candidate path $P$.
    In this case the update path $P'$ is equal to $P$.

    $P'$ has no irregularity by the definition of this case. Moreover, there are no irregularities with one opinionated on $P'$ and one endpoint on $\C$ since $v_i$ and $v_j$ are both adjacent to $\pS$ we can not have any chords and any $2$-chords must have $\pS$ as middle vertex.
    \begin{figure}[b]
      \centering
      \includegraphics[scale=1]{unifiedAlgo/img/sweep/cases/noIrregularity}
      \caption{Updating path when $P$ contains no irregularity.}
      \label{fig:sweep:noIrregularity}
    \end{figure}

  \mypar{We have a chord on $\mathbf{P}$}
    Note that we can not have a chord incident to one of the exterior vertices of the candidate path $P$ since any such chord would violate Invariant \ref{i:uni:no2Chords} of $\C$ as can be seen in Figure \ref{fig:sweep:noChordOnExteriorVertex}.

    \begin{figure}[h]
      \centering
      \includegraphics[scale=1]{unifiedAlgo/img/sweep/noChordOnExtriorVertex.pdf}
      \caption{Hypothetical situation where $P$ would have a chord on an exterior vertex.}
      \label{fig:sweep:noChordOnExteriorVertex}
    \end{figure}

    We identify the chords by their ranges. Of the chords with the smallest end index $j$ we consider the one with the largest start index $i$. We denote this chord by $C$.
    Note that this chord can not contain any other chords since such a chord would have either a large start index or a smaller end index.
    The way in which we find $C$ is illustrated in Figure \ref{fig:sweep:chordsOnCandidatePath}.

    \begin{figure}[b]
      \centering
      \includegraphics[scale=1]{unifiedalgo/img/sweep/chordsOnCandidatePath}
      \caption{Finding the chord $C$.}
      \label{fig:sweep:chordsOnCandidatePath}
    \end{figure}

    What we do now depends on whether a 2-chord shows up in the interior of the chord concatenated to the candidate path $P|_{i, j} \oplus \rev{{C}}$.

    \emph{No separating 2-chord.}
      If there is no separating 2-chord in the interior of $P|_{i, j} \oplus \rev{{C}}$ we let $v_k$ be the shared neighbor in $\C$ of $p_{i}$ and $p_{i +1}$ and we let $v_l$ the shared neighbor in $\C$ of $p_{j -1}$ and $p_{j}$. The updating path $P'$ is the right neighbor path of $\cpath|_{v_k, v_l}$. See Figure \ref{fig:sweep:chordUpdate}.

      $P'$ is entirely inside a chord containing no more chords and thus can not contain a chord. Moreover, there are no separating $2$-chords on $P|_{p_i, p_j}$ so $P'$ can not have separating $2$-chords. Any irregularity with one endpoint in $P'$ and one in $\C$ has to cross $v_k p_i p_j v_l$ so we can not have a chord and any 2-chord has $p_i$ or $p_j$ as middle vertex. But, with this restriction such a 2-chord can not be separating.

      \begin{figure}[h]
        \centering
        \includegraphics[scale=1]{unifiedAlgo/img/sweep/cases/chordUpdate}
        \caption{Updating path when $P$ has a chord not containing a separating 2-chord.}
        \label{fig:sweep:chordUpdate}
      \end{figure}

    \emph{At least one separating 2-chord.}
      Let $j'$ be end index of the separating 2-chord with the lowest end index in the interior of $P|_{i, j} \oplus \rev{{C}}$. And let $v_k$ be the shared neighbor on the sweepcycle of $p_{i}$ and $p_{i +1}$ and let $v_l$ be the shared neighbor on the sweepcycle  of $p_{j' -1}$ and $p_{j'}$.
      Then the updating path $P'$ is the right neighbor path of $\cpath|_{v_k, v_l}$. See Figure \ref{fig:sweep:2chordInChordUpdate}.

      $P'$ is entirely inside a chord containing no more chords and thus can not contain a chord. Moreover, all separating $2$-chords are evaded since we evaded the end of the first one ending.  Just as in the above case we can not have any chord or separating $2$-chord with one endpoint in $P'$ and one in $\C$ any chord or separating 2-chord to connect outside the containing chord $p_i p_j$.
      That leaves irregularities within the second endpoint inside containing chord $p_i p_j$. \fxnote{TODO make this a complete cycle}
      Suppose that we have a separating $2$-chord, then this would have been a chord of $P$. This is in contradiction with $p_i p_j$ being a minimal chord.
      We also can not have a chord since this would break $P|_{p_i, p_j} \oplus p_j p_i$ .

      \begin{figure}[h]
        \centering
        \includegraphics[scale=1]{unifiedAlgo/img/sweep/cases/2chordInChordUpdate}
        \caption{Updating path when $P$ has a chord containing at least one separating 2-chord.}
        \label{fig:sweep:2chordInChordUpdate}
      \end{figure}

  \mypar{Only separating 2-chords}
    In this case the candidate path $P$ has no chords since we would otherwise be in the above case.

    \emph{Any separating 2-chords with $v_j$ as end vertex.}
      We find the smallest separating 2-chord with $v_j$ as end vertex (i.e. the one with the highest start index). Say this separating $2$-chord has start index $i$.
      Let $v_k$ be the shared neighbor on the sweepcycle of $p_{i}$ and $p_{i +1}$. The updating path is the right neighbor path of $\cpath|_{v_k, v_j}$. See Figure \ref{fig:sweep:pEBound}.

      $P'$ starts inside the smallest separating $2$-chord hence it has no $2$-chords, and furthermore has no chords since $P$ already had none.

      We have no chords with one endpoint in $P'$ and one in $\C$ since these would have to break $v_j x p_i v_k \oplus \rev{P'}$ or be adjacent to $v_j$, which is in violation of Invariant \ref{i:uni:no2Chords}.
      Any $2$-chords with one endpoint in $P'$ and one in $\C$ would have $x$ or $p_i$ as middle vertex.
      However, the first yields a $2$-chord of the candidate path with a higher start range, this is a contradiction.
      And the second can not yield a separating $2$-chord. \fxwarning{TODO or it is just contained, this can happen see back of draft 1.4}

    \begin{figure}[h]
      \centering
      \includegraphics[scale=1]{unifiedAlgo/img/sweep/cases/pEBound}
      \caption{Updating path when $P$ has a separating 2-chord with $v_j$ as end vertex.}
      \label{fig:sweep:pEBound}
    \end{figure}

    \emph{Only other separating 2-chords}
      Find the $2$-chord with the lowest end index, say that this is $j$.
      Let $v_l$ be the shared neighbor on the sweepcycle of $p_{j}$ and $p_{j-1}$.
      The updating path is the right neighbor path of $\cpath|_{v_i, v_l}$. See Figure \ref{fig:sweep:free2chord}.

      Any updating path stops before the end of a separating 2-chord and furthermore contains no chords since $P$ already did not.

      Any chord or $2$-chord with one vertex in the updating path and the other on the old sweepcycle must end to the right of the updating path since the updating path starts at $v_i$, a vertex adjacent to $\pS$.
      Suppose that we have a separating $2$-chord then that would have been a chord of the candidate path. This is in contradiction with the assumption that the candidate path had no chords.
      We also have no chord since such a chord would violate Invariant \ref{i:uni:no2Chords} of the old sweepcycle. Furthermore, the second-to-last vertex of the updating path has no chords since it would break $v_l p_j x p_u$. (see Figure \ref{fig:sweep:free2chord}).

      \begin{figure}[b]
        \centering
        \includegraphics[scale=1]{unifiedAlgo/img/sweep/cases/free2chord}
        \caption{Updating path when $P$ has separating 2-chords none of which have $v_j$ as end vertex.}
        \label{fig:sweep:free2chord}
      \end{figure}

\subsubsection{Updating}
  \label{sss:sweep:update}
  Once we found the updating path $P'$, we can update the sweepcycle with this path.  Let $p_a$ and $p_b$ indicate the two unique vertices of $P'$ that are also part of $\C$. We then let $\cpath|_{P'}$ denote the path $\cpath|_{p_a, p_b}$.
  In this section we describe how to update the sweepcycle with an updating path and we show that the update maintains all sweepcycle invariants (Lemma \ref{lm:sweep:updateMaintainsInvariants}).
  To execute the update we color all interior edge of $\cpath|_{P'} \oplus \rev{P'}$ red and orient them towards $\cpath_|{P'}$.
  We also color all edges of $\cpath|_{P'}$  blue and orient them from lower to higher induces.
  We then update the sweepcycle to $\C'$ by replacing $\cpath|_{P'}$ by $P'$ in $\C$.
  An example of the whole update for an updating path $P'$ can be seen in Figure \ref{fig:sweep:update}.

  \begin{figure}
      \centering
      \begin{subfigure}[b]{0.45 \textwidth}
          \includegraphics[width = \textwidth]{unifiedAlgo/img/sweep/updateBefore.pdf}
          \caption{Before.}
      \end{subfigure}
      ~
      \begin{subfigure}[b]{0.45 \textwidth}
          \includegraphics[width =\textwidth]{unifiedAlgo/img/sweep/updateAfter.pdf}
          \caption{After.}
      \end{subfigure}
      	\caption{The update.}
  \label{fig:sweep:update}
  \end{figure}


  \begin{lemma}
    \label{lm:sweep:updateMaintainsInvariants}
    Updating with a path $P'$ maintains all sweepcycle invariants.
  \end{lemma}
  \begin{proof}
    Invariant \ref{i:uni:SWandSE} remains true. Invariant \ref{i:uni:intVertCond} holds due to the way we colored the edges around the new interior vertices as can be seen in Figure \ref{fig:sweep:update}.
    Furthermore, Invariant \ref{i:uni:redOutgoing} holds because every internal vertex of $P'$ has a left neighbor by Lemma \ref{lm:right:leftNeighborsOfTheRightNeighborPath}.

    To see that Invariants \ref{i:uni:noChords} and \ref{i:uni:no2Chords} hold, note that there can be no chords or $2$-chords with both endpoints in the overlap of the old and new sweepcycle $\C \cap \C'$ by Invariants \ref{i:uni:noChords} and \ref{i:uni:no2Chords}.
     Since the updating path itself also has no irregularities (Lemma \ref{lm:sweep:augNoIregularity}),
    we know any chord or separating $2$-chord $C$ has to have one vertex on $\P$ and one vertex on the unchanged part of old sweepcycle $\C \cap \C'$.
    However, these potential chords can not exist by Lemma \ref{lm:sweep:noConnectingIregularity}.
    Hence, $\C'$ is a valid new sweepcycle.
  \end{proof}

  If after the update the new sweepcycle $\C'$ has no interior vertices we terminate the algorithm,  this is described in Section \ref{sss:terminating}.
  Otherwise, we start the update loop again by finding a new candidate path.

\subsubsection{Terminating the algorithm}
  \label{sss:terminating}
  When the sweepcycle has no more interior vertices, we can not update it anymore.
  However, at this point it is easy to color the remainder of the graph.
  All vertices in $\cpath$ must be adjacent to $\pS$ since $\pS \pW$ and $\pS \pE$ are part of $\C$ by Invariant \ref{i:uni:SWandSE}, $\cpath$ has no chords by Invariant \ref{i:uni:noChords} and $\C$ does not contain any interior vertices.
  All sweepcycle interior edges are adjacent to $\pS$, since otherwise we would have a chord in $\cpath$ (violating Invariant \ref{i:uni:noChords}).

  We color all interior edges of $\C$ red and orient them towards $\cpath$ and the edges in $\cpath$ are colored blue and oriented towards $\pE$. The termination step can be seen in Figure \ref{fig:sweep:terminate}. This last move completes the interior vertex condition for vertices in $\C \sm {\pW, \pS, \pE}$ and also correctly completes the exterior vertex condition.


  \begin{figure}[h]
    \centering
    \begin{subfigure}[b]{0.45 \textwidth}
        \includegraphics[width = \textwidth]{unifiedAlgo/img/sweep/terminateBefore.pdf}
        \caption{Before.}
    \end{subfigure}
    ~
    \begin{subfigure}[b]{0.45 \textwidth}
        \includegraphics[width =\textwidth]{unifiedAlgo/img/sweep/terminateAfter.pdf}
        \caption{After.}
    \end{subfigure}
    \caption{The termination step.}
    \label{fig:sweep:terminate}
  \end{figure}

  \begin{lemma}
    \label{lm:sweep:REL}
    The resulting structure is a regular edge labeling
  \end{lemma}

  \begin{proof}
    After running the whole algorithm the interior vertex condition holds for all vertices in the graph by Invariant \ref{i:uni:intVertCond}. Furthermore, the poles are also colored correctly due to Invariant \ref{i:uni:SWandSE}.
  \end{proof}

\subsubsection{A useful property of this regular edge labelling}
  There still is a useful property of this regular edge labeling left to discuss (Lemma \ref{lm:sweep:NoTwoSplitsAboveEachOther}). Before we can state this property, we first need to introduce fans.

  \mypar{Fans}
    We want to better describe the interior of blue (or red) faces. Every interior edge of such face goes from one boundary path to the other (otherwise its start or end vertex would violate the interior vertex condition or create a face with a boundary path of length one violating Lemma \ref{lm:rel:noBpOfLength1}). We now describe the edges from the split to the merge vertex of $F$.

    \begin{figure}[t]
      \centering
      \includegraphics[scale=.9]{rectangularDuals/img/fans}
      \caption{An example face with fans.}
      \label{fig:uni:fans}
    \end{figure}

    \begin{figure}[b]
      \centering
      \includegraphics[scale=1]{rectangularDuals/img/fanterms}
      \caption{A number of fan-related terms.}
      \label{fig:rect:fanTerms}
    \end{figure}

    Let $u_0 , u_1, \ldots u_n$ be the vertices of the top boundary path of $F$ and $v_0, v_1, \ldots, v_m$ the vertices of the bottom boundary path.
    That is $u_0=v_0$ is the split vertex and $u_n = v_m$ is the merge vertex.
    Since our graph is a triangulation, $u_1v_1$ must be an edge.
    For the second edge in the face we have two options, either $u_1v_2$ or $u_2v_1$, otherwise this edge and the previous one would not form a triangle.
    This argument holds for every subsequent edge, we can either increase the index of the top boundary path or the index of the bottom boundary path.
    Hence, this face is, for the readers that know this term, a \emph{triangle strip}.

    We call a maximal sequence of at least two edges increasing the index on the top boundary path (and thus keeping the index on the upper path fixed) a \emph{bottomfan} or simply \emph{B-fan} and a maximal sequence of at least two edges increasing the index on the bottom boundary path is called a \emph{topfan} or just \emph{T-fan}.
    The \emph{size} of such a fan is the number of edges contained in the sequence. By the definition of a fan it has size of at least $2$.
    We use \emph{fans} to refer to both these \emph{types} of fans (i.e. T- and B-fans).
    We call a fan of size $3$ or larger a \emph{large fan} and a fan of size $2$ a \emph{small fan}.

    In a face we alternately encounter B- and T-fans. If we would have two adjacent fans of the same type, we would just have a single larger fan of that type.
    In Figure \ref{fig:uni:fans} we see a blue face consisting of subsequently a B-fan of size $3$, a T-fan of size 2, a B-fan of size $2$, a T-fan of size $6$, a B-fan of size $3$ and a T-fan of size $3$.


   We introduce some more terminology for fans: \emph{outer edges}, \emph{fan handles} and the \emph{rim} as can be seen in Figure \ref{fig:rect:fanTerms}. The \emph{fan handle} $v$ is the vertex shared by all edges in the fan. Let $H$ be the induced subgraph of vertices incident to the edges in the fan. $H$ contains no edges not belonging to $F$ since these would lead to separating 3-cycles. The \emph{rim} is the path given by $F\sm{v}$.
   \fxnote{Q is this definition of Rim more clear or still unclear?}
   The \emph{outer rim} are the two extreme edges of this path and the \emph{outer edges} are the edges between the fan handles and the extreme vertices of the \emph{rim}.

   A similar discussion can be given for red faces. However then we have \emph{right fans} and \emph{left fans} instead of bottom and top fans.

  \mypar{The property}

    \begin{wrapfigure}{r}{5cm}
      \centering
      \includegraphics[scale=.9]{unifiedAlgo/img/sweep/bottompath.pdf}
      \caption{The bottom path of this splitvertex is given in bold.}
      \label{fig:sweep:bottomPath}
    \end{wrapfigure}

    Before finally discussing the property, we introduce two more definitions.
    A \emph{splitvertex} is a vertex with more than one outgoing blue edge.
    Given a splitvertex $v$, the \emph{bottom} path is the path that comes in trough the first edge in the interval of incoming blue edges in the rotation at $v$ and leaves through the last edge in the interval of outgoing blue edge in the rotation at $v$.
    See Figure \ref{fig:sweep:bottomPath}.

    \begin{lemma}
      \label{lm:sweep:NoTwoSplitsAboveEachOther}
      Let $v$ be any splitvertex. Then the subsequent vertex on the bottom path $w$ can not be the handle of a large topfan.
    \end{lemma}

    \begin{proof}
      There are two possible causes of $v$ being a splitvertex, $v$ is either adjacent to $\pS$ or $v$ is a splitvertex due to a chord.

      If $v$ is a splitvertex because it is adjacent to $\pS$ then since $w$ is on the bottom path it also has to be adjacent to $\pS$ by the definition of the bottom path.
      Hence, $w$ is not the handle of a large topfan.

      If $v$ is a splitvertex due to a chord $v a b x$ we can continue the bottom path past $w$ as a bottom path that eventually goes to $x$ since every chord is evaded by a single path from $v_k$ to $v_\ell$ in the algorithm.
      We denote this extended bottom path by $\P$.
      The situation is depicted in Figure \ref{fig:sweep:botomPathChord}.

      \begin{figure}[b]
        \centering
        \includegraphics[scale=1]{unifiedAlgo/img/sweep/bottompathChord.pdf}
        \caption{The situation in the proof of Lemma \ref{lm:sweep:NoTwoSplitsAboveEachOther}}
        \label{fig:sweep:botomPathChord}
      \end{figure}

      The interior of  $vabx \oplus \rev \P$ has no vertices. Suppose there would be such a vertex . Then, since $\P$ is a bottom path the blue path going trough this vertex has to start at $a$ and end at $b$. But this gives a blue face with only one edge on its bottom boundary path violating Lemma \ref{lm:rel:noBpOfLength1}. Since our graph is a regular edge labeling, $vabx \oplus \rev \P$ has no interior vertices.

      This also implies all interior edges are red (by the definition of bottom path) and thus that $ab$ is blue otherwise we would get a monochromatic triangle.

      Now $w$ can not be connected to any vertex in $\P$ since that would again give a face with a boundary path of length 1 (Lemma \ref{lm:rel:noBpOfLength1}).
      So $w$ can only be connected to $a$ and $b$ and is thus a topfan of size at most $2$.
      (If it is a topfan at all, since we do not consider topfans of size 1 as topfans.)
    \end{proof}
