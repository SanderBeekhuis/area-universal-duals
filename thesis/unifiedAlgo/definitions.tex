%!TEX root = ../thesis.tex

\section{Some Definitions}
\label{s:algo}
Kant and He \cite{Kant1997} were the first to design algorithms that determine a regular edge labeling.

Fusy \cite{Fusy2006} recently developed a different algorithm computing a specific regular edge labeling using a method shrinking a sweepcycle while coloring the outside in accordance with a regular edge labeling.\footnote{The specific regular edge labeling Fusy obtained was the minimal element of the distributive lattice of regular edge labellings.}

All algorithms in this section will have the same core (based on \cite{Fusy2006}). Consisting of shrinking a sweepcycle by so called \emph{valid} paths.\footnote{In Fusy's work he calls these \emph{eligible paths}} But will differ in which valid paths they choose (if there are multiple).

We will start this section with some notation and preliminaries in Subsection \ref{ss:not}. Then we will state the core algorithm and show that it always computes a regular edge labeling in Subsection \ref{ss:core}. Afterwards we show in Subsections \ref{ss:minimal}, \ref{ss:blue} and Section \ref{s:red} how one can adapt the choice of the valid paths to obtain regular edge labellings with certain properties. Namely a the minimal element of the distributive lattices of regular edge labellings and regular edge labeling corresponding to horizontal and vertical rectangular duals.


\subsection{Notation and Preliminaries}
  \label{ss:not}
  \begin{defi}[Interior path]
  We call a path $P$ an internal path of a cycle $C$ if all vertices except the first and last one are in the interior of $C$ and it connects two distinct vertices of $C$
  \end{defi}

  We will use a script $\C$ to indicate the current sweep cycle.
  We will repeatedly only consider the path $\cpath$. In that case we will always order it from $\pW$ to $\pE$. That these edges are always in $\C$ is a result of Invariant \ref{i:SWandSE}.
  paragraph name
  We will let $\P$ denote a interior path. Given such a path of $k$ vertices we will index it's nodes by $p_1, \ldots, p_k$ in such a way that $p_1$ is closer to $\pW$ then $p_k$ is (and thus that $p_k$ is closer to $\pE$ then $p_1$ is).

  Then $p_1$ and $p_k$ indicate the two unique vertices of the walk that are also part of the cycle. We will then let $\restC{\P}$ denote the part of $\cpath$ that is between $p_1$ and $p_k$ (including). $\C_\P$ will denote the cycle we get when we paste $\restC{\P}$ and $\P$.



  \subsubsection{More chords}
    Recall that a \emph{chord} of a path $\P$ is a edge that connects two non-subsequent vertices of that path. Note that this edge can't be in the path. A \emph{k-chord} is a path $\Q$ of $k$ edges that connects two nonsubsequent vertices $v_i, v_j$ of the path such that $\P \cap \Q = \braces{v_i, v_j}$.

    Note that $\P|_{v_i, v_j} \oplus \rev(\Q)$ is a cycle. We call this chord \emph{separating} if this is a separating cycle. \fxnote{Define Rev}


  \subsubsection{The neighbor walk of a path}
    \fxwarning{Note that the same/similar things hold for the left boundary walk}
    During this proof we will frequently use the concept of the right neighbor walk of a path.
    Given a path $P = p_1 \ldots p_k$ in a graph $G$
    The \emph{right neighbor walk} $W$ of $P$ will consist of $p_1$ and the vertices adjacent to $p_{2}$ between $p_1$ and $p_{3}$ in the clockwise rotation at $p_{2}$ followed by the vertices between $p_{2}$ and $p_{4}$ in the rotation at $p_{3}$ and so further until we add the vertices between $p_{k-2}$ and $p_k$ in the rotation around $p_{k-1}$ and finally we finish by adding $p_k$ to $W$.
    We then remove all subsequent duplicates from $W$

    \begin{lemma}
      \label{lm:uni:neighborWalk}
      The right neighbor walk $W$ is a walk.
    \end{lemma}
    \begin{proof}
      Let $w$ and $w'$ be two subsequent vertices in $W$. We will show they are connected. We first consider the case $\braces{w, w'} \cap \braces{p_1, p_k } = \emptyset$.
      Now there are two cases. Either $(a)$ $w$ and $w'$ are vertices adjacent to some $p_i$ an thus subsequent in the rotation at $p_i$  or $(b)$ $w$ was the last vertex adjacent to some $p_i$ and thus $w'$ is the first vertex adjacent to $p_{i+1}$.

    \begin{figure}[h]
      \centering
      \begin{subfigure}[b]{0.5\linewidth}
              \includegraphics[width=\linewidth]{unifiedAlgo/img/walkProofA}
              \caption{}
          \end{subfigure}%
          \begin{subfigure}[b]{0.5\linewidth}
              \includegraphics[width=\linewidth]{unifiedAlgo/img/walkProofB}
              \vspace{1cm}

              \caption{}
          \end{subfigure}

            \caption{The two main cases of the proof showing that $W$ is a walk}
        \label{fig:uni:walkproof}
      \end{figure}

      In case $(a)$ we note that since $w$ and $w'$ are subsequent in the rotation at $p_i$ $ww'$ is an edge by Lemma \ref{lm:prelim:rotationEdge}.

      In case $(b)$ we note that $p_i w$ and $p_i p_{i+1}$ are edges subsequent in clockwise order, hence $wp_{i+1}$ is also an edge. Hence $w$ is the first vertex adjacent to $p_{i+1}$ subsequent to $v_i$ in the clockwise rotation. Thus $w= w'$. They are duplicates and one of them must have been removed.

      Now for the edge cases: Let $x$ be the first vertex adjacent to $p_{i+1}$ and let $y$ be the last vertex adjacent to $p_{j-1}$. $p_i$ and $x$ are vertices adjacent to $p_{i+1}$ subsequent in the clockwise rotation, and hence connected by Lemma \ref{lm:prelim:rotationEdge}. In the same way $y$ and $v_j$ are subsequent vertices in the rotation at $v_n$ and hence connected.

      Hence $\W$ is a walk.
    \end{proof}


     \fxwarning{TODO Change to path iff I4 is satisfied}
    \begin{lemma}
      \label{lm:uni:neighborWalkNoncrossing}
      The right neighbor walk $W$ is a non-crosssing walk.
    \end{lemma}
    \begin{proof}
      Suppose that the right neighbor walk is crossing at a vertex $w= w_i =w_j$. Then one of $w_{j-1}$ and $w_{j+1}$ is in the clockwise interval $[w_{i-1}, w_{i+1} ]$ at the rotation at the triangle containing $w$ is a separating triangle.

      We conclude that $W$ must be a non-crossing walk.
    \end{proof}


 \fxnote{Actullly W rev(P)}
    \fxnote{Change wording to cycle}
    \begin{lemma}
      \label{lm:uni:neighbourwalkNoInteriorVertex}
      The closed non-crosing walk $WP$ has no interior vertex.
    \end{lemma}
    \begin{proof}subsequent
      The interior of $WP$ consists of only triangles with all vertices in $WP$. We can see this from the construction of the neighbor walk. Both cases in Figure \ref{fig:uni:walkproof} add a triangle to the interior with all vertices in $WP$.

      Suppose there is a interior vertex. Then the triangle containing this vertex is a separating triangle.
    \end{proof}


    \begin{lemma}
      \label{lm:uni:neighbourwalkChordFree}
      The left of the of a right neighbor walk is chordfree.
    \end{lemma}
    \begin{proof}
      Suppose that the right neighbor walk $W = w_1 \ldots w_k$  has a chord on the left, say between $w_i$ and $w_j$ with $i< j -1 $. There is a vertex $p_\ell \in P$ on the path such that $w_{i+1}$ is a neighbor of $p_\ell$ to the left of $p_\ell$ Consider now the following non-crossing closed walk $P w_k \ldots w_{j+1} w_j w_i w_{i-1} \ldots w_1$
      (Thick in Figure \ref{fig:uni:neihbourwalkChordFree})this walk has $w_{i+1}$ in its exterior. But then $p_\ell w_{i+1}$ is a crossing edge. Which is forbidden.

      \begin{figure}[h]
        \centering
        \includegraphics[scale=1]{unifiedAlgo/img/neighbourWalkChords}
        \caption{The construction in the proof of Lemma \ref{lm:uni:neighbourwalkChordFree}}
        \label{fig:uni:neihbourwalkChordFree}
      \end{figure}
    \end{proof}

    \fxnote{Alternate proof based on   ($\W$ being oriented from $\pW$ to $\pE$), since if it would lie on the left of $\W$ the vertices $w_{i+1},\ldots, w_{j-1}$ would not have been chosen in the construction of the prefence.}
