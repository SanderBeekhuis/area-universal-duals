%!TEX root = ../thesis.tex

\subsection{The right neighbor path of a path}
\fxinnote{Q: It might be useful to coin a common name for chords and separating 2-chords.}
\thispagestyle{plain}
  \label{ss:rightNeighbour}
  In the sweepcycle step of the algorithm (Section \ref{ss:sweep}) we will use the \emph{right neighbor path} of a path. In this section we show that for any path $P = p_1 \ldots p_k$ with no interior vertices incident to the outer face and without chords or separating 2-chords on the right of the path, the right side neighbors of $P$ are a path $Q$ (Lemma \ref{lm:right:neighborPath}).
  We even show some additional properties hold for $Q$ (Lemma  \ref{lm:right:neighbourwalkNoInteriorVertex} and \ref{lm:right:neighbourwalkChordFree}).
  Similar things also hold for the left neighbors of $P$ but we will not need this for the proof of our algorithm.

  The right side of a path is not yet defined. To do so we start this section by introducing the notion of rotations at a vertex. During the proofs in this section we will also need various types of chords so we subsequently introduce these. Then we finally can start discussing right neighbor paths.

  \mypar{Rotations}
    We assume a fixed embedding for $G$. The \emph{rotation} at a vertex $v$ is the clockwise order of the edges incident to $v$. We will identify these edges with their other endpoints.
    Two vertices $x, y$ are said to be \emph{consecutive} in the rotation at $v$ when the edges $vx$ and $vy$ are consecutive in the rotation.
    We sometimes want to denote number of subsequent vertices, which we call an \emph{interval}, in the rotation. We let $[x,y]$ denote all the vertices in the rotation of $v$ from $x$ to $y$ and we let the \emph{exclusive interval} $(x,y)$ denote the same vertices without $x$ and $y$. In Figure \ref{fig:right:rot} for example the interval $[v,p_{i+1}]$ consists of the vertices $v,w,p_{i+1}$ and $(w,u)$ consists of the vertices $p_{i+1}, x$.

    Given a path $P$ and an interior vertex $p_i \in P$. A neighbor $v \nin \P$ of $p_i$ lies on the \emph{left} of $P$ if it lies in the interval $(p_{i-1}, p_{i+1})$ in the rotation of $p_{i}$. Otherwise, $v$ lies in the interval $(p_{i+1}, p_{i-1})$ in the rotation of $p_i$. In this case $v$ lies on the \emph{right} of $P$.
    We will use the same notion of left and right for edges. That is, an edge $e\nin P$ adjacent to $p_i$ lies to left or right if its other end point lies to the left or right, respectively. In Figure \ref{fig:right:rot} $v$ and $p_i v$ lie on the left of $P$ and $u$ and $p_i u$ lie on the right of $P$.

    \begin{figure}[h]
      \centering
      \includegraphics[scale=1]{unifiedAlgo/img/rightNeighbourwalk/rotation}
      \caption{Rotation at $p_i$.}
      \label{fig:right:rot}
    \end{figure}

  \mypar{Path manipulations}
    With $\rev{P}$ we denote the \emph{reversed path} $p_k \ldots p_1$. We use $\oplus$ to denote the \emph{concatenation} of paths.
    That is, given a second path $Q$ with vertices $q_1 \ldots q_\ell$ and $p_k = q_1$ the path $P \oplus Q$ consists of $p_1 \ldots p_{k-1} q_1 q_2 \ldots q_\ell$.
    Recall that a cycle is simply a path starting and ending at the same vertex.
    Hence if we have two  internally disjoint paths $P, Q$ from $s$ to $t$ then $P \oplus \rev{Q}$ is a cycle.
    Furthermore we use a vertical bar to denote the \emph{restriction} of a path to a certain set of vertices. So $P|_{p_i, p_j}$ with $i<j$ is the subpath of $P$ with vertices $p_i \ldots p_j$.

  \mypar{Chords}
    A \emph{chord} of a path is an edge that connects two non-subsequent vertices. A path without chords is \emph{chordfree}. The path $P$ in Figure \ref{fig:right:chord} has the chord $p_1 p_3$.
    A \emph{k-chord} is a path $Q$ of length $k$ that connects two non-subsequent vertices $p_i, p_j$ of $P$ such that $P \cap Q = \braces{p_i, p_j}$.
    Note that $P|_{p_i, p_j} \oplus \rev{Q}$ is a cycle.
    A ($k$-)chord $Q$ is \emph{separating} if this cycle is separating.
    In Figure \ref{fig:right:chord} there are two $2$-chords, $p_3 u p_5$ and $p_3 v p_5$, but only one of them is separating, namely $p_3 v p_5$.

    \begin{figure}[h]
      \centering
      \includegraphics[scale=1]{unifiedAlgo/img/rightNeighbourwalk/chords.pdf}
      \caption{A path with a chord and a separating 2-chord.}
      \label{fig:right:chord}
    \end{figure}

  \mypar{Right neighbor paths}
    We already mentioned that in the sweepcycle step we use the right neighbor path of a path $P$. Recall that we assumed $P$ has no interior vertices incident to the outer face and no chords or separating 2-chords on the right.
    We first show that every vertex has right neighbors.
    Then we give the procedure for finding the right neighbor path.
    Afterwards we show that the right neighbor path is indeed a path (Lemma \ref{lm:right:neighborPath}) and some additional properties in Lemmas \ref{lm:right:neighbourwalkNoInteriorVertex} and \ref{lm:right:neighbourwalkChordFree}.

    \begin{lemma}
      \label{lm:right:pHasRightNeihgbours}
      Every interior vertex of $P$ has at least one neighboring vertex on the right.
    \end{lemma}

    \begin{proof}
      Suppose that an interior vertex $p_i$ has no neighbor on the right of the path. Then $ \ldots p_{i-1} p_i p_{i+1} \ldots $ is a partial face border. Since $p_i$ is not incident to the outer face $p_i$ must be incident to a face of degree $3$. Thus $p_{i-1} p_i p_{i+1}$ is a face. However, this would imply a chord on the right of $P$ as can be seen in Figure \ref{fig:right:pHasRightNeighbor}. Hence by contradiction $p_i$ must have a neighbor on the right.
    \end{proof}

    \begin{figure}[h]
      \centering
      \includegraphics[scale=1]{unifiedAlgo/img/rightNeighbourwalk/pHasRightNeighbor.pdf}
      \caption{The hypothetical situation that $p_i$ has no right neighbor.}
      \label{fig:right:pHasRightNeighbor}
    \end{figure}

    The right neighbors of the interior vertices of $P$ will form the \emph{right neighbor path} $Q$ of $P$.
    Let us first define a larger list of vertices $Q'$ from which we later remove vertices to get $Q$. $Q'$ will consist of $p_1$ and those vertices adjacent to $p_{2}$ that are in the interval $(p_1, p_3)$ of the clockwise rotation at $p_2$. Followed by the vertices in the interval $(p_2, p_4)$ of the rotation at $p_{3}$. We continue this up to the vertices in the interval $(p_{k-2}, p_k)$ of the rotation at $p_{k-1}$ and finally $p_k$.
    We get $Q$ from $Q'$ by removing all subsequent duplicates from $Q$.
    In Figure \ref{fig:right:neighborPath} an example of a right neighbor path is given.

    \begin{figure}[h]
      \centering
      \includegraphics[scale=1]{unifiedAlgo/img/rightNeighbourwalk/neighborPath.pdf}
      \caption{A right neighbor path.}
      \label{fig:right:neighborPath}
    \end{figure}

    To break the proof that $Q$ is a path into two parts we define a \emph{walk} as a path without the constraint that the same vertex does not occur twice. Hence a walk is a sequence of vertices that are connected to each other but vertices may repeatedly occur.
    \begin{lemma}
      \label{lm:right:neighborWalk}
      The right neighbor path $Q$ is a walk.
    \end{lemma}
    \begin{proof}
      Let us denote the vertices of $Q$ by $q_1 q_2 \ldots q_\ell$.
      Let $q_i$ and $q_{i+1}$ be two subsequent vertices of $Q'$. We will show they are either connected or the same vertex. We first consider the case where $1 < i < \ell-1$.
      Now there are two sub-cases. Either $(a)$ $q_i$ and $ q_{i+1}$ are vertices adjacent to the same vertex $p_j$ an thus subsequent in the rotation at $p_j$ or $(b)$ $q_i$ was the last vertex adjacent to $p_j$ and thus $q_{i+1}$ is the first vertex adjacent to $p_{j+1}$ since by Lemma \ref{lm:right:pHasRightNeihgbours} every interior vertex of $P$ has right neighbors.
      Both cases are depicted in Figure \ref{fig:uni:walkproof}

      In case $(a)$ we note that since $q_i$ and $q_{i+1}$ are subsequent in the rotation at $p_j$ $q_i q_{i+1}$ is an edge since $p_j$ is not incident to the outer face and every interior face of $G$ is a triangle.

      In case $(b)$ we note that $p_i q_i$ and $p_i p_{i+1}$ are edges subsequent in clockwise order, hence $q_{i} p_{i+1}$ is also an edge. Hence $q_i$ is the first vertex adjacent to $p_{i+1}$ subsequent to $v_i$ in the clockwise rotation. Thus $q_{i} = q_{i+1}$, that is $q_i$ and $q_{i+1}$ are duplicates.

      Now for the cases $i=1$ and $i=k-1$. $q_1$ and $q_2$ are vertices adjacent to $p_{2}$ subsequent in the clockwise rotation of ${p_2}$ and hence connected since every interior face is a triangle. In the same way $q_{k-1}$ and $q_k$ are subsequent vertices in the rotation at $q_{k-1}$ and hence connected. This can also be seen in Figure \ref{fig:right:neighborPath}.

      Since all pairs of subsequent vertices in $Q'$ are connected or duplicates the step removing all duplicates from $Q'$ ensures $Q$ is a walk.

      \begin{figure}[b]
        \centering
        \begin{subfigure}[b]{0.5\linewidth}
            \includegraphics[width=\linewidth]{unifiedAlgo/img/walkProofA}
            \caption{ }
        \end{subfigure}%
        \begin{subfigure}[b]{0.5\linewidth}
            \includegraphics[width=\linewidth]{unifiedAlgo/img/walkProofB}
            \vspace{1cm}
            \caption{ }
        \end{subfigure}
        \caption{The two main cases of the proof showing that $W$ is a walk.}
        \label{fig:uni:walkproof}
      \end{figure}
    \end{proof}

    \begin{lemma}
      \label{lm:right:neighborPath}
      The right neighbor path $Q$ is a path
    \end{lemma}
    \begin{proof}
      We already know $Q$ is a path by Lemma \ref{lm:right:neighborWalk}. Hence we only have to show that $Q$ contains no duplicate vertices.

      Suppose that $Q$ has a duplicate vertex $q_i=q_j$ with $i<j$. Then this vertex must have been a neighbor to two different vertices in $P$. Since it can not have been twice while being connected to only one vertex. We denote these vertices $p_\ell$, $p_k$ with $\ell<k$. We are now in the situation of Figure \ref{fig:right:path}.

      By the order in which we added vertices to $Q'$, which is preserved by the removal of vertices when we go to $Q$, we know that any vertices in-between $q_i$ and $q_j$ in $Q$ must be one of the following:
      \begin{enumerate}
        \item Adjacent to $p_\ell$ and in the interval $(q_i, p_{\ell+1})$ in $p_\ell$'s rotation.
        \item Adjacent to one of $p_{\ell+1},  p_{\ell+2},\ldots, p_{k-1}$ and to the right of $P$.
        \item Adjacent to $p_k$ and in the interval $(p_{k-1}, q_j)$ in $p_k$'s rotation.
      \end{enumerate}


      All three cases describe a vertex that lies in the interior of the cycle $q_i p_i p_{i+1} \ldots p_j$. However, since $P$ has no separating $2$-chords on the right this cycle must be empty. Therefore there are no vertices in-between $q_i$ and $q_j$. But $Q$ is a walk and $G$ is simple  so $Q$ has no subsequent duplicates. Hence $Q$ contains no duplicates at all and is thus a path.
    \end{proof}

    \begin{figure}[h]
      \centering
      \includegraphics[scale=1]{unifiedAlgo/img/rightNeighbourwalk/neighborPathisPath.pdf}
      \caption{A hypothetical duplicate vertex.}
      \label{fig:right:path}
    \end{figure}

    \begin{lemma}
      \label{lm:right:neighbourwalkNoInteriorVertex}
      The cycle $P \oplus \rev{Q}$ has no interior vertices.
    \end{lemma}
    \begin{proof}
      In the construction of the right neighbor path both cases in Figure \ref{fig:uni:walkproof} add a triangle to the interior with all vertices of the triangle in $P \oplus \rev{Q}$. Hence, the interior of $P \oplus \rev{Q}$ can be subdivided in a number triangles.
      Suppose there is an interior vertex in the cycle $P \oplus \rev{Q}$. Then the triangle containing this vertex is a separating triangle. Hence $P \oplus \rev{Q}$ has no interior vertices.
    \end{proof}

    \begin{lemma}
      \label{lm:right:leftNeighborsOfTheRightNeighborPath}
      Every interior vertex of a right neighbor path $Q$ has a left neighbor.
    \end{lemma}
    \begin{proof}
      Let $q$ be an interior vertex of $Q$. Consider that $q$ was added as right neighbor of some interior vertex $p$ of $P$. Since $Q$ does not start or end at $p$, $p$ must also be a left neighbor of $q$ in $Q$.
    \end{proof}

    \begin{lemma}
      \label{lm:right:neighbourwalkChordFree}
      The left of a right neighbor path is chordfree.
      %TODO we could probably simplify this proof using the above lemma, I just do not trust it that much.
    \end{lemma}
    \begin{proof}
      Suppose that the right neighbor path $Q = q_1 \ldots q_k$  has a chord on the left, say between $q_i$ and $q_j$ with $i< j -1 $. Then $q_i q_j$ is an interior edge of $P \oplus \rev{Q}$.  There is also vertex $p_\ell \in P$ such that $q_{i+1}$ is a right neighbor of $p_\ell$, hence $p_\ell q_{i+1}$ is an interior edge of $P \oplus \rev{Q}$.  But now we must have a crossing between the edges $q_i q_j$ and $p_\ell q_{i+1}$ since both run in the interior of $P \oplus \rev{Q}$.
      See figure \ref{fig:uni:neihbourwalkChordFree}.
      Since we assume $G$ is planar thee can be no chords on the left of the right neighbor path.

      \begin{figure}[h]
        \centering
        \includegraphics[scale=1]{unifiedAlgo/img/rightNeighbourwalk/neighbourWalkChords}
        \caption{The construction in the proof of Lemma \ref{lm:right:neighbourwalkChordFree.}}
        \label{fig:uni:neihbourwalkChordFree}
      \end{figure}
    \end{proof}
