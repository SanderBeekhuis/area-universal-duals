%!TEX root = ../thesis.tex

\section{Algorithm}
\label{s:algo}
Kant and He \cite{Kant1997} were the first to design algorithms that determine a regular edge labeling. Fusy \cite{Fusy2006} recently developed a different algorithm computing a specific regular edge labeling using a method shrinking a sweepcycle while coloring the outside in accordance with a regular edge labeling.
\mypar{Sweepcycle algorithms}
In such a sweepcycle approach one starts by denoting the outer cycle as the sweepcycle $\C$ and then shrinks this sweepcycle by updating it with \emph{interior paths} of $\C$. An \emph{interior path} is a path connecting two distinct vertices of $\C$ and whose edges are in the interior of $\C$. During this update one maintains some invariants on the structure of both the cycle and its exterior.

After a finite amount of updates the sweepcycle has no more interior vertices. At this point we can complete the algorithm and we obtain a \rel.

\mypar{Our algorithm}
For our algorithm we will assume that $\ext G$ has no separating $4-cycles$. Let $d$ be the highest degree among vertices of $G$ after adding the corners. Then our algorithm will calculate a $d$ sided layout.

Unfortunatly a straight up sweepcycle approach did not work. So instead of that the algorithm will use a sweep cycle as first stage in Section \ref{s:sweep}.
%Consisting of shrinking a sweepcycle by so called \emph{valid} paths.\footnote{In Fusy's work he calls these \emph{eligible paths}}

Before we can dive into this we first need to introduce some more theory on right neighbor walks. We do this in Section \ref{s:rightNeighbour}.

After the sweepcycle we still need to do $3$ postprocessing steps before we obtain a $d$-sided layout.

The second stage of the algorithm will consist of a procedure to get rid of most large topfans. This will be treated in Section \ref{s:fanflip}.

The final stage of the algorithm will consist of flipping edges when blue faces are still to large. This will be described in Section \ref{s:subdiv}.
