%!TEX root = ../thesis.tex

\section{Algorithm}
\label{s:algo}
Kant and He \cite{Kant1997} were the first to design algorithms that determine a regular edge labeling. Fusy \cite{Fusy2006} recently developed a different algorithm computing a specific regular edge labeling using a method shrinking a sweepcycle while coloring the outside in accordance with a regular edge labeling.

In Section \ref{s:fix} we see that corner assignments with a separating $4$-cycle may lead to a $\infty$-sided graph.

Hence for our algorithm we will assume that $\ext G$ has no separating $4-cycles$. Let $d$ be the highest degree among vertices of $G$ after adding the corners. Then the largest possible fan in a \rel is of size $d-3$. Our algorithm will calculate a $d$ sided layout.


Our algorithm will use an adaption of the algorithm by Fusy as a first stage in Section \ref{s:sweep}.  consisting of shrinking a sweepcycle by so called \emph{valid} paths.\footnote{In Fusy's work he calls these \emph{eligible paths}}

Before we can dive into this we first need to introduce some more theory on right neighbor walks. We do this in Section \ref{s:rightNeighbour}.

The second stage of the algorithm will consist of a procedure to get rid of most large topfans. This will be treated in Section \ref{s:fanflip}.

The final stage of the algorithm will consist of flipping edges when blue faces are still to large. This will be described in Section \ref{s:subdiv}.
