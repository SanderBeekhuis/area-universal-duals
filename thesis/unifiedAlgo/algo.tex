%!TEX root = ../thesis.tex

\section{Algorithm}
\thispagestyle{plain}
\label{s:algo}

\mypar{Algorithms for regular edge labellings}
  Kant and He \cite{Kant1997} were the first to design algorithms that determine a regular edge labeling given a graph $G$. Fusy \cite{Fusy2006} recently developed a different algorithm computing a specific regular edge labeling using a method which shrinks a cycle while coloring the exterior of this cycle in accordance with a regular edge labeling. He refers to such a cycle as a \emph{sweepcycle}. Unfortunately his proof of this algorithm is rather concise.

  In this algorithm Fusy starts by denoting the outer cycle of $\ext G \sm{\pN}$ as the sweepcycle $\C$. He then shrinks this sweepcycle by updating it with interior paths of $\C$. During this update he maintains invariants on the structure of both the cycle and its exterior.
  After doing a finite amount of updates the sweepcycle has no more interior vertices. At this point he completes the algorithm and obtains a regular edge labeling.

\mypar{Our algorithm}
  In this section we will present an algorithm providing a $d$-sided rectangular dual for any corner assignment $\ext G$ without separating $4$-cycles, where $d$ is the highest degree among vertices of $G$ in $\ext G$.  Hence we will proof the following theorem.

  \begin{thrm}
  \label{th:dsided}
  Triangulations of the $k$-gon $G$ that have a corner assignment $\ext G$ without separating 4-cycles are $d-1$-sided, where $d$ is the maximal degree of the vertices of $G$ in $\ext G$.
  \end{thrm}

  To describe this algorithm we introduce the notion of right neighbor paths in Section \ref{ss:rightNeighbour}.  We use these right neighbor paths to find a regular edge labeling with some nice properties in Section \ref{ss:sweep}. Afterwards we unfortunately still have to do three post-processing steps. In Section \ref{ss:flipBlueZ} we make sure our regular edge labeling is \emph{vertically one-sided}, that is, that all vertical segments are one-sided. In Section \ref{ss:fanflip} we will remove most occurrences of something called topfans. Then, in Section \ref{ss:subdiv} we can finally subdivide all blue faces that are too large without creating red faces that are too large. This finishes the algorithm.
