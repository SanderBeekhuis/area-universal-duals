%!TEX root = ../thesis.tex
\renewcommand{\Q}{\scr Q}

\section{The algorithm}

We will assume the entire graph has no seperating 3 or 4 cyles. We will show that in this case ...

\subsection{Definitions}
  \subsubsection{The neighbor walk of a path}
    \fxwarning{Note that the same/similar things hold for the left boundary walk}
    During this proof we will frequently use the concept of the left or right neighbor walk of a path.
    Given a path $P = p_1 \ldots p_k$ in a graph $G$
    The \emph{right neighbor walk} $W$ of $P$ will consist of $p_1$ and the vertices adjacent to $p_{2}$ between $p_1$ and $p_{3}$ in the clockwise rotation at $p_{2}$ followed by the vertices between $p_{2}$ and $p_{4}$ in the rotation at $p_{3}$ and so further until we add the vertices between $p_{k-2}$ and $p_k$ in the rotation around $p_{k-1}$ and finally we finish by adding $p_k$ to $W$.
    We then remove all subsequent duplicates from $W$

    \begin{lemma}
      \label{lm:uni:neighborWalk}
      The right neighbor walk $W$ is a walk.
    \end{lemma}
    \begin{proof}
      Let $w$ and $w'$ be two subsequent vertices in $W$. We will show they are connected. We first consider the case $\braces{w, w'} \cap \braces{p_1, p_k } = \emptyset$.
      Now there are two cases. Either $(a)$ $w$ and $w'$ are vertices adjacent to some $p_i$ an thus subsequent in the rotation at $p_i$  or $(b)$ $w$ was the last vertex adjacent to some $p_i$ and thus $w'$ is the first vertex adjacent to $p_{i+1}$.

      The following two situations can also be seen in Figure \ref{fig:uni:walkproof}.

      \begin{figure}[h]
          \centering
          \begin{subfigure}[b]{0.5\linewidth}
              \includegraphics[width=\linewidth]{unifiedAlgo/img/walkProofA}
              \caption{}
          \end{subfigure}%
          \begin{subfigure}[b]{0.5\linewidth}
              \includegraphics[width=\linewidth]{unifiedAlgo/img/walkProofB}
              \vspace{1cm}

              \caption{}
          \end{subfigure}

            \caption{The two main cases of the proof showing that $W$ is a walk}
        \label{fig:uni:walkproof}
      \end{figure}

      In case $(a)$ we note that since $w$ and $w'$ are subsequent in the rotation at $p_i$ $ww'$ is an edge by Lemma \ref{lm:prelim:rotationEdge}.

      In case $(b)$ we note that $p_i w$ and $p_i p_{i+1}$ are edges subsequent in clockwise order, hence $wp_{i+1}$ is also an edge. Hence $w$ is the first vertex adjacent to $p_{i+1}$ subsequent to $v_i$ in the clockwise rotation. Thus $w= w'$. They are duplicates and one of them must have been removed.

      Now for the edge cases: Let $x$ be the first vertex adjacent to $p_{i+1}$ and let $y$ be the last vertex adjacent to $p_{j-1}$. $p_i$ and $x$ are vertices adjacent to $p_{i+1}$ subsequent in the clockwise rotation, and hence connected by Lemma \ref{lm:prelim:rotationEdge}. In the same way $y$ and $v_j$ are subsequent vertices in the rotation at $v_n$ and hence connected.

      Hence $\W$ is a walk.
    \end{proof}


    \begin{lemma}
      \label{lm:uni:neighborWalkNoncrossing}
      The right neighbor walk $W$ is a non-crosssing walk.
    \end{lemma}
    \begin{proof}
      Suppose that the right neighbor walk is crossing at a vertex $w= w_i =w_j$. Then one of $w_{j-1}$ and $w_{j+1}$ is in the clockwise interval $[w_{i-1}, w_{i+1} ]$ at the rotation at $w$. We will denote this vertex by $w'$. It is clear that $w'$ cannot be on the path unless $w'$ is $p_1$ or $p_k$. In this case however we see that $w_{i-1}$ or $w_{i+1}$ respectively couldn't have been part of the path.

      So we continue with $w'$ not on the path. All neighbors of $w$ between $w_{i-1}$ and $w_{i+1}$ in the clockwise rotation are on the path. \fxwarning{TODO make this a lemma}. So we have a series of triangles by Lemma \ref{lm:prelim:rotationEdge}. Now $w'$ must be inside one of these triangles, otherwise we would have a crossing edge (and thus a non-planar graph.) Now the triangle containing $w$ is a separating triangle.

      We conclude that $W$ must be a non-crossing walk.
    \end{proof}


    \fxnote{Actullly W rev(P)}
    \begin{lemma}
      \label{lm:uni:neighbourwalkNoInteriorVertex}
      The closed non-crosing walk $WP$ has no interior vertex.
    \end{lemma}
    \begin{proof}subsequent
      The interior of $WP$ consists of only triangles with all vertices in $WP$. We can see this from the construction of the neighbor walk. Both cases in Figure \ref{fig:uni:walkproof} add a triangle to the interior with all vertices in $WP$.

      Suppose there is a interior vertex. Then the triangle containing this vertex is a separating triangle.
    \end{proof}


    \begin{lemma}
      \label{lm:uni:neighbourwalkChordFree}
      The left of the of a right neighbor walk and the right of the left neighbor walk are chordfree.
    \end{lemma}
    \begin{proof}
      Suppose that the right neighbor walk $W = w_1 \ldots w_k$  has a chord on the left, say between $w_i$ and $w_j$ with $i< j -1 $. There is a vertex $p_\ell \in P$ on the path such that $w_{i+1}$ is a neighbor of $p_\ell$ to the left of $p_\ell$ Consider now the following non-crossing closed walk $P w_k \ldots w_{j+1} w_j w_i w_{i-1} \ldots w_1$
      (Thick in Figure \ref{fig:uni:neihbourwalkChordFree})this walk has $w_{i+1}$ in its exterior. But then $p_\ell w_{i+1}$ is a crossing edge. Which is forbidden.

      \begin{figure}[h]
        \centering
        \includegraphics[scale=1]{unifiedAlgo/img/neighbourWalkChords}
        \caption{The construction in the proof of Lemma \ref{lm:uni:neighbourwalkChordFree}}
        \label{fig:uni:neihbourwalkChordFree}
      \end{figure}
    \end{proof}

    \fxnote{Alternate proof based on   ($\W$ being oriented from $\pW$ to $\pE$), since if it would lie on the left of $\W$ the vertices $w_{i+1},\ldots, w_{j-1}$ would not have been chosen in the construction of the prefence.}

  \subsubsection{More chords}
    Recall that a \emph{chord} of a path $\P$ is a edge that connects two non-subsequent vertices of that path. Note that this edge can't be in the path. A \emph{k-chord} is a path $\Q$ of $k$ edges that connects two nonsubsequent vertices $v_i, v_j$ of the path such that $\P \cap \Q = \braces{v_i, v_j}$.

    Note that $\P|_{v_i, v_j} \oplus \rev(\Q)$ is a cycle. We call this chord \emph{separating} if this is a separating cycle. \fxnote{Define Rev}

    We call a chord \emph{unaivoidable} if on the other side of the path $v_i$ and $v_j$ are connected with an $2-chord$. That is, a $3$-chord on the right of $\P$ is unavoidable if there is a $2-chord$ on the left of $\P$. \fxnote{Figure of this situation, two non-seperating chords}

\subsection{Outline}
  The algorithm will consist of two phases

  \begin{enumerate}
    \item Find a \emph{vertically} 1-sided layout with additional properties
    \item Do additional post-processing steps to make this layout $k$-sided.
  \end{enumerate}

\subsection{Description Phase 1}
  In Phase 1 we execute a sweepcycle algorithm comparable to Fusy's \cite{Fusy2006}.
  During the algorithm we will want to maintain several invariants. The first three are equivalent to those imposed by Fusy. The final two invariants are new and impose a nice structure on the sweepcycle so far.

  \begin{invariants}
    \itemsep=-4pt

    \item \label{i:uni:SWandSE} The cycle $\C$ contains the two edges $\pS \pW$ and $\pS \pE$.
    \item \label{i:uni:noChords} $\cpath$ has no chords
    \item \label{i:uni:intVertCond} All inner edges of $T$ outside of $\C$ are colored and oriented in such that the inner vertex condition holds. %TODO what is the inner vertex condition
    \fxerror{We need to add a partial inner vertex condition}
    \item \label{i:uni:no2Chords} $\C\sm{\pW, \pS, \pE}$ has no separating 2-chords
    \item \label{i:uni:no3chords} $\cpath$ has no avoidable separating 3-chords
  \end{invariants}





  \begin{defi}[Prefence]
  A prefence $\W$ is a interior path of $\C$ starting at $v_i \in \C$ and ending at $v_j \in \C$ a both adjacent to $\pS$
  \begin{enumerate}
   \renewcommand*{\labelenumi}{(P\arabic{enumi})}%
   \renewcommand*{\theenumi}{(P\arabic{enumi})}%
    \item  $\C_\W$ Has no interior vertex
    \label{p:noInteriorVertex}
    \item  $\W$ has no chords on the left     \label{p:Wchordfree}

    \item  $\restC{\W}$ has no chords on the right     \label{p:Cchordfree}

  \end{enumerate}
  \end{defi}

  We do the following
  \begin{enumerate}
    \item Find the right neighbor walk
    \item Evade any future irregularities
    \item Update with this valid path
  \end{enumerate}

  We then repeat this until the sweepcycle does not contain any more interior vertices.

  \paragraph{Find the right neighbor path}
    Let $v_i$ denote all the vertices of $\cpath$ in the following order $\pW =  v_1 \  v_2 \  \ldots v_{n-1} \  v_n = \pE$.
    Some intervals of these vertices will be adjacent to $\pS$. However, they can't be all adjacent to $S$ since then the sweepcycle would be non-separating since we can't have separating triangles. We denote by $v_i$ the last vertex of fist interval of vertices adjacent to $S$ and by $v_j$ the first vertex of the second interval.
    As candidate walk we will take the right neighbor path of $C\sm{S, v_1, \ldots v_{i-1}, v_{j+1}, \dots v_n}$.
    \fxnote{beter notation}

    \begin{lemma}
      \label{lm:uni:isPrefence}
    The collection $W$ described above is a prefence.
    \end{lemma}
    \begin{proof}
    $W$ is a walk by lemma \ref{lm:uni:neighborWalk}. Furthermore it is a path since any NSP would offend Invariant \ref{i:uni:no2Chords} of the sweepcycle.


    We note \ref{p:noInteriorVertex} holds due to Lemma \ref{lm:uni:neighbourwalkNoInteriorVertex}

    We note \ref{p:Wchordfree} holds due to Lemma \ref{lm:uni:neighbourwalkChordFree}.

    We note \ref{p:Cchordfree} holds due to invariant \ref{i:uni:noChords}.
    \end{proof}

    We then orient $\W$ from $v_i$ (the vertex closest to $\pW$)to $v_j$ (the vertex closest to $\pE$) and denote it's vertices by $w_1 \ldots w_k$.


    \subsubsection{Irregularities}
      Now the prefence we found can have several structures we want to avoid
      namely
      \begin{enumerate}
        \itemsep=-4pt
        \item Chords
        \item Separating 2-chords
        \item Separating avoidable $3$-chords
      \end{enumerate}

      All of these structures are on the right of the prefence since the prefence satifies \ref{p:Wchordfree} (no chords on the left) and \ref{p:noInteriorVertex} (no 2- or 3-chords on the left).

      \paragraph{Avoid chords}

      \paragraph{Avoid 2-chords}

      \paragraph{Avoid 3-chords}

    \begin{lemma}
      \label{lm:}
      If the prefence has no chord on the right then it is a valid path.
    \end{lemma}

\subsection{Proofs Phase 1}
