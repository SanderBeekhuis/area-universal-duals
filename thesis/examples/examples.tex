%!TEX root = ../thesis.tex

\section{Example execution} %And worst case scenearios
\label{s:ex}
\thispagestyle{plain}

%\subsection{A small example}
On the four next pages we will give an example execution of our algorithm on a fairly simple graph.
We hope this helps the reader to get some intuition of how the algorithm operates. We consider the graph given in Figure~\ref{fig:ex:simple:1} this graph has as highest degree vertices several vertices of degree 6. Our algorithm should thus give at least a $5$-sided layout, but it actually gives a $2$-sided layout in this case.

The application of the algorithm to the graph is straightforward. Figures \ref{fig:ex:simple:1} to \ref{fig:ex:simple:7} are steps in the sweepline algorithm. This graph does not have any blue $Z$'s or topfans so we can skip these steps of the algorithm.
Then finally the subdivision of large red faces happen in Figure \ref{fig:ex:simple:8}.

In the captions of the subfigures of Figure~\ref{fig:ex:simple} more details about each step can be found.

\begin{figure}
    \centering
    \begin{subfigure}[b]{.9 \textwidth}
      \includegraphics[width=\textwidth]{examples/img/smallExample/smallExample-1}
      \caption{The graph upon which we will execute the algorithm.}
      \label{fig:ex:simple:1}
    \end{subfigure}

    \begin{subfigure}[b]{.9 \textwidth}
      \includegraphics[width=\textwidth]{examples/img/smallExample/smallExample-2}
      \caption{The initial sweepcycle.}
      \label{fig:ex:simple:2}
    \end{subfigure}
    \label{fig:ex:vert}
    \caption{The steps of the algorithm.}
\end{figure}

\begin{figure}
    \centering
    \ContinuedFloat
    \begin{subfigure}[b]{.9 \textwidth}
      \includegraphics[width=\textwidth]{examples/img/smallExample/smallExample-3}
      \caption{Advancing by one update of the sweepcycle.}
      \label{fig:ex:simple:3}
    \end{subfigure}

    \begin{subfigure}[b]{.9 \textwidth}
      \includegraphics[width=\textwidth]{examples/img/smallExample/smallExample-4}
      \caption{Another update step.}
      \label{fig:ex:simple:4}
    \end{subfigure}
    \caption{The steps of the algorithm.}
\end{figure}

\begin{figure}
    \centering
    \ContinuedFloat
    \begin{subfigure}[b]{.9 \textwidth}
      \includegraphics[width=\textwidth]{examples/img/smallExample/smallExample-5}
      \caption{This update the candidate path (dashed) had a chord leading to this smaller update.}
      \label{fig:ex:simple:5}
    \end{subfigure}

    \begin{subfigure}[b]{.9 \textwidth}
      \includegraphics[width=\textwidth]{examples/img/smallExample/smallExample-6}
      \caption{Another update step. This time again without violating chords.}
      \label{fig:ex:simple:6}
    \end{subfigure}
    \caption{The steps of the algorithm.}
\end{figure}

\begin{figure}
    \centering
    \ContinuedFloat
    \begin{subfigure}[b]{.9 \textwidth}
      \includegraphics[width=\textwidth]{examples/img/smallExample/smallExample-7}
      \caption{Terminating the sweepcycle step of the algorithm.}
      \label{fig:ex:simple:7}
    \end{subfigure}
    
    \begin{subfigure}[b]{.9 \textwidth}
      \includegraphics[width=\textwidth]{examples/img/smallExample/smallExample-8}
      \caption{The flips we make during blue face subdivision.}
      \label{fig:ex:simple:8}
    \end{subfigure}
  \caption{The steps of the algorithm.}
  \label{fig:ex:simple}

\end{figure}




%\subsection{A example containing a large red face}
%
%Consider the graph given in Figure~\ref{fig:ex:vert:graph} this graph has as highest degree %vertices two vertices of degree 12. We show how the proposed algorithm handles this graph.
%
%The way in which the algorithm handles this graph is far from ideal leading to a 10-sided face. %While the provided upperbound would lead to (at worst) a 11-sided \fxnote{11 or 12?} face. %Furthermore the approach in this graph is scaleabale as is shown in Figure~\ref{some figure TODO %Show where to scale using dashed lines etc..}
%
%\begin{figure}[h]
%  \centering
%  \includegraphics[width=\textwidth]{examples/img/vertWorstCase/graph}
%  \caption{}
%  \label{fig:ex:vert:graph}
%\end{figure}
%
%The application of the algorithm to this graph is straightforward. Figures \ref{fig:ex:vert:sweep1} %to \ref{fig:ex:vert:sweepfinal} are steps in the sweepline algorithm. This graph does not permit %any topfanflips since all large topfans are either incident to a pole or starting above a split %vertex. \fxnote{In a sense a pole is a splitvertex}
%Then finally the subdivision of large red faces happen in Figures \ref{fig:ex:vert:subdiv1} and %\ref{fig:ex:vert:subdivfinal}.
%
%In the captions of the subfigures of Figure~\ref{fig:ex:vert} more details about each step can be %found.
%
%
%\begin{figure}
%    \centering
%    \begin{subfigure}[b]{.9 \textwidth}
%      \includegraphics[width=\textwidth]{examples/img/vertWorstCase/sweep1}
%      \caption{The initial sweepcycle.}
%      \label{fig:ex:vert:sweep1}
%    \end{subfigure}
%    ~
%    \begin{subfigure}[b]{.9 \textwidth}
%      \includegraphics[width=\textwidth]{examples/img/vertWorstCase/sweep2}
%      \caption{Advancing by one update of the sweepcycle.}
%      \label{fig:ex:vert:sweep2}
%    \end{subfigure}
%    \label{fig:ex:vert}
%    \caption{The steps of the algorithm.}
%\end{figure}
%
%\begin{figure}
%    \ContinuedFloat
%    \begin{subfigure}[b]{.9 \textwidth}
%      \includegraphics[width=\textwidth]{examples/img/vertWorstCase/sweep3}
%      \caption{This update the candidate path had a chord leading to this smaller update step.}
%      \label{fig:ex:vert:sweep3}
%    \end{subfigure}
%    ~
%    \begin{subfigure}[b]{.9 \textwidth}
%      \includegraphics[width=\textwidth]{examples/img/vertWorstCase/sweep4}
%      \caption{Several similar update steps combined into one figure.}
%      \label{fig:ex:vert:sweep4}
%    \end{subfigure}
%    \caption{The steps of the algorithm.}
%\end{figure}
%
%\begin{figure}
%    \ContinuedFloat
%    \begin{subfigure}[b]{.9 \textwidth}
%      \includegraphics[width=\textwidth]{examples/img/vertWorstCase/sweep5}
%      \caption{An update step, this time without violating chords.}
%      \label{fig:ex:vert:sweep5}
%    \end{subfigure}
%    ~
%    \begin{subfigure}[b]{.9 \textwidth}
%      \includegraphics[width=\textwidth]{examples/img/vertWorstCase/sweep6}
%      \caption{Another update step. Note that the original candidate path would have a polebound %2-chord with the edge $\pE \pS$.}
%      \label{fig:ex:vert:sweep6}
%    \end{subfigure}
%    \caption{The steps of the algorithm.}
%\end{figure}
%
%\begin{figure}
%    \ContinuedFloat
%    \begin{subfigure}[b]{.9 \textwidth}
%      \includegraphics[width=\textwidth]{examples/img/vertWorstCase/sweep7}
%      \caption{}
%      \label{fig:ex:vert:sweep7}
%    \end{subfigure}
%    ~
%    \begin{subfigure}[b]{.9 \textwidth}
%      \includegraphics[width=\textwidth]{examples/img/vertWorstCase/sweep8}
%      \caption{}
%      \label{fig:ex:vert:sweep8}
%    \end{subfigure}
%  \caption{The steps of the algorithm.}
%  \label{}
%
%\end{figure}
%
%
%
%\begin{figure}
%    \ContinuedFloat
%    \begin{subfigure}[b]{.9 \textwidth}
%      \includegraphics[width=\textwidth]{examples/img/vertWorstCase/sweepfinal}
%      \caption{}
%      \label{fig:ex:vert:sweepfinal}
%    \end{subfigure}
%    ~
%    \begin{subfigure}[b]{.9 \textwidth}
%      \includegraphics[width=\textwidth]{examples/img/vertWorstCase/subdiv1}
%      \caption{}
%      \label{fig:ex:vert:subdiv1}
%    \end{subfigure}
%  \caption{The steps of the algorithm.}
%  \label{}
%\end{figure}
%
%
%
%\begin{figure}
%    \ContinuedFloat
%    \begin{subfigure}[b]{.9 \textwidth}
%      \includegraphics[width=\textwidth]{examples/img/vertWorstCase/subdivfinal}
%      \caption{}
%      \label{fig:ex:vert:subdivfinal}
%    \end{subfigure}
%  \caption{The steps of the algorithm.}
%  \label{}
%\end{figure}
%
