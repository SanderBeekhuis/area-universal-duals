%!TEX root = thesis.tex

\documentclass[a4paper]{article}
%\usepackage{a4wide}                     % Tell latex to use the room there is on the page
\usepackage{fancyhdr}                   % Nice and shiny headers
\usepackage{paralist}                   % Compact enumerate and itemize
\usepackage{datetime}                   % For formatting dates
\usepackage{makeidx}                    % Needed for the index
\usepackage{hyperref}                   % For inserting PDF options (author etc.)

\usepackage{amsmath}
\usepackage{amsthm}
\usepackage[utf8]{inputenc}
\usepackage{csquotes}
\usepackage[english]{babel}
\usepackage{graphicx}
\usepackage{enumitem}
\usepackage{subcaption}  %ALLOWS SUBFIGURES
\usepackage{wrapfig}
\usepackage{fnpct}
\usepackage[draft]{fixme}
\fxsetup{theme = color}
\definecolor{fxnote}{rgb}{0.0000, 0.6000,0.0000}
\definecolor{fxwarning}{rgb}{1.0000,0.5490,0.0000}
\definecolor{fxerror}{rgb}{1.0000,0.2706,0.0000}
\definecolor{fxfatal}{rgb}{1.0000,0.0000,0.0000}
\newcommand{\fxinnote}[1]{\fxnote[inline, nomargin]{#1}}

\usepackage[backend=biber, giveninits =true, isbn=false, url=false, maxbibnames=100]{biblatex}
\usepackage{hyperref}

\usepackage{accents}





% For the TUe style...
% Use the package 'textpos' with the option 'absolute' to place
% a textbox a given distance from the edge of the page
\usepackage[absolute]{textpos}

\usepackage{listings}                   % For code formatting
\lstset{
% From http://en.wikibooks.org/wiki/LaTeX/Packages/Listings
% Added settings using http://ftp.snt.utwente.nl/pub/software/tex/macros/latex/contrib/listings/listings.pdf
language=XML,                    % choose the language of the code
basicstyle=\footnotesize,        % the size of the fonts that are used for the code
%numbers=left,                    % where to put the line-numbers
%numberstyle=\footnotesize,       % the size of the fonts that are used for the line-numbers
%stepnumber=5,                    % the step between two line-numbers. If it's 1 each line will be numbered
%numbersep=5pt,                   % how far the line-numbers are from the code
%backgroundcolor=\color{white},   % choose the background color. You must add \usepackage{color}
showspaces=false,                % show spaces adding particular underscores
showstringspaces=false,          % underline spaces within strings
%showtabs=false,                  % show tabs within strings adding particular underscores
frame=single,	                 % adds a frame around the code
tabsize=2,	                     % sets default tabsize to 2 spaces
captionpos=b,                    % sets the caption-position to bottom
breaklines=true,                 % sets automatic line breaking
breakatwhitespace=false,         % sets if automatic breaks should only happen at whitespace
%escapeinside={\%*}{*)}           % if you want to add a comment within your code
deletekeywords={Timestamp},      % We don't use timestamp as a keyword in our case...
morekeywords={encoding, log, trace, event, string, boolean, float, int},
keywordstyle=\color[rgb]{0,0,1},    %
commentstyle=\color[rgb]{0.133,0.545,0.133}, %
stringstyle=\color[rgb]{0.627,0.126,0.941},  %
}


%
% Layout
%

% header!

%Common header attributes - see also http://en.wikibooks.org/wiki/LaTeX/Page_Layout
\fancyfoot[RE,LO]{\doctitle}
\fancyfoot[C]{}
%\fancyfoot[RE,LO]{\me}
\fancyfoot[LE,RO]{\thepage}
\renewcommand{\footrulewidth}{0.4pt} %thickness of the decorative lines on the footer

% Headers for the pages that start a chapter, also for the TOC
\fancypagestyle{plain}{
\fancyhead[LE,RO]{}
\fancyhead[LO,RE]{}
\renewcommand{\headrulewidth}{0pt} %thickness of the decorative lines on the header
}

%Header for sections
\fancypagestyle{FooBar}{%
    \fancyhead{}
    \renewcommand{\headrulewidth}{1pt}
}

% Headers for the other pages
\pagestyle{fancy}{
\fancyhead[LE,RO]{\slshape \leftmark} % \rightmark = represent section heading
\fancyhead[LO,RE]{\slshape } % \leftmark = represent chapter heading
}






%Theorems
\newtheorem{thrm}{Theorem}
\newtheorem{con}[thrm]{Conjecture}
\newtheorem{lemma}[thrm]{Lemma}
\newtheorem{prop}[thrm]{Proposition}
\newtheorem{remark}[thrm]{Remark}

\theoremstyle{definition}
\newtheorem*{defi}{Definition}

%%BeginIpePreamble
\usepackage{amsmath}
\usepackage{amssymb}
\usepackage{amsopn}

\newcommand{\scr}[1]{\mathcal{#1}}
\newcommand{\Z}{\mathbb{Z}}
\newcommand{\F}{\mathbb{F}}
\newcommand{\R}{\mathbb{R}}
\newcommand{\N}{\mathbb{N}}
%\newcommand{\Q}{\mathbb{Q}}


%Operators
\newcommand{\id}{\operatorname{Id}}



%braces etc
\newcommand{\braces}[1]{\left\lbrace {#1} \right\rbrace}
\newcommand{\sqbr}[1]{\left\lbrack {#1} \right\rbrack }
\newcommand{\abs}[1]{\left\lvert {#1} \right\rvert }
\newcommand{\ceil}[1]{\left\lceil{ #1 } \right\rceil}
\newcommand{\floor}[1]{\left \lfloor {#1}\right\rfloor}
\newcommand{\parens}[1]{\left( {#1} \right)}


%utility
\newcommand{\inv}[1]{{#1}^{-1}}
\newcommand{\half}{\frac{1}{2}}
\newcommand{\third}{\frac{1}{3}}
\newcommand{\goes}{\rightarrow}
\newcommand{\nin}{\not \in}
\newcommand{\sm}[1]{\setminus \braces{#1} }

%vectors and matrices
\newcommand{\zerov}{\vec{0}}
\newcommand{\onev}{\vec{1}}

\newcommand{\twovec}[2]{\parens{ \begin{array}{c}#1 \\ #2\end{array} }}
\newcommand{\threevec}[3]{\prens{ \begin{array}{c}#1 \\ #2\\#3 \end{array} }}
\newcommand{\fourvec}[4]{\parens{ \begin{arr\newcommand{\ifftext}{if and only if }ay}{c}#1 \\ #2\\#3\\#4 \end{array} }}
\newcommand{\twomatrix}[4]{\parens{\begin{array}{cc}#1 & #2 \\ #3 & #4 \end{array}  }}
\newcommand{\twodiagmatrix}[2]{\parens{\begin{array}{cc}#1 & 0 \\ 0 & #2 \end{array}  }}

%%%%THIS THESIS
\newcommand{\intplus}{\operatorname{Int^{+}}}
\newcommand{\interior}{\operatorname{Int}}
\newcommand{\spl}{\operatorname{split}}
\newcommand{\mrg}{\operatorname{merge}}


\newcommand{\ext}[1]{\bar{#1}}
\newcommand{\tightext}[1]{\bar{#1}_t}
\newcommand{\dualgraph}[1]{\G(#1)}
\newcommand{\extdualgraph}[1]{\G_{\scr E}(#1)}

\newcommand{\mypar}[1]{{\bfseries #1.}}
\newcommand{\rev}[1]{\accentset{\leftharpoonup}{#1}}



\newcommand{\W}{\scr W}
\renewcommand{\P}{\scr P}
\newcommand{\C}{\scr C}
\newcommand{\Q}{\scr Q}
\renewcommand{\L}{\scr L}
\newcommand{\I}{\scr I}

\newcommand{\G}{\scr G}
\newcommand\restrict[1]{\raisebox{-.5ex}{$|$}_{#1}}
\newcommand{\restC}[1]{\ensuremath{\C\restrict{#1}}}

%p is for pole
\newcommand{\pN}{\mathrm{N}}
\newcommand{\pS}{\mathrm{S}}
\newcommand{\pE}{\mathrm{E}}
\newcommand{\pW}{\mathrm{W}}

\newcommand{\cpath}{\C \setminus \braces{\pS}} %cycle path

\newcommand{\rel}{\text{regular edge labeling }}

%%EndIpePreamble


%invariant environment
\newenvironment{invariants}{%
  \refstepcounter{thrm}%
  \paragraph{Invariants~\theprop}%
  \renewcommand*{\theenumi}{\theprop\,(I\arabic{enumi})}%
  \renewcommand*{\labelenumi}{(I\arabic{enumi})}%
  \enumerate
}{%
  \endenumerate
}

%bib stuff
\bibstyle{plain}
